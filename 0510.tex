
\newpage

\vspace*{\fill}
\noindent \large Attenzione: da questo punto in poi, le dispense sono state scritte durante la sessione d'esame. Per questo motivo, gli autori non hanno potuto dedicarvi il tempo e l'attenzione che un lavoro di questo tipo richiede. Crediamo che possano essere comunque più precise e complete degli appunti presi in classe, ma non garantiamo nulla. Appena possibile le dispense verranno ricontrollate e rese definitive.
\vspace*{\fill}
\newpage


%%%%lezione 10 maggio%%%%

Lezione del 10/05, ultima modifica 10/06, Andrea Gadotti\\
\\
Il teorema di fattorizzazione costituisce un criterio per l'individuazione, se esiste, di una statistica sufficiente.
\\
\\
\textbf{Esempio} Sia $(X_1,...,X_n)$ da $N(\mu,\sigma^2)$.
\begin{enumerate}
\item [(a)] se $\mu$ è noto, allora cerco una statistica sufficiente per $\sigma^2$.\\
$T_n (X_1,...,X_n) = \displaystyle\sum_{i=1}^n (X_i-\mu)^2$
\item [(b)] Se $\sigma^2$ è noto, allora cerco una statistica sufficiente per $\mu$.
\item [(c)] Se $\mu$ e $\sigma^2$ non sono noti, allora cerco una statistica congiuntamente sufficiente per $\mu$ e $\sigma^2$.\\
$S_n (X_1,...,X_n) = \left( \sum_{i=1}^n X_i, \sum_{i=1}^n (X_i - \overline{X}_n^2) \right)$.
\end{enumerate}

Risulta quindi evidente che il concetto di statistica sufficiente è \emph{problem dependent}.
\\
\begin{esempio} Sia $(X_1,...,X_n)$ da $U(\theta, \theta +1)$ con $\theta \in \mathbb{R}$. Vogliamo trovare una statistica sufficiente per $\theta$.\\
Calcoliamo innanzitutto la funzione di verosimiglianza:
\begin{eqnarray}
L(\theta;\underline{x}) &=& \displaystyle\prod_{i=1}^n \mathbbm{1}_{[\theta; \theta +1]}(x_i)\\
&=& \displaystyle\prod_{i=1}^n \mathbbm{1}_{\mathbb{R}}(x_i) \mathbbm{1}_{[\theta; \theta +1]}(x_{(n)}) \mathbbm{1}_{[\theta; \theta +1]}(x_{(1)})\\
&=& \displaystyle\prod_{i=1}^n \mathbbm{1}_{\mathbb{R}}(x_i) \mathbbm{1}_{[X_{(n)}-1; X_{(n)}]}(\theta)
\end{eqnarray}
Notiamo che l'ultima espressione è il prodotto di due funzioni dove la prima dipende solo dal campione, mentre la seconda dipende sia dal parametro $\theta$ che dal campione. Quindi grazie al teorema di fattorizzazione di Neyman abbiamo che $(X_{(1)},X_{(n)})$ è statistica sufficiente per $U$.\\
Nota: poiché $U_{(\theta, \theta +1)}$ non appartiene alla famiglia regolare non è necessariamente vero che \emph{dimensione statistica sufficiente = dimensione vettore parametri}. Infatti in questo caso abbiamo $2 \neq 1$.
\end{esempio}

\begin{oss} Spesso vediamo il campione casuale come diverso da una statistica, che è funzione del campione. Ma esso porta in sè tutta l'informazione disponibile relativamente al vettore parametrico. Anche il campione stesso è una statistica, in particolare è una statistica sufficiente, l'unico problema è che non è per nulla "sintetico".
\end{oss}

\textbf{Problema:} esiste una statistica sufficiente che sia migliore (ovvero più sintetica) delle altre, a parità di informazione contenuta circa $\theta$? In effetti una tale statistica esiste e viene detta \textit{statistica sufficiente minimale}.