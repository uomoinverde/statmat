\documentclass[12pt]{article}

\title{Statistica Matematica}
\author{Git}
\date{}		

\usepackage{amsmath}%
\usepackage{bbm}%

\newtheorem{corollario}{Corollario}
\newtheorem{definizione}{Definizione}
\newtheorem{esempio}{Esempio}
\newtheorem{exercise}{Exercise}
\newtheorem{proposizione}{Proposizione}
\newtheorem{teorema}{Teorema}

\begin{document}
\maketitle


\section{Introduzione}
\subsection{Funzione generatrice dei momenti}
\subsection{Famiglia esponenziale a k parametri}
\subsection{Trasformazioni di variabili casuali}
\paragraph{Discrete}
\begin{teorema}
Sia X una vc con funzione di massa $f_X(x)=P(X=x)$,  sia W=h(X) una nuova vc. Allora $$
P(W=w)=\sum_{\{x\in (supporto di X):h(x)=w\}}P(X=x)$$
\end{teorema}
\begin{esempio}
Sia X $\sim$ b(n,p) con $f_X(x,p)=\binom{n}{p}p^x(1-p)^{n-x}\mathbbm{1}_{0,1,...,n}(x)$, n noto e $p\in(0,1)$.
Considero quindi W=n-X. Come si distribuisce W? 
$$P(W=w)=P(X=n-w)=\binom{n}{n-w}p^{n-w}(1-p)^w\mathbbm{1}_{0,1,...,n}(w)$$
\end{esempio}
\begin{esempio}
X vc...
\end{esempio}
\paragraph{Assolutamente continue}

\begin{teorema}
Sia X una variabile casuale (ass continua) con funzione di densità $f_X(x)$ e sia W=h(X), ove h è una funzione monotona. 
Supponiamo inoltre che $f_X(x)$ sia continua sul supporto di X e che $h^{-1}(w)$ abbia derivata continua sul supporto di W. Allora
%$$f_W(w)=f_X(h^{-1}(w))\left.\left|\frac{d}{dw}h^{-1}(w)\right|\mathbbm{1}_{A_W}(w)$$
\end{teorema}
\begin{esempio}
Standardizzazione di una vc normale:
\end{esempio}
\begin{teorema}
Se W=h(X) ove h è monotona a tratti (un numero finito k di tratti) e valgono le condizioni del teorema precedente, allora
$$f_W(w)=\sum_{n=1}^k f_X(h_n^{-1}(w))\left|\frac{d}{dw}h_n^{-1}(w)\right|\mathbbm{1}_{A_W}(w)$$
\end{teorema}
\begin{esempio}
CHI-QUADRATO:
\end{esempio}
\subsection{Convergenze}
\paragraph{Convergenza in probabilità}
\begin{definizione}

Sia $\{X_n\}_{n\in\mathbbm{N}}$ una successione di variabili casuali e sia X un'altra variabile casuale, tutte definite sullo stesso spazio campionario.
Diciamo che $X_n$ converge in probabilità a X se
% $\forall\varepsilon>0$ $$\lim_{n \to \infty}P(|X_n-X|\geqslant\varepsilon)=0$$
\end{definizione}
\paragraph{Convergenza in distribuzione}

\end{document}