\documentclass[11pt,a4paper,oneside]{book}

\usepackage[T1]{fontenc}
\usepackage[utf8]{inputenc}
\usepackage[italian]{babel}

\title{Appunti del corso di Statistica Matematica}
\author{Gadotti Andrea, Nardin Michele, Peruzzetto Marco}
\date{}

\usepackage{amsmath}%
\usepackage{bbm}%
\usepackage{longtable}
\usepackage{amsthm}
\usepackage{amscd}
\usepackage{amssymb}
\usepackage{amsfonts}
\usepackage{amsmath}
\usepackage{mathtools}
\usepackage{enumitem}
\usepackage{vmargin}
\usepackage[scale=3]{ccicons}
\usepackage{hyperref}
%\usepackage{graphicx}


\theoremstyle{plain}
\newtheorem{thm}{Teorema}[chapter]
\newtheorem{cor}{Corollario}[thm]
\newtheorem{corollario}{Corollario}
\newtheorem{prp}{Proposizione}
\newtheorem{lem}{Lemma}
\newtheorem{proposizione}{Proposizione}
\newtheorem{teo}{Teorema}
\newtheorem*{theorem}{Teorema}
\newtheorem*{lemma}{Lemma}
\theoremstyle{definition}
\newtheorem{dfn}{Definizione}[chapter]
\newtheorem{definizione}{Definizione}
\theoremstyle{remark}
\newtheorem{oss}{Osservazione}
\newtheorem{ese}{Esempio}
\newtheorem{esempio}{Esempio}
\newcommand{\ud}{\,\mathrm{d}}
\DeclareMathOperator{\im}{im}
\DeclareMathOperator{\ri}{Risk}
\DeclareMathOperator{\ggt}{ggT}
\DeclareMathOperator{\mse}{MSE}
\DeclareMathOperator{\eendd}{End}
\DeclareMathOperator{\tr}{tr}
\DeclareMathOperator{\lo}{Loss}
\DeclareMathOperator{\dom}{Dom}
\DeclareMathOperator{\id}{id}
\DeclareMathOperator{\bi}{Bin}
\DeclareMathOperator{\sign}{sign}
\DeclareMathOperator{\uni}{Unif}
\DeclareMathOperator{\rg}{rg}
\DeclareMathOperator{\homm}{Hom}
\DeclareMathOperator{\iso}{Iso}
\DeclareMathOperator{\asin}{asin}
\DeclareMathOperator{\acos}{acos}
\DeclareMathOperator{\supp}{supp}
\DeclareMathOperator{\var}{Var}
\DeclareMathOperator{\ar}{arg}
\DeclareMathOperator{\eff}{eff}
\DeclareMathOperator{\cova}{Cov}

\newcommand\restr[2]{{% we make the whole thing an ordinary symbol
  \left.\kern-\nulldelimiterspace % automatically resize the bar with \right
  #1 % the function
  \vphantom{\big|} % pretend it's a little taller at normal size
  \right|_{#2} % this is the delimiter
  }}

\providecommand{\abs}[1]{\lvert#1\rvert}
\providecommand{\Abs}[1]{\bigg\lvert#1\bigg\rvert}
\providecommand{\norm}[1]{\lVert#1\rVert}



\begin{document}
\setpapersize{A4}
\setmarginsrb{35mm}{20mm}{25mm}{35mm}%
             {0mm}{10mm}{0mm}{10mm}
             
\maketitle

\pagenumbering{gobble}
\null
\vfill
\noindent \ccbysa  \\
\\
\textbf{Quest'opera è distribuita con Licenza Creative Commons Attribuzione - Condividi allo stesso modo 3.0 Italia.} \\
Per leggere una copia della licenza visita il sito web \url{http://creativecommons.org/licenses/by-sa/3.0/it/} o spedisci una lettera a Creative Commons, 171 Second
Street, Suite 300, San Francisco, California, 94105, USA.


\newpage
\pagenumbering{arabic}

\tableofcontents
\part{Prima parte del corso}
%%%%lezione 18 febbraio%%%%
\chapter{Introduzione}
In questa prima sezione vengono presentati i richiami di teoria della probabilità, affrontati nelle primissime lezioni del corso.

\section{Funzione generatrice dei momenti}
Lezione del 18/02, ultima modifica 09/04, Andrea Gadotti

\begin{dfn}[fgm]
Sia \(X\) una variabile casuale, discreta o assolutamente continua.
Se esiste $t_0 \in \mathbb{R}_{+}$ tale per cui \(\mathbb{E}(e^{tX}) < +\infty\) per ogni \(t \in (-t_0 , t_0)\), chiameremo la funzione 
\begin{equation}
  M_X := \mathbb{E}(e^{tX})
\end{equation}
\emph{funzione generatrice dei momenti} di \(X\).
\end{dfn}

\subsection{Esempi}
\begin{description}
\item[Bernoulli] Sia \(X \sim b(1,p)\), con \(p \in (0,1)\). Si ha:
\begin{equation*}
  M_X (t) = \mathbb{E}(e^{tX})
          = \sum_{x=0}^1 e^{tx} \mathbb{P}(X=x)
          = \sum_{x=0}^1 e^{tx} p^x (1-p)^{1-x}
          = p e^t + (1-p).
\end{equation*}

\item[Poisson] Sia \(X \sim \mathcal{P}(\lambda)\), con \(\lambda \in \mathbb{R}_+\). Si ha:
\begin{equation*}
  M_X (t) = \mathbb{E}(e^{tX}) 
          = \sum_{x=0}^{+\infty} e^{tx} \frac{e^{-\lambda} \lambda^x}{x!}
          = e^{\lambda (e^t -1)}.
\end{equation*}

\item[Gamma] Sia \(X \sim \mathcal{G}(\alpha, \beta)\): allora, la densità di \(X\) è data da 
\begin{equation*}
  f_X (x; \alpha, \beta) := \frac{1}{\Gamma (\alpha) \beta^{\alpha}} x^{\alpha -1} e^{-\frac{1}{\beta} x} \mathbb{I}_{\mathbb{R}_+}(x), \quad
  \alpha, \beta \in \mathbb{R}_+
\end{equation*}
dove \(\Gamma{}\) indica la funzione di Eulero
\begin{equation*}
  \Gamma(\alpha) := \int_0^{+\infty} \! x^{\alpha -1} e^{-x} \mathrm{d}x,
\end{equation*}
la quale è tale che \(\Gamma(\alpha) = (\alpha - 1)!\) se \(\alpha \in \mathbb{N}\).

La generatrice dei momenti di \(X\) risulta essere: 
\begin{align*}
  M_X (t)	&= \mathbb{E}(e^{tX})
           = \int_0^{+\infty} \! e^{tx} \frac{1}{\Gamma (\alpha) 
            \beta^{\alpha}} x^{\alpha -1} e^{-\frac{1}{\beta} x} \mathrm{d}x \\
          &= \frac{1}{\Gamma(\alpha)\beta^{\alpha}}
             \int_0^{+\infty} e^{-\tau}
             \frac{\beta^{\alpha-1}}{(1-\beta t)^{\alpha-1}}
             \tau^{\alpha-1}
             \frac{\beta}{1-\beta t}\mathrm{d}\tau,\qquad \tau := x\left(\frac{1}{\beta} - t\right) \\
          &= \frac{1}{(1-\beta t)^{\alpha}}
             \frac{\int_0^{+\infty} \! \tau^{\alpha-1} e^{-\tau} \mathrm{d}\tau}%
             {\Gamma(\alpha)}
           = \frac{1}{(1-\beta t)^{\alpha}}, \qquad t < \frac{1}{\beta}.
\end{align*}
\end{description}

\paragraph{Momenti di una variabile casuale}

\begin{definizione}
Se una variabile casuale ammette FGM derivabile infinite volte in un intorno di $t=0$ e se tutti i suoi momenti sono finiti, allora definiamo il momento di ordine $s$ non centrato: 
$$\mu'_s := \mathbbm{E} (X^s) = \frac{d^s}{dt^s} M_X (t) \! \mid_{t=0}$$ 
Il momento di ordine $s$ centrato in $a \in \mathbbm{R}$ è: 

$$\mu_s (a) := \mathbbm{E} ((X-a)^s)$$
Ovvero $\mu'_s = \mu_s (0)$. E' chiaro che $\mu'_1 = \mathbbm{E} (X)$.
Chiameremo infine momento di ordine $s$ centrato (senza specificare altro, intenderemo centrato in $\mu'_1$):

$$\mu_s := \mathbbm{E} ((X-\mu'_1)^s)$$
\end{definizione}

\begin{teo}
Dall'ultima definizione si vede facilmente (sviluppando l'elevamento a potenza) che vale la seguente relazione tra momenti centrati e non:
$$\mu_s = \mathbbm{E} ((X-\mu'_1)^s) = \displaystyle\sum_{m=0}^s (-1)^m \binom{s}{m} \mu'_{s-m} (\mu'_1)^m$$
\end{teo}
Osserviamo che $\mu_2 = \mathbbm{E} ((X-\mu'_1)^2) = Var(X) = \mathbbm{E} (X^2) - (\mathbbm{E} (X))^2 = \mu'_2 - (\mu'_1)^2$


\begin{teo}
Date due (o più) v.c. $X$ e $Y$ indipendenti aventi $f$ densità / $f$ massa $f_X$ e $f_Y$ e fgm $M_X(t)$ e $M_Y(t)$ rispettivamente, allora si ha

$$M_{X+Y}(t) = M_X(t) M_Y(t)$$
\end{teo}

\begin{teo}
Siano $X$ e $Y$ v.c. con funzioni di ripartizione $F_X(x)$ e $F_Y(y)$ rispettivamente. Siano $M_X(t)$ e $M_Y(t)$ le fgm di $X$ e $Y$. Se $M_X(t) = M_Y(t)$ per ogni $t$ in un intorno dell'origine, allora 

$$X \stackrel{d}{=} Y$$
\end{teo}

\begin{oss}
Il teorema appena visto ci dice sostanzialmente che, se esiste, la fgm caratterizza la distribuzione della corrispondente v.c.
\end{oss}
\noindent \textbf{Esempio}
Siano $(X_1,...,X_n)$ risultati della replicazione di un esperimento casuale dicotomico $(X_i \sim b(1,p))$. Vogliamo trovare la distribuzione di $S_n := \displaystyle\sum_{i=1}^n X_i$. Calcoliamo quindi la sua fgm:

$$M_{S_n}(t) = \mathbbm{E} \left( {e^{t S_n}} \right)  = \mathbbm{E} \left( {e^{t \sum_{i=1}^n X_i}} \right) $$
$$\stackrel{TEO 2}
{=} \displaystyle\prod_{i=1}^n \mathbbm{E}({e^{tX_i}}) = \displaystyle\prod_{i=1}^n M_{X_i}(t) = \displaystyle\prod_{i=1}^n (p e^t + (1-p)) = (p e^t + (1-p))^n$$ ovvero $S_n$ è distribuita come $b(n,p)$ per il Teorema 2.

\noindent \textbf{Esercizio}
Ripetere il calcolo precedente supponendo $X_i \sim P(\lambda), \forall i$.

\subsection{Famiglia Esponenziale a $k$ parametri}
Una famiglia di $f$ densità / $f$ massa è detta essere una Famiglia Esponenziale a $k$ parametri $\theta_1,...,\theta_k$ se la corrispondente $f$ densità / $f$ massa (che è indicizzata da $\theta_1,...,\theta_k$) può essere scritta come

$$f_X(x;\theta) = C^*(x) D^*(\theta) \exp \lbrace \displaystyle\sum_{m=1}^k A_m(\theta) B_m (x) \rbrace$$

dove $C^*(x)$ è una funzione della sola $x$, $D^*(\theta)$ è una funzione del solo $\theta$, $A_m(\theta)$ è una funzione del solo $\theta$ e $B_m(x)$ è una funzione della sola $x$.

\noindent \textbf{Esempi}
\begin{enumerate}
\item 
$X \sim G(\alpha, \beta) \Longrightarrow f_X(x; \alpha, \beta) = \frac{1}{\Gamma(\alpha) \beta^\alpha} x^{\alpha -1} e^{- \frac{1}{\beta} x} \mathbbm{1}_{\mathbbm{R}^+}(x)$, $\alpha >0$, $\beta >0$
$\mathbbm{1}_{\mathbbm{R}^+}$ è detto supporto della distribuzione.
Quindi possiamo riscrivere $f_X(x; \alpha, \beta)$ come
$$f_X(x; \alpha, \beta) = \frac{1}{\Gamma(\alpha) \beta^\alpha} \mathbbm{1}_{\mathbbm{R}^+}(x) exp((\alpha -1) ln(x) - \frac{1}{\beta} x)$$
e quindi ponendo $D^*(\alpha, \beta) := \frac{1}{\Gamma(\alpha) \beta^\alpha}$, $C^*(x) := \mathbbm{1}_{\mathbbm{R}^+}(x)$, $A_1(\alpha, \beta) := (\alpha -1)$, $B_1(x) := ln(x)$, $A_2(\alpha, \beta) := - \frac{1}{\beta}$ e $B_2(x) := x$, otteniamo $G(\alpha, \beta)$ come famiglia esponenziale con $k=2$.

\item $X \sim b(n,p) \Longrightarrow f_X(x; n,p) = \binom{n}{x} p^x (1-p)^{n-x} \mathbbm{1}_{\lbrace 0,1,...,n \rbrace}(x)$ con $n \in \mathbbm{N}$ noto.
Quindi possiamo riscrivere $f_X(x; n,p)$ come
$$f_X(x; n,p) = \binom{n}{x} \mathbbm{1}_{\lbrace 0,1,...,n \rbrace}(x) (1-p)^n exp(x \ln(\frac{p}{1-p}))$$ con $\frac{p}{1-p}$ detto odd ratio o parametro naturale della famiglia esponenziale.\\
Quindi ponendo $D^*(p) := (1-p)^n$, $C^*(x) := \binom{n}{x} \mathbbm{1}_{\lbrace 0,1,...,n \rbrace}(x)$, $A_1(p) := ln(\frac{p}{1-p})$, $B_1(x) := x$, otteniamo $b(n,p)$ come famiglia esponenziale con $k=1$.

\item X vc con $f_X(x,\vartheta)=\frac{e^{1-x/\vartheta}}{\vartheta}\mathbbm{1}_{(\vartheta,\infty)}(x)$: la distribuzione di X non appartiene a famiglia esponenziale.
Il fatto che il supporto di $f_X$ dipenda dal parametro $\vartheta$ NON permette a $f_X$ di appartenere ud una famiglia esponenziale!
\end{enumerate}

\begin{oss}
Le famiglie di esponenziali hanno interessanti proprietà matematiche (proprietà di regolarità).

Dal punto di vista statistico, ciò si traduce in un'interessante conseguenza: tutta l'informazione contenuta nei dati a disposizione $(X_1,...,X_n)$ relativa alla funzione $f_X (x; \theta)$ può essere sintetizzata attraverso $k$ quantità (funzioni di $(X_1,...,X_n)$) che potranno essere impiegate per costruire procedure inferenziali (stima, test per la verifica di ipotesi) riguardanti il parametro $\theta$.\\
Ovvero, l'appartenenza ad una famiglia esponenziale permette una riduzione dei dati $(X_1$, ... ,$X_n)$ via $B_m$.\end{oss}
%%%%lezione 1 marzo%%%%

\section{Trasformazioni di variabili casuali}
Lezione del 01/03, ultima modifica 09/04, Michele Nardin
\subsection{Variabili discrete}
\begin{thm}
Sia \(X\) una variabile casuale con funzione di massa \(f_X(x)=P(X=x)\), e sia \(A\) il suo supporto.
Sia \(W=h(X)\) una nuova variabile casuale, e
\begin{equation*}
  B_w = \lbrace x \in A \colon h(x) = w\rbrace.
\end{equation*}
Allora, la funzione di massa probabilistica di \(W\) corrisponde a
\begin{equation}
  \mathbb{P}(W=w)=\sum_{x \in B_w}\mathbb{P}(X=x).
\end{equation}
\end{thm}

\noindent \textbf{Esempi}
\begin{enumerate}

\item Sia $X \sim b(n,p)$ con relativa funzione di massa 
$f_X(x,p)=\binom{n}{x}p^x(1-p)^{n-x}\mathbbm{1}_{0,1,...,n}(x)$,
$n$ noto e $p\in(0,1)$.

Considero quindi $W=n-X$. Come si distribuisce $W$? 
$$P(W=w)=P(X=n-w)=\binom{n}{n-w}p^{n-w}(1-p)^w\mathbbm{1}_{0,1,...,n}(w)$$
\item Sia $X$ una vc tale che 
$f_X(x)=P(X=x)=\left(\frac{1}{2}\right)^x\mathbbm{1}_\mathbbm{N}(x)$, $W=X^3$. 
%%lasciando una riga vuota va a capo senza dare badbox! figata
$$P(W=w)=P(X^3=w)=P(X=\sqrt[3]{w})=\left(\frac{1}{2}\right)^{\sqrt[3]{w}}\mathbbm{1}_{1,8,27,64,...}(w)$$
\end{enumerate}

\subsection{Variabili assolutamente continue}
\begin{thm}
Sia \(X\) una variabile casuale assolutamente continua, con funzione di densità \(f_X(x)\) e sia \(W=h(X)\), ove \(h\) è una funzione monotona. 
Supponiamo inoltre che \(f_X(x)\) sia continua sul supporto di \(X\) e che $h^{-1}(w)$ abbia derivata continua sul supporto di \(W\). Allora
\begin{equation}
  f_W(w)=f_X(h^{-1}(w))
  \left\lvert\frac{d}{dw}h^{-1}(w)\right\rvert\mathbbm{1}_{A_W}(w).
\end{equation}
\end{thm}
\noindent \textbf{Esempio}
(Standardizzazione di una vc normale)
Sia $X \sim N(m,s^2)$. 
Considero $W=h(X)=\frac{X-m}{s}$. 
Allora, dato che $h^{-1}(w)=sw+m$, 
che ha derivata continua su tutto $\mathbbm{R}$,  
$$f_W(w)=f_X(sw+m)|s|=\frac{e^{\frac{-w^2 }{2}}}{\sqrt{2\pi}} = f_{N(0,1)}$$
\begin{thm}
Se $W=h(X)$ ove h è monotona a tratti (un numero di tratti finito k) e valgono le condizioni del teorema precedente (su ogni tratto), allora
$$f_W(w)=\sum_{n=1}^k f_X(h_n^{-1}(w))\left|\frac{d}{dw}h_n^{-1}(w)\right|\mathbbm{1}_{A_W}(w)$$
\end{thm}
\noindent \textbf{Esempio} (Chi-quadro)

\noindent Sia $X \sim N(0,1)$ e $W=h(X)=X^2$. $h$ è monotona sui tratti $A_0={0}$, $A_1=(-\infty,0)$, $A_2=(0,+\infty)$. 

\noindent Considero $h_1(x)=x^2$ per $x<0$ mentre $h_2(x)=x^2$ per $x>0$.

\noindent Trovo che $h_1^{-1}(w)=-\sqrt{w}$ (NB:$h_1^{-1}(w)\in A_1 \forall w \geq 0$), mentre $h_2^{-1}(w)=\sqrt{w}$ (NB:$h_2^{-1}(w)\in A_2 \forall w \geq 0$).
 
\noindent $\frac{d}{dw} h_1^{-1}(w)=-\frac{1}{2\sqrt{w}}$, $\frac{d}{dw} h_2^{-1}(w)=\frac{1}{2\sqrt{w}}$ sono entrambe continue su $\mathbbm{R}_+$.
$$f_W(w)=\frac{1}{\sqrt{2\pi}}e^{\frac{-(-\sqrt{w})^2 }{2}}\left|\frac{1}{2\sqrt{w}}\right|+\frac{1}{\sqrt{2\pi}}e^{\frac{-(\sqrt{w})^2 }{2}}\left|\frac{1}{2\sqrt{w}}\right|$$ 
$$=\frac{1}{\sqrt{2\pi w}}
e^{\frac{-w}{2}}\mathbbm{1}_{\mathbbm{R}+}
(w)=\frac{1}{2^{1/2}\Gamma(1/2)}
w^{\frac{1}{2}-1}e^{\frac{-w}{2}}$$
Si riconosce che $W \sim \mathcal{G}(\alpha=1/2,\beta=2)$ e si chiama Chi quadrato con $\nu=1$ gradi di libertà.

In generale, una vc Chi Quadro con $\nu=n$ gradi di libertà è $W=\sum_{i=1}^n X_i^2$, ove $X_1,X_2,...,X_n$ sono vc iid N(0,1). Per il Teorema 2 sulla FGM di una somma di vc iid si trova immediatamente che $W \sim \mathcal{G}(\alpha=n \cdot 1/2,\beta=2)$.

\section{Convergenze di variabili casuali}

\subsection{Convergenza in probabilità}

\begin{dfn}[Convergenza in probabilità]
Siano \(\lbrace X_n\rbrace_{n\in\mathbb{N}}\) una successione di variabili
casuali e \(X\) una variabile casuale, tutte definite sullo stesso spazio
campionario. Diciamo che \(X_n\) \emph{converge in probabilità} a \(X\), in
simboli \(X_n\stackrel{p}{\rightarrow}X\), se per ogni \(\varepsilon > 0\)
\begin{equation}
  \lim_{n \to\infty} \mathbb{P}\left(\lvert X_n-X\rvert > \varepsilon\right)
  = 0.
\end{equation}
\end{dfn}
\begin{oss}
Se $X_n\stackrel{p}{\rightarrow}X$ diciamo che la "massa" della differenza $|X_n-X|$ converge a 0.
Inoltre, quando scriviamo $X_n\stackrel{p}{\rightarrow}X$, stiamo sottintendendo tutta la parte iniziale della definizione precendete, cioè il "sia $\{X_n\}_{n\in\mathbbm{N}}$ una successione di variabili casuali...".
\end{oss}

\begin{thm}[Risultati di convergenza]
  Siano \(\lbrace X_n \rbrace_{n\in\mathbb{N}}\),
  \(\lbrace Y_n \rbrace_{n\in\mathbb{N}}\) una successione di variabili
  casuali.
  \begin{enumerate}
  \item Se \(X_n\stackrel{p}{\rightarrow}X\) e
        \(Y_n\stackrel{p}{\rightarrow}Y\), allora
        \(X_n+Y_n\stackrel{p}{\rightarrow}X+Y\);
  \item Se \(X_n\stackrel{p}{\rightarrow}X\) e \(a\) è una costante,
        allora \(aX_n\stackrel{p}{\rightarrow}aX\); 
  \item Se \(X_n\stackrel{p}{\rightarrow}a\) con \(a\) costante, e \(g\) è
        una funzione reale continua in \(a\), allora
        \(g(X_n)\stackrel{p}{\rightarrow}g(a)\);
  \item Se \(X_n\stackrel{p}{\rightarrow}X\) e
        \(Y_n\stackrel{p}{\rightarrow}Y\), allora
        \(X_nY_n\stackrel{p}{\rightarrow}XY\).
  \end{enumerate}
\end{thm}

%\textbf{Esempi}
%\begin{enumerate}
%\item
%\item
%\item
%  \item (Corollario di 3.) Se $X_n\stackrel{p}{\rightarrow}a$, allora $X_n^2\stackrel{p}{\rightarrow}a^2$, $\frac{1}{X_n}\stackrel{p}{\rightarrow}\frac{1}{a}$ (se $a\neq0$), $\sqrt{X_n}\stackrel{p}{\rightarrow}\sqrt{a}$ ($a\geq0$).
%\end{enumerate}
\subsection{Convergenza in distribuzione}
\begin{dfn}

Sia $\{X_n\}_{n\in\mathbbm{N}}$ una successione di variabili casuali e 
sia X un'altra variabile casuale, tutte definite sullo stesso spazio campionario.

Siano $F_{X_n}$ e $F_X$ le relative funzioni di ripartizione (dette anche ''di distribuzione'').
Sia $C(F_X)$ l'insieme dei punti ove $F_X$ è continua. 
Diciamo che $X_n$ converge in distribuzione (o in legge) a X (scriviamo $X_n\stackrel{d}{\rightarrow}X$) se 
$$\lim_{n \rightarrow\infty} F_{X_n}(x)=F_X(x) \; \; \; \forall x \in C(F_X)$$
\end{dfn}

\begin{thm}
Se $X_n\stackrel{p}{\rightarrow}X$ allora $X_n\stackrel{d}{\rightarrow}X$.
\end{thm}
\begin{oss}
Il contrario in generale non vale, tranne nel caso in cui X è una vc degenere (cioè costante).
\end{oss}
\begin{thm} Supponiamo che $X_n\stackrel{d}{\rightarrow}X$ e sia g una funzione continua sul supporto di X. Allora $g(X_n)\stackrel{d}{\rightarrow}g(X)$
\end{thm}
\begin{thm} [Slutsky] Supponiamo che $X_n\stackrel{d}{\rightarrow}X$, $A_n\stackrel{p}{\rightarrow}a$ costante e $B_n\stackrel{p}{\rightarrow}b$ costante. Allora $A_n+B_n X_n\stackrel{d}{\rightarrow}a+bX$
\end{thm}
%%%%fine lezione del 1 marzo%%%%
\subsection{Teoria asintotica}
Lezione del 04/03, ultima modifica 09/04, Michele Nardin

\begin{teo}
\noindent
($\Delta$-method) Sia $\{X_n\}_{n \in\ \mathbbm{N}}$ una successione di vc
tale che 

\noindent $\sqrt{n}(X_n-\vartheta)\stackrel{d}{\rightarrow}N(0,\sigma^2)$. 
Supponiamo che una funzione g(X) sia derivabile in $\vartheta$ e che $g'(\vartheta)\neq0$. Allora $$\sqrt{n}(g(X_n)-g(\vartheta))\stackrel{d}{\rightarrow}N(0,\sigma^2(g'(\vartheta))^2)$$
\end{teo}

\noindent\textbf{Esempio}
Considero $$Y_n=\frac{\chi^2_n-n}{\sqrt{2n}}=\sqrt{n}\left(\frac{\chi^2_n}{\sqrt{2}n}-\frac{1}{\sqrt{2}}\right)$$ ove $\chi^2_n$ è la chiquadro con n gradi di libertà. 
Ricordiamo che $\mathbbm{E}(\chi^2_n)=n$ e che $Var(\chi^2_n)=2n$ (discende dal fatto che $\chi^2_n \sim \mathcal{G}(\alpha=n/2,\beta=2)$. 
Affermiamo che $Y_n \stackrel{d}{\rightarrow} N(0,1)$. Infatti:
$$Y_n = \frac{\chi^2_n-n}{\sqrt{2n}} = \frac{\sum_{i=1}^n X_i^2 - n \cdot 1}{\sqrt{n} \sqrt{2}}$$
dove $X_i \sim N(0,1)$, e quindi $X_i^2 \sim \chi^2_1$, quindi le $X_i^2$ hanno media $\mu=1$ e varianza $\sigma^2=2$. Quindi per il Teorema centrale del Limite (vedi sotto) si ha quanto voluto.\\
Scrivendo ora $Y_n$ nella forma $Y_n=\sqrt{n}\left(\frac{\chi^2_n}{\sqrt{2}n}-\frac{1}{\sqrt{2}}\right)$ riconosciamo che la prima parte delle ipotesi del $\Delta$-method sono soddisfatte.
Considero quindi $g(t)=\sqrt{t}$, che è derivabile in $\vartheta=1/\sqrt{2}$, $g'(t)=\frac{1}{2\sqrt{t}}|_{\vartheta=1/\sqrt{2}}=2^{-3/4}$.
Allora $$\sqrt{n}(g\left(\frac{\chi^2_n}{\sqrt{2}n}\right)-g(\vartheta))=
\sqrt{n}\left(\sqrt{\frac{\chi^2_n}{\sqrt{2}n}}-\sqrt{\frac{1}{\sqrt{2}}}\right)
\stackrel{d}{\rightarrow}N(0,1^2\cdot 2^{-3/2})$$

\begin{teo}
(Teorema centrale del limite) Siano $X_1,...X_n$ vc iid dotate di media $\mu$ e varianza finita $\sigma^2$. Allora 
$$\frac{\sum_{i=1}^n X_i - n\mu}{\sqrt{n}\cdot \sigma} = \frac{\sqrt{n}(\overline{X}_n-\mu)}{\sigma}\stackrel{d}{\rightarrow}N(0,1)$$
con $\overline{X}_n$ media aritmetica delle $X_i$.
\end{teo}

\noindent\textbf{Esempi/Applicazioni}
\begin{enumerate}
\item $X_n \sim b(n,p)$, $X_n \stackrel{a}{\sim}N(np,np(1-p))$ (ricordiamo che $X_n \sim \sum_{i=1}^n b_i$, ove $b_i \sim b(1,p)$).
Quando scriviamo $\stackrel{a}{\sim}$ stiamo considerando un "andamento asintotico", ossia sottintendiamo un'approssimazione (via via migliore con l'aumentare di n) giustificata dal TLC (il senso è che per n 'grandi' la distribuzione 'funziona circa così').
\item $X_1,...,X_n$ vc 
$P(\lambda =1)$. 
Considero $Y_n=\sum X_i$.
Dato che $Y_n \stackrel{a}
{\sim} N(n\lambda ,n \lambda )$ e $\lambda=1$,
 $\bar{Y}_n:=\frac{Y_n}{n} \stackrel{a}
 {\sim} N(1,1/n)$

 Considero quindi $W_n=\sqrt{n}(\frac{Y_n}{n}-1)=\frac{\frac{Y_n}{n}-1}
{1/\sqrt{n}}=\frac{\bar{Y_n} - \mathbbm{E}(\bar{Y}_n)}
{\sqrt{Var(\bar{Y}_n)}}$, trovo che $W_n \stackrel{a}{\sim} N(0,1) $
\end{enumerate}

\begin{teo}
Sia $\{X_n\}$ una succ di vc iid, ognuna con con FGM $M_{X_n}(t)$ definita e $<\infty$ per $t\in(-h,h)$, e sia X un'altra vc con FGM $M_X(t)$ definita e $<\infty$ per $t \in (-h_1,h_1), \; h_1\leq h$. Se $$\lim_{n \rightarrow +\infty} M_{X_n}(t)=M_X(t) \; \; \; \forall |t| \leq h_1$$ allora $X_n \stackrel{d}{\rightarrow} X$.
\end{teo}

\noindent\textbf{Applicazione}

\begin{enumerate}
\item Sia $X_n \sim b(n,p)$. 
Ricordiamo che $X_n=\sum X_i$ ove $X_i \sim b(1,p)$, ed inoltre $\mu=\mathbbm{E}(X)=np$.
Siccome $M_{X_n}(t)=
\mathbbm{E}(e^{tX_n})=[(1-p)+pe^t]^n=
[1+\frac{\mu}{n}(e^t - 1)]^n$, 
$$ M_{X_n}(t) \stackrel{n \rightarrow \infty}{\longrightarrow} 
e^{\mu(e^t-1)}$$
che è la FGM di una Poisson di parametro $\mu$, ovvero $X_n \stackrel{d}{\rightarrow} \mathcal{P}(n,p).$
\end{enumerate}
\newpage
%%%modificato titolo della sezione: sottolineamo il fatto che la prima parte del corso è applicativa

\chapter{Approccio applicativo alla Statistica Matematica}
Questa sezione corrisponde alla parte di corso svolta dalla seconda settimana di marzo fino a metà aprile, che riguarda gli aspetti pratici della statistica: verranno introdotte le statistiche d'ordine, gli intervalli di confidenza e i test per verifiche d'ipotesi.

\begin{dfn}[Campione casuale]
Se il vettore casuale \((X_1,\dotsc,X_n)\) è composto da variabili casuali indipendenti ed identicamente distribuite (iid), esso si dice essere un \emph{campione casuale} di dimensione \(n\) dalla distribuzione comune.
\end{dfn}

\noindent\textbf{Osservazione}
Il fatto che le vc siano i.i.d. implica che $$F_{X_1,...,X_n}(X_1,...,X_n)=\prod_{i=1}^n F_{X_i} (X_i)$$ e $$f_{X_1,...,X_n}(X_1,...,X_n)=\prod_{i=1}^n f_{X_i} (X_i)$$

\begin{dfn}[Statistica]
Sia \((X_1,\dotsc,X_n)\) un campione casuale su una variabile casuale \(X\), e sia \(\Omega\) lo spazio campionario in cui vive il vettore casuale. Una funzione \(T\) della forma
\begin{equation}
  \begin{split}
    \Omega &\to \mathbb{R}^k \\ (X_1\dotsc,X_n) &\mapsto T((X_1\dotsc,X_n)) 
  \end{split}
\end{equation}
non dipendente da parametri incogniti è detta \emph{statistica}.
\end{dfn}

%%Ho tolto le osservazioni di seguito, se poi riteniamo di rimetterle ok, altrimenti io le considero superflue e fuorvianti.

%\textbf{Osservazioni}
%Le cose scritte tra virgolette '' sono concetti e/o definizioni non ancora introdotti, che vengono usati per dare un'idea intuitiva di quello che si andrà a vedere, cose che poi durante il corso verranno trattate con rigore.
%\begin{enumerate}
%\item Una statistica T è una 'caratteristica numerica' del campione: si presta a $sintetizzare$ l'informazione su $\vartheta$ contenuta nel campione.%
%\item $\sum_{i=1}^n X_i$ e $\sum_{i=1}^n X_i^2$ sono entrambe statistiche: sono alla base di due 'stimatori' molto importanti:

%Media Campionaria: $\bar{X}_n =\frac{1}{n}\sum X_i$

%Varianza Campionaria: $S_n^2=\frac{1}{n-1} \sum (X_i-\bar{X}_n)^2 = \frac{1}{n-1} \sum X_i^2 - \frac{n}{n-1}\bar{X}_n^2$
%\item Ogni statistica è una vc: ha quindi una distribuzione, che dipende dal parametro.

%\textbf{Esempio} Considero $\bar{X}=\frac{1}{n} \sum X_i$ ove $X_i \sim N(\mu,\sigma^2)$. Allora $\bar{X} \sim N(\mu,\frac{\sigma^2}{n})$. Da questo si potrà dedurre la "bontà" di $\bar{X}$ come "stimatore" di $\mu$.
%\item Tra tutti i modi di sintetizzare l'informazione contenuta in $(X_1,...,X_n)$ relativamente a $\vartheta$, siamo interessati a quelli che NON tralasciano informazioni o quote di informazioni rilevanti per il parametro.
%\item In relazione ad uno stimatore potremmo essere interessati ad alcune proprietà, in particolare a queste due:

%$\cdot$ Accuratezza [concetto legato alla media dello stimatore] ($Non$ $distorsione$)

%$\cdot$ Precisione [concetto legato alla varianza dello stimatore] ($efficienza$ o $consistenza$)
%\end{enumerate}
%Fine lezione del 4 marzo. ultima modifica 5 marzo, 14.08, Michele
\subsection{Statistiche d'ordine}
Lezioni 08 e 11 Marzo, ultima modifica 21/03, Scritte da: Marco Peruzzetto

\begin{definizione} Sia $\left(X_1,\ldots ,X_n\right)$ un campione casuale con distribuzione $F_X\left(x,\theta\right)$, densità $f_X\left(x,\theta\right)$ e supporto $\supp\{X\}\coloneqq (a,b)\subset\mathbb{R}$ ove $X\in \{X_1,\ldots,X_n\}$ e $-\infty\leq a<b\leq +\infty$. Definiamo ricorsivamente le seguenti variabili casuali:
\begin{itemize}
\item $X_{(1)}\coloneqq \min\left(\{X_1,\ldots,X_n\}\right)$;
\item $X_{(i)}\coloneqq \min\left(\{X_1,\ldots,X_n\}\setminus\{X_{(1)},\ldots,X_{(i-1)}\}\right)$ $\forall 1<i\leq n$. 
\end{itemize}
Chiameremo allora $X_{(i)}$ la $i$-esima \textit{Statistica d'Ordine} del campione.
\end{definizione}

\textit{Osservazione:} La statistica d'ordine consiste semplicemente nel vettore per il quale le variabili casuali vengono appunto ordinate, in base al valore che assumono in un determinato punto del loro dominio comune, in ordine crescente. In particolare $X_{(i)}$ sarà l' $i$-esima variabile più piccola. Naturalmente, se il campione ha lunghezza $n$, allora $X_{(n)}=\max\left(\{X_1,\ldots,X_n\}\right)$. 
Osserviamo che la funzione $\left(X_1,\ldots,X_n\right)\longmapsto \left(X_{(1)},\ldots,X_{(n)}\right)$ è essa stessa una Statistica.
\begin{teo}\label{dist:ordine}
Sia $\left(X_1,\ldots,X_n\right)$ un campione casuale come sopra. Allora si ottiene $\forall$ $1\leq m\leq n$ che la densità dell' $m$-esima statistica d'ordine è data da: 
\begin{displaymath}
f_{X_{(m)}}\left(x,\theta\right)=\frac{n!}{(m-1)!(n-m)!}f_X\left(x,\theta\right)\cdot F_X\left(x,\theta\right)^{m-1}\cdot \big(1-F_X\left(x,\theta\right)\big)^{n-m}
\end{displaymath}
\end{teo}
Daremo due dimostrazioni, la seconda più bella della prima.
\begin{proof} (a cura di Marco Perruzzetto) Innanzi tutto si ha che il supporto $(a,b)$ può essere partizionato in $n$ parti, per cui evidentemente si ha: 
\footnotesize{
\begin{displaymath}
f_{(X_{(1)},\ldots,X_{(n)})}(x_{(1)},\ldots,x_{(n)},\theta)=
\left\{
\begin{array}{lr}
n!\prod_{i=1}^n f_{X}(x_{(i)},\theta) & \mbox{se } a<x_{(1)}<x_{(2)}<\ldots <x_{(n)}<b \\
0 & \mbox{altrimenti.}
\end{array}
\right.
\end{displaymath}
} \normalsize{ove la produttoria è giustificata dal fatto che le variabili sono tutte indipendenti e che devono essere ciascuna minore dell'altra per l'ordinamento assegnato; il coefficiente fattoriale è presente poiché le $n$ parti dell'intervallo $(a,b)$ possono essere assegnate alle $n$ variabili in tale numero di modi, dato che ciascuna $X_i$ $\forall 1\leq i\leq n$ ha la stessa distribuzione.  \\ Adesso per trovare la distribuzione di ciascuna $X_{(m)}$ sarà dunque sufficiente integrare $f_{(X_{(1)},\ldots,X_{(n)})}$ nei domini possibili di tutte le altre funzioni di distribuzione di ciascuna $X_{(i)}$ con $i\neq m$. In particolare, ciascuna $f_{X_{(i)}}$ per $i<m$ dovrà assumere a piacere valori necessariamente inferiori a $f_{X_{(m)}}$, viceversa ogni $f_{X_{(i)}}$ per $i>m$ dovrà assumere valori obbligatoriamente superiori a quelli di $f_{X_{(m)}}$ in ogni punto. Ricordando allora che possiamo scrivere la distribuzione come $\int_a^x f_X(\theta,t)dt=F_X(\theta,x)$ essendo la densità la derivata della funzione di distribuzione, otterremo quindi che $\forall a<x_{(m)}<b$ la distribuzione sarà data da: \\\\ $f_{X(m)}(x_{(m)},\theta)=$	}
\small{
\begin{eqnarray*}
&=&\int_a^{x_{(2)}} dx_{(1)}\cdots\int_a^{x_{(m)}} dx_{(m-1)}\int_{x_{(m)}}^b dx_{(m+1)}\cdots\int_{x_{(n-1)}}^b dx_{(n)}f_{(X_{(1)},\ldots,X_{(n)})}(x_{(1)},\ldots,x_{(n)},\theta)\\
&=&\int_a^{x_{(2)}} dx_{(1)}\cdots\int_a^{x_{(m)}} dx_{(m-1)}\int_{x_{(m)}}^b dx_{(m+1)}\cdots\int_{x_{(n-1)}}^b dx_{(n)} n!\prod_{i=1}^n f_{X}(x_{(i)},\theta) \\
&=& f_{X}(x_{(m)})\int_a^{x_{(2)}} dx_{(1)}\cdots\int_a^{x_{(m)}} dx_{(m-1)}\int_{x_{(m)}}^b dx_{(m+1)}\cdots\int_{x_{(n-1)}}^b dx_{(n)} n!\prod_{i=1,i\neq m}^n f_{X}(x_{(i)},\theta) \\
&=& \frac{n!}{(m-1)!(n-m)!}f_X\left(x,\theta\right)\cdot F_X\left(x,\theta\right)^{m-1}\cdot \big(1-F_X\left(x,\theta\right)\big)^{n-m}, 
\end{eqnarray*}
}
\normalsize{dove è stato usato il fatto che $\int_a^b F_X^\alpha(\theta, t)f_X(\theta,t)dt=\frac{F_X^{\alpha+1}}{\alpha+1}$, $\forall \alpha\neq -1$.}
\end{proof} 

\begin{proof}
Sia $\Omega$ il dominio comune del campione casuale. Definiamo per $x\in \mathbb{R}$ la nuova variabile casuale $Y_x$ come:
\begin{align*}
\Omega &\longrightarrow  \{0,\ldots,n\} \\
Y_x(\omega)  &\coloneqq   \sum_{i=1}^n \mathbbm{1}_{\{X_i(\omega)\leq x\}}(\omega)=\#\big\{i\in\{1,\ldots,n\}: X_i\leq x\big\},
\end{align*}
funzione che, per così dire, ``conta'' il numero di variabili casuali $X_i$ che non superano $x$. Si vede immediatamente che $\forall 1\leq m\leq n$, si ha la distribuzione 
\begin{eqnarray*}
F_{X_{(m)}}(\theta,x) &=& \mathbb{P}[X_{(m)} \leq x] = \mathbb{P}[\text{almeno } X_{(1)},...,X_{(m)} \text{ stanno sotto $x$}] = \\
&=& \mathbb{P}[Y_x\geq m] = \sum_{k=m}^n \mathbb{P}[Y_x=k] = \\
&=& \sum_{k=m}^n \binom{n}{k}F_X^k(\theta,x)\big(1-F_X(\theta,x)\big)^{n-k}.
\end{eqnarray*}
Come nella prima dimostrazione usiamo il fatto che la densità si può vedere come derivata della funzione di ripartizione. Ne segue che per calcolare la densità sarà sufficiente calcolare la derivata in ciascun punto $x$ della distribuzione appena trovata. In particolare si potrà vedere che coesisteranno il termine che vogliamo ottenere con altre due sommatorie, che tuttavia si elidono l'una con l'altra lasciando quindi la relazione espressa dal teorema. Si ha infatti che: 
\\
\small{
\begin{equation*}\label{e:barwq}\begin{split}
f_{(m)}(\theta,x)&=\frac{\partial}{\partial x}F_{X_{(m)}}(\theta,x)= \\
&=\sum_{k=m}^n \binom{n}{k}\cdot f_{X}(\theta,x) \bigg\{k F_X^{k-1}(\theta,x) \big(1-F_X(\theta,x)\big)^{n-k}-(n-k) F_X^{k}(\theta,x)\big(1-F_X(\theta,x)\big)^{n-k-1}\bigg\} \\
&=m\binom{n}{m}\cdot f_{X}(\theta,x)F_X^{m-1}(\theta,x) \big(1-F_X(\theta,x)\big)^{n-m} + \\ &\quad \sum_{k=m+1}^n k\binom{n}{k} f_{X}(\theta,x) F_X^{k-1}(\theta,x) \big(1-F_X(\theta,x)\big)^{n-k}- \\ &\quad\sum_{k=m}^{n-1} (n-k)\binom{n}{k} f_{X}(\theta,x) F_X^{k}(\theta,x)\big(1-F_X(\theta,x)\big)^{n-k-1} \\
&=\frac{n!}{(m-1)!(n-k)!}\cdot f_{X}(\theta,x)F_X^{m-1}(\theta,x) \big(1-F_X(\theta,x)\big)^{n-m} + \\
&\quad \sum_{j=m}^{n-1} (j+1)\binom{n}{j+1} f_{X}(\theta,x) F_X^{j}(\theta,x) \big(1-F_X(\theta,x)\big)^{n-j-1}- \\
& \quad \sum_{k=m}^{n-1} (n-k)\binom{n}{k} f_{X}(\theta,x) F_X^{k}(\theta,x)\big(1-F_X(\theta,x)\big)^{n-k-1} \\
&=\frac{n!}{(m-1)!(n-k)!}\cdot f_{X}(\theta,x)F_X^{m-1}(\theta,x) \big(1-F_X(\theta,x)\big)^{n-m} + \\
&\quad \sum_{j=m}^{n-1} \frac{n!}{j!(n-j-1)!} f_{X}(\theta,x) F_X^{j}(\theta,x) \big(1-F_X(\theta,x)\big)^{n-j-1}- \\
& \quad \sum_{k=m}^{n-1} \frac{n!}{k!(n-k-1)!} f_{X}(\theta,x) F_X^{k}(\theta,x)\big(1-F_X(\theta,x)\big)^{n-k-1} \\
&=\frac{n!}{(m-1)!(n-k)!}\cdot f_{X}(\theta,x)F_X^{m-1}(\theta,x) \big(1-F_X(\theta,x)\big)^{n-m}.
\end{split}\end{equation*}
}
\end{proof}

\textbf{Definizione.} Sia $\left(X_{(1)},\ldots,X_{(n)}\right)$ una statistica d'ordine di un campione casuale. Allora possiamo definire le nuove seguenti variabili:
\begin{itemize}[noitemsep]
\item $X_{(n)}-X_{(1)}$, detta \textit{Range} oppure \textit{Misura di Dispersione};
\item $\frac{X_{(1)}+X_{(n)}}{2}$ detta \textit{Mid Range} oppure \textit{Misura di Centralità};
\item 
$$
\left.
\begin{array}{rl}
\forall n \mbox{ pari} & \frac{X_{(\frac{n}{2})}+X_{(\frac{n}{2}+1)}}{2} \\
\forall n \mbox{ dispari} & X_{(\frac{n+1}{2})}
\end{array}
\right\} \mbox{dette ciascuna }\textit{Mediana campionaria};
$$
\item Sia $\frac{1}{2(n+1)}<p<1-\frac{1}{2(n+1)}$, che possiamo in ogni caso pensare come $0<p<1$ per $n$ molto grande. A questo punto possiamo definire l'intero $k_p\coloneqq \big\lfloor p(n+1) \big\rfloor + \big\lfloor 2\big(p(n+1)- 2 \lfloor p(n+1) \rfloor \big) \big\rfloor$, che risulta essere così ben definito in quanto compreso tra 1 e $n$ e restituisce l'approssimazione all'intero più vicino al variare di $p$ del reale $p(n+1)$. 
\item A questo punto, se scegliamo $\xi_p\in F_X^{-1}(p)$, chiameremo $\xi_p$ \textit{Quantile di popolazione} di ordine $p$. In seguito troveremo utile stimare tale valore. Perciò introduciamo la variabile casuale ad esso collegata $X_{(k_p)}$, detta \textit{Quantile campionario} di ordine $p$. Se $p=\frac{i}{m}$, allora $X_{(k_p)}$ è detta anche $i$-esimo $m$-ile campionario. In particolare con $Q_1$ e $Q_3$ si indicano rispettivamente il primo e il terzo quartile.\\
Intuitivamente, $X_{(k_p)}$ mi dà la v.c. che sta al $k_p$-esimo posto, ovvero al $p(n+1)$-esimo posto (se $p(n+1)$ è intero). Ad esempio, se $p=\frac{1}{3}$, $X_{(k_{\frac{1}{3}})}$ è la v.c. che nel vettore ordinato sta alla posizione $\frac{n+1}{3}$.
\item Le variabili $LF\coloneqq Q_1-h$ e $UF\coloneqq h+Q_3$, ove $h\coloneqq \frac{3}{2}(Q_3-Q_1)$ sono dette rispettivamente \textit{Lower} e \textit{Upper Fence}. 
\end{itemize}
\textit{Osservazione:} Osserviamo che più la misura di centralità si discosta dalla mediana, più vi è asimmetria nella funzione di densità $f$ (\textit{i.e.:} una funzione di distribuzione è simmetrica $:\Longleftrightarrow$ $\exists x_0\in \mathbb{R}:f(x_0+x)=f(x_0-x), \forall x\in \dom(f).$) Inoltre, ponendo che la funzione di ripartizione sia iniettiva e la funzione di densità sia simmetrica, si vede immediatamente che la media di popolazione, ovvero il quantile di popolazione di ordine $p=\frac{1}{2}$ coincide con il valore di aspettazione della variabile casuale, il quale a sua volta deve coincidere con $x_0$. \\ \\
Dato un campione casuale di parametro $\theta\in \mathbb{R}$ fissato, sappiamo che una qualsiasi funzione di statistiche su tali variabili è, proprio per definizione, uno stimatore del parametro $\theta$. L'esistenza di un'infinità non numerabile di stimatori è sicuramente un problema da ovviare in merito alla scelta tra essi di uno stimatore che effettivamente permetta di stimare il più correttamente possibile il parametro $\theta$. Cercheremo dunque di individuare alcune proprietà che possano effettivamente giustificare la scelta di un determinato stimatore, affinché esso risulti il più possibile affidabile.\\ \\
\textbf{Definizione.} Sia $(X_1\ldots,X_n)$ un campione di parametro $\theta$ e $T_n(X_1,\ldots,X_n)$ uno stimatore. La funzione $B_{\theta}[T_n(X_1,\ldots,X_n)]\coloneqq \mathbb{E}_{\theta}[T_n(X_1,\ldots,X_n)]-\theta$ si dice \textit{distorsione} di $T_n$ (nota: con $\mathbb{E}_{\theta}$ formalmente intendiamo semplicemente $\mathbb{E}$). In particolare $T_n$ si dirà \textit{non distorto} se e solo se la sua distorsione è nulla $\forall \theta\in \mathbb{R}$ (nel senso che uno stimatore -e.g. la media campionaria- non può stimare bene solo "alcune" medie, ma qualsiasi media reale, ad esempio la media di qualsiasi normale centrata in qualsiasi punto). Altrimenti si dice \textit{distorto.} Se infine si ottiene che $\lim_{n\rightarrow +\infty} B_{\theta}[T_n(X_1,\ldots,X_n)]=0$, $T_n$ si dice \textit{asintoticamente non distorto}.
\\ \\
\noindent \textbf{Esempio} Sia $(X_{(1)},\ldots,X_{(n)})$ un campione casuale con distribuzione simmetrica (senza perdita di generalità, la assumiamo simmetrica rispetto all'origine) e scegliamo come stimatore $T_n$ proprio la mediana campionaria. È chiaro innanzi tutto che essa in generale gode delle seguenti due proprietà: \begin{itemize}
\item $\forall b\in \mathbb{R}, $ $T_n(X_1+b,\ldots,X_n+b)=T_n(X_1,\ldots,X_n)+b$;
\item $T_n(-X_ 1,\ldots,-X_n)=-T_n(X_1,\ldots,X_n)$.
\end{itemize}
Abbiamo inoltre che la distribuzione di $(X_ 1,\ldots,X_n)$ e del vettore $(-X_ 1,\ldots,-X_n)$ coincidono (ricordando che l'origine è il centro di simmetria). Si avrà dunque: 
\begin{eqnarray*}
\mathbb{E}[T_n] &=& \mathbb{E}[T_n(X_1,\ldots, X_n)]=\mathbb{E}[T_n(-X_1\ldots ,-X_n)] \\
&=& \mathbb{E}[-T_n(X_1,\ldots, X_n)] = -\mathbb{E}[T_n(X_1\ldots ,X_n)] \\
&=& -\mathbb{E}[T_n]
\end{eqnarray*}
perciò, in definitiva, $2\mathbb{E}[T_n]=0$ ovvero $\mathbb{E}[T_n]=0$. Quindi, nel caso di una distribuzione simmetrica, la media campionaria è uno stimatore non distorto del valore di aspettazione, del punto di simmetria e della media della popolazione (dato che tutti loro nel nostro caso coincidono). \\

\noindent \textbf{Esempio} Sia $(X_1,\ldots,X_n)$ un campione casuale di parametro $\theta\in \mathbb{R}$ fissato e con $\mu\coloneqq \mathbb{E}[X]$, $\sigma^2\coloneqq \var[X]$. Vogliamo provare a calcolare la distorsione di due stimatori ``classici'':
\begin{enumerate}
\item Scegliamo come stimatore la \textit{media campionaria} $\overline{X}_n\coloneqq \frac{1}{n}\sum_{i=1}^n X_i$. Allora $B_{\theta}[\overline{X}_n]=\mathbb{E}_{\theta}[\overline{X}_n]-\theta=\frac{1}{n}\sum_{i=1}^{n}\mathbb{E}_{\theta}[X_i]-\theta=\frac{1}{n}\cdot n\mathbb{E}_\theta[X]=\mu-\theta$. Dunque la distorsione è costante $\forall n\in \mathbb{N}$. In particolare è uno stimatore non distorto per il valore di aspettazione $\mu$. Possiamo calcolare facilmente anche la varianza di $\overline{X}_n$ che risulta essere $\frac{\sigma^2}{n}$. La media campionaria si rivela essere quindi un buon stimatore. (nota: nel calcolo della varianza stiamo trattando lo stimatore come una variabile casuale essa stessa, quindi più la varianza è piccola megliore è lo stimatore).
\item Prendiamo ora come stimatore la \textit{varianza campionaria}, data dalla variabile $S_n^2\coloneqq \frac{1}{n-1}\sum_{i=1}^n (X_i-\overline{X}_n)^2$. Allora: \\
 $\mathbb{E}[S_n^2]=\frac{1}{n-1}\sum_{i=1}^n \mathbb{E}[(X_i-\overline{X}_n)^2]=\frac{1}{n-1}\sum_{i=1}^n \mathbb{E}\big[\left((X_i-\mu)-(\overline{X}_n-\mu)\right)^2\big]=$
 
 
 $=\frac{1}{n-1}\left(\sum_{i=1}^n \mathbb{E}[(X_i-\mu)^2]-\sum_{i=1}^n\mathbb{E}[(\overline{X}_n-\mu)^2]\right)$
$=\frac{1}{n-1}\cdot (n-1)\sigma^2=\sigma^2.$ Perciò $S_n^2$ è uno stimatore non distorto di $\sigma^2$. Notiamo che lo stimatore $S_n^*\coloneqq \frac{1}{n}\sum_{i=1}^n (X_i-\overline{X}_n)^2$ avrebbe distorsione $-\frac{\sigma^2}{n}$, e dunque è peggiore della varianza campionaria, anche se è asintoticamente non distorto. \\ 
\end{enumerate}
Calcoleremo adesso la varianza della varianza campionaria. Assumiamo per il momento che il campione provenga da una distribuzione normale $N(\mu,\sigma^2)$. In tal caso mostriamo che $\frac{n-1}{\sigma^2}S_n^2 \sim \chi_{n-1}^2$ e dunque si avrà subito $\var[S_n^2]=\frac{2\sigma^4}{n-1}$. Infatti, $\frac{n-1}{\sigma^2}S_n^2=\sum_{i=1}^n \left(\frac{X_i-\overline{X}_n}{\sigma}\right)^2=\sum_{i=1}^n\left(\frac{X_i-\mu}{\sigma}-\frac{\overline{X}_n-\mu}{\sigma}\right)^2=\sum_{i=1}^n\left(\frac{X_i-\mu}{\sigma}\right)^2-n\left(\frac{\overline{X}_n-\mu}{\sigma}\right)^2=\sum_{i=1}^n\left(\frac{X_i-\mu}{\sigma}\right)^2-\left(\frac{\overline{X}_n-\mu}{\frac{\sigma}{\sqrt{n}}}\right)^2\Rightarrow \sum_{i=1}^n \left(\sim \chi_1^2\right)-\left(\sim \chi_1^2\right)\Rightarrow \frac{n-1}{\sigma^2}S_n^2 \sim \chi_{n-1}^2$, ove abbiamo usato il seguente teorema: \\
\\
\begin{teo}
Sia $(X_1,\ldots,X_n)$ un campione casuale ove la funzione generatrice di ciascuna $X_i$, $1\leq i\leq n$ è $M_{X}(t)$. Allora $M_{\overline{X}_n}(t)=(M_{X}(\frac{t}{n}))^n$.
\end{teo}
$\vspace{3 mm}$
per mostrare che $\overline{X}_n\sim N(\mu, \frac{\sigma^2}{n})$. Infatti: \\ 
$M_{\overline{X}_n}(t)=(M_{X}(\frac{t}{n}))^n=\left(e^{\mu\frac{t}{n}+\frac{\sigma^2 t^2}{2n^2}}\right)^n=e^{\mu t+\frac{\sigma^2}{n}\cdot\frac{t}{2}}$, da cui la tesi. \\

\begin{teo}
Sia $(X_1,\ldots,X_n)$ un campione casuale da una popolazione con distribuzione discreta o assolutamente continua dove la densità associata sia della forma $f(x,\theta)=C(x)D(\theta)\exp\{\sum_{m=1}^k A_m(\theta)B_m(x)\}$ con $k$ naturale positivo. Siano $T_1,\ldots,T_k$ statistiche definite $\forall 1\leq m\leq k$ da $T_m(X_1,\ldots,X_n)\coloneqq \sum_{i=1}^n B_m(X_i)$. Allora la distribuzione di $(T_1,\ldots,T_k)$ sarà ancora della forma esponenziale:
$$f_{(T_1,\ldots,T_k)}(\theta,t_1,\ldots,t_k)=C(t_1,\ldots,t_k)D(\theta)^n exp\bigg\{\sum_{i=1}^k A_m(\theta)t_m\bigg\}$$.
\end{teo}

\textit{Esempio.} Sia $(X_1,\ldots,X_n)$ un campione casuale. Supponiamo che $X\sim \bi(1,p)$. Allora la densità sarà discreta, ossia sarà $f(p,x)=\mathbb{P}[X=x]=\mathbbm{1}_{\{0,1\}}(x)(1-p)\exp\big\{x \log\big(\frac{p}{1-p}\big)\big\}$. Applicando il teorema otteniamo $T_1(X_1,\ldots,X_n)=$ $\sum_{i=1}^n B_1(X_i)=\sum_{i=1}^n X_i$ da cui si deduce subito che $T_1\sim \bi(n,p)$. In particolare possiamo scriverne la densità: $f_{T_1}(p,t_1)=\mathbbm{1}_{\{0,\ldots,n\}}(t_1)\binom{n}{t_1}(1-p)^n\exp\{t_1 \log\big(\frac{p}{1-p}\big)\}$. \\ 

\paragraph{Massimo campionario.} Sia \((X_1,\dotsc,X_n)\) un campione casuale da distribuzione uniforme \(\mathcal{U}([0,\theta])\). Essendo \(\theta\) il massimo valore che ciascuna variabile può assumere, uno stimatore plausibile
per \(\theta\) potrebbe essere il massimo campionario, cioè \(X_{(n)} = \max\lbrace X_1,\dotsc,X_n\rbrace\). Per il Teorema~\ref{dist:ordine}, ne conosciamo già la distribuzione:
\begin{equation*}
  F_X(x,\theta) = \frac{x}{\theta} \implies
  f_{X_{(n)}}(x,\theta) = \frac{n}{\theta^n}x^{n-1}
\end{equation*} 
Allora $\mathbb{E}[T_n]=\int_0^\theta \frac{n}{\theta^n}x^n dx=\frac{n}{n+1}\theta \neq \theta\Longrightarrow B_{\theta}[T_n]=\frac{-\theta}{n+1}$. 
Ne segue che è distorto, ma asintoticamente non distorto per $\theta$. 
Possiamo anche calcolarne la varianza: $\var[T_n]=\mathbb{E}[T_n^2]-\mathbb{E}[T_n]^2=\int_0^\theta \frac{n}{\theta^n}x^{n+1} dx -\frac{n}{n+1}=\frac{n\theta^2}{(n+1)^2(n+2)}$
$\xrightarrow[n\rightarrow \infty]{}0$. 
Perciò il massimo $X_{(n)}$ rimane in ogni caso uno stimatore affidabile. Osserviamo che possiamo tuttavia introdurre un nuovo stimatore che ci assicura la non distorsione, ovvero $T_n^*\coloneqq \frac{n+1}{n}T$, che possiede le proprietà cercate. \\ \\

\begin{dfn}[Consistenza]
  Sia \(X\) una variabile casuale avente distribuzione \(F(x,\theta)\).
  Sia \((X_1,\dotsc,X_n)\) un campione casuale dalla distribuzione di \(X\) e
  \(T_n\) una statistica. Diciamo che \(T_n\) è uno stimatore \emph{consistente}
  per \(\theta \in \Theta\) se %\(T_n \xrightarrow[]{\mathbb{P}} \theta\):
  \(T_n\) converge in probabilità a \(\theta\).
\end{dfn}

\noindent\textit{Esempio.} Sia $(X_1,\ldots,X_n)$ un campione casuale, ove $X\in \mathcal{L}^2(\mathbb{R)}$. Indichiamo come al solito media e varianza rispettivamente con $\mu$ e $\sigma^2$. Allora abbiamo:
\begin{enumerate}
\item La media campionaria $\overline{X}_n\coloneqq \frac{1}{n}\sum_{i=1}^n X_i\xrightarrow[n\rightarrow \infty]{\mathbb{P}} \mu$, grazie alla legge debole dei grandi numeri poiché $\lim_{n\rightarrow +\infty} \mathbb{P}[(\overline{X}_n-\mu)>\varepsilon ]=0$, $\forall \varepsilon >0$.
\item Consideriamo adesso la varianza campionaria $$S_n^2\coloneqq \frac{1}{n-1}\sum_{i=1}^n (X_i-\overline{X}_n)^2=\frac{n}{n-1}\left(\frac{1}{n}\sum_{i=1}^n X_i^2-\overline{X}_n^2\right)$$. Abbiamo ora i seguenti tre termini:
\begin{itemize}
\item $\lim_{n\rightarrow +\infty} \frac{n}{n-1}=1$, un semplice limite;
\item $\frac{1}{n}\sum_{i=1}^n X_i^2 \xrightarrow[n\rightarrow \infty]{\mathbb{P}} \mathbb{E}[X^2]$, ancora grazie alla legge debole dei grandi numeri e al fatto che $X^2$ rimane ancora sommabile;
\item $\overline{X}_n^2 \xrightarrow[n\rightarrow \infty]{\mathbb{P}} \mu^2=\mathbb{E}[X]^2$ grazie al Teorema 4 sulla convergenza.
\end{itemize}
Ne segue quindi che $S_n^2 \xrightarrow[n\rightarrow \infty]{\mathbb{P}} \sigma^2$, sempre per i teoremi sulla convergenza di somma, prodotto e prodotto per costanti di variabili casuali.
\item Consideriamo ancora il campione casuale distribuito uniformemente $\uni([0,\theta])$ con stimatore $T_n(X_1\ldots,X_n)\coloneqq X_{(n)}$. Troviamo che anch'esso è consistente per la stima del massimo. Infatti, $\mathbb{P}[|T_n-\theta|>\varepsilon]=\mathbb{P}[\theta-T_n>\varepsilon]=\mathbb{P}[X_{(n)}\leq \theta-\varepsilon]=F_{X_{(n)}}(\theta -\varepsilon)=\left(1-\frac{\varepsilon}{\theta}\right)^n \xrightarrow[n\rightarrow \infty]{} 0$. Allo stesso modo si può verificare che anche $T_n^*$ è consistente per $\theta$.
\end{enumerate}
\textbf{Definizione.} Sia $(X_1,\ldots,X_n)$ un campione casuale e $T_n: \mathfrak{X}\longrightarrow\mathcal{Y}_{T_n}$ una statistica (stimatore). Vi sia inoltre una funzione di parametri $a: \Theta\longrightarrow\mathcal{Y}_{\Theta}$. Allora la funzione non negativa $\lo: \left(\mathcal{Y}_{T_n}\cup\mathcal{Y}_{\Theta}\right)\times\mathcal{Y}_{\Theta}\longrightarrow \mathbb{R}_{\geq 0}$ viene detta \textit{Funzione di Perdita} se soddisfa alle seguenti condizioni:
\begin{enumerate}[noitemsep]
\item $\lo\big(a(\theta),a(\theta)\big)=0$, $\forall \theta\in \Theta$;
\item Per ogni $T_n\in \mathcal{T}$, esiste una funzione $\ri: Y_{T_n}\times Y_{\Theta}\longrightarrow \mathbb{R}$, detta \textit{Funzione di Rischio}, tale che $\ri\big(T_n,a(\theta)\big)=\mathbb{E}_{\theta}[\lo\big(T_n,a(\theta)\big)]$, $\forall \theta\in \Theta$. 
\end{enumerate}
\textit{Osservazione.} La funzione di perdita può essere pensata come una misura della discrepanza tra l'azione $T_n$ e lo stato della natura $a(\theta)$. \\ \\
\textbf{Definizione.} Possiamo già definire due tipologie di funzioni di perdita che spesso vengono utilizzate in statistica: 
\begin{enumerate}[noitemsep]
\item $\lo_1\big(T_n,a(\theta)\big)\coloneqq |T_n-a(\theta)|$, chiamata \textit{Errore assoluto};
\item $\lo_2\big(T_n,a(\theta)\big)\coloneqq \big(T_n-a(\theta)\big)^2$. Essa ammette anche come possibile funzione di rischio $\ri_2\big(T_n,a(\theta)\big)\coloneqq \mathbb{E}_{\theta} \left[(T_n-a(\theta))^2\right]$; se tuttavia $a=\id_{\Theta}$, allora la funzione $\mse_{\theta}(T_n)\coloneqq \ri_2\big(T_n,\theta\big)$ prende il nome di \textit{Mean Square Error} (oppure \textit{Errore Quadratico Medio}).
\end{enumerate}
\textit{Osservazione.} Semplicemente aggiungendo e sottraendo il valore $\mathbb{E}[T_n]^2$ si ottiene subito la seguente uguaglianza: $\mse_{\theta}(T_n)=\var_\theta [T_n]+B_\theta [T_n]^2$.
\begin{teo}
Sia $T_n$ uno stimatore di $\theta$ (non necessariamente non distorto). Allora si ha che $\lim_{n\rightarrow +\infty} \mse_\theta (T_n)=0$ è condizione sufficiente (ma non necessaria) per la consistenza di $T_n.$
\end{teo}
\begin{proof} Si ha infatti la seguente semplice catena di diseguaglianze: 
\begin{eqnarray*}
\mathbb{P}[|T_n-\theta|>\varepsilon] &=& \int_{|T_n-\theta|>\varepsilon} f_{T_n}(\theta,t_n)dt_n \\ 
&<& \int_{|T_n-\theta|>\varepsilon} \frac{(t_n-\theta)^2}{\varepsilon^2}f_{T_n}(\theta,t_n)dt_n < \frac{1}{\varepsilon^2}\mse_\theta (T_n).
\end{eqnarray*}
\end{proof}

\chapter{Statistica inferenziale}
\input{intervalli-confidenza.tex}
\input{0315.tex}
%%% lezione 18 marzo %%%
%%%in fondo lezione 21 marzo %%%


\noindent\textbf{Distribuzione esatta della statistica pivot: distribuzione t di Student}
\\ \\
Lezione del 18/03, ultima modifica 26/03, Michele Nardin
\\
\\
La distribuzione $t$ di Student con $\nu$ gradi di libertà è definita come 
$T=\frac{Z}{\sqrt{S^2 / \nu}}$ ove $Z \sim N(0,1)$ mentre $S^2 \sim \chi^2_\nu$ (chiquadro con $\nu$ gradi di libertà).
La funzione di densità è $$f_{t_\nu}(t,\nu)=\frac{\Gamma((\nu + 1)/2)}{\Gamma(\nu / 2)}
\frac{1}{\sqrt{\pi \nu}} \frac{1}{[1+t^2/\nu]^{\frac{v+1}{2}}} \mathbbm{1}_\mathbbm{R} (t)$$
tale funzione è simmetrica, ha la classica forma a campana come la normale, ma a differenza di quest'ultima ha le code più pesanti.
Risulta che la statistica pivot per la media in campioni poco numerosi 
\footnote{In realtà vale per tutti i campioni, è solo che da un certo punto in poi la differenza con la normale è davvero trascurabile! Sulle tavole si riporta solo per $\nu < 120$} (in caso di campionamento da normale) 
ha distribuzione esatta t di Student. Infatti 
$$Q=\frac{\overline{X}_n - \mu}{S_n / \sqrt{n}}=\frac{\frac{\overline{X}_n - \mu}{\sigma / \sqrt{n}}}{\sqrt{\frac{S^2_n}{\sigma^2}}}$$
troviamo al numeratore $\frac{\overline{X}_n - \mu}{\sigma / \sqrt{n}} \sim N(0,1)$, (grazie al fatto che le $X_i$ sono equi distribuite normalmente) 
mentre al denominatore abbiamo che 
$$\sqrt{\frac{S^2_n}{\sigma^2}}= \sqrt{\frac{(n-1)S^2_n}{(n-1) \sigma^2}}= \sqrt{\frac{H}{(n-1)}} $$
Abbiamo già dimostrato che $H=\frac{(n-1)S^2_n}{\sigma^2} \sim \chi^2_{n-1}$, quindi in definitiva al denominatore abbiamo la radice di una chiquadro diviso i suoi gradi di libertà, ovvero siamo proprio in presenza di una distribuzione t di Student.
\\ \\
\noindent\textbf{Osservazione importante:} Quindi, quando il campione casuale è poco numeroso, è conveniente usare i quantili della distribuzione t di student per costruire gli intervalli di confidenza. Per numerosità campionarie $n>30$, approssimare la distribuzione t di student con la distribuzione normale offre risultati soddisfacenti. Ricordiamo che per il tlc $Q\rightarrow N(0,1)$)
\\ \\
\noindent\textbf{Intervallo di confidenza esatto}
\\ \\
Fissato un livello di confidenza $1-\alpha$, consideriamo i quantili della distribuzione t di student (con n-1 gradi di libertà, ove n è la dimensione campionaria) 
$\pm t_{(\alpha/2;n-1)}$, 
troviamo $$ P\left(-t_{(\alpha/2;n-1)} \leq \frac{\overline{X}_n - \mu}{S_n / \sqrt{n}}
 \leq t_{(\alpha/2;n-1)}\right) = 1 - \alpha $$
Notiamo che questa volta vale l'uguaglianza 'vera', poiché non stiamo considerando approssimazioni asintotiche. 
In presenza del campione effettivamente estratto, $(x_1,...,x_n)$, 
scriviamo $\overline{x}_n$ e $s^2_n$ i valori assunti da media e varianza campionaria,
l'intervallo di confidenza è $$IC_{\mu}(1-\alpha)=
\left[\overline{x}_n -
 t_{(\alpha / 2;n-1)} 
 \sqrt{\frac{s^2_n}{n}},
  \overline{x}_n + t_{(\alpha / 2;n-1)}\sqrt{\frac{s^2_n}{n}}\right]$$
\begin{oss}
Alcune osservazioni che, pur sembrando banali, è bene tenere a mente:
\begin{enumerate}
\item Al crescere del livello di confidenza $(1-\alpha)$ e/o della varianza campionaria $S^2_n$ cresce anche l'ampiezza di IC
\item Al crescere dell'ampiezza campionaria $n$, (fermo restando il livello di confidenza) l'ampiezza di IC diminuisce
\end{enumerate}
\end{oss}

\noindent\textbf{Intervalli di confidenza per la varianza}
\\ \\
Sia \((X_1,\dotsc,X_n)\) un campione casuale da $N(\mu,\sigma^2)$.
Consideriamo la statistica pivot $$W=\frac{n-1}{\sigma^2} S^2_n$$ Abbiamo già mostrato che $W \sim \chi^2_{n-1}$. 
Ma allora, dato che noi cerchiamo $q_1,q_2$ t.c. 
$$P \left( q_1 \leq \frac{n-1}{\sigma^2} S^2_n \leq q_2 \right) =1-\alpha$$ 
troviamo che essi sono i quantili di ordine $\alpha / 2$ e $1 - \alpha / 2$ della chiquadro con n-1 gradi di libertà, che indicheremo $q_1=\chi^2_{(n-1,\alpha / 2)}$ e $q_2=\chi^2_{(n-1,1 - \alpha / 2)}$.
Con qualche passaggio otteniamo:
$$P \left( \frac{1}{q_2} \leq \frac{\sigma^2}{(n-1) S^2_n} \leq \frac{1}{q_1} \right) =1-\alpha$$
$$P \left( \frac{(n-1) S^2_n}{q_2} \leq \sigma^2 \leq \frac{(n-1) S^2_n}{q_1} \right) =1-\alpha$$
Troviamo così l'intervallo casuale (e di conseguenza il relativo intervallo di confidenza, una volta estratto il campione e trovato un valore a $S^2_n$) $$IC=\left[ \frac{(n-1)S^2_n}{q_2};\frac{(n-1)S^2_n}{q_1} \right]$$
\\ \\
\subsection{Intervallo di confidenza per differenza di medie}

Vogliamo confrontare due distribuzioni: \textit{sintetizziamo} la differenza tra due popolazioni tramite la differenza delle loro media.

Supponiamo di avere raccolto due campioni casuali indipendenti
\((X_1,\dotsc,X_n)\) e \((Y_1,\dotsc,Y_m)\) provenienti da distribuzioni \(D_X\) e \(D_Y\) di medie \(\mu_X\), \(\mu_Y\) ignote e deviazioni standard \(\sigma_X\), \(\sigma_Y\) note.

Per stimare le medie delle due popolazioni, possiamo ricorrere alle medie campionarie \(\bar{X}_n\) e \(\bar{Y}_m\). Volendo determinare un intervallo di confidenza per \(\Delta := \mu_X - \mu_Y\), osserviamo i seguenti fatti:
\begin{enumerate}
  \item La variabile casuale \(\hat{\Delta} := \bar{X} - \bar{Y}\) avrà valore atteso \(\Delta\) e varianza
  \begin{equation*}
    \var(\hat{\Delta}) = \frac{\sigma_X^2}{n} + \frac{\sigma_Y^2}{m},
  \end{equation*}
  grazie all'assunzione di indipendenza dei due campioni casuali.
  \item La funzione
  \begin{equation*}
    Z := \frac{(\bar{X}-\bar{Y}) - (\mu_X - \mu_Y)}{\sqrt{n^{-1}\sigma_X^2+m^{-1}\sigma_Y^2}}
  \end{equation*}
  risulta essere statistica pivot per \(\Delta\).
  \item Per il teorema del limite centrale, \(Z\) ha distribuzione asintotica Normale standard indipendentemente dalla distribuzione di \(X\) e \(Y\). Ciò ci consente, per campioni di grandi dimensioni, di determinare un'intervallo di confidenza approssimato.
\end{enumerate}
L'intervallo di confidenza approssimato avrà forma
\begin{equation}
  \mathrm{IC}_{\Delta}(1-\alpha) \colon
  \left[(\bar{x} - \bar{y}) - z_{\alpha/2} \sqrt{ \frac{\sigma_X^2}{n} + \frac{\sigma_Y^2}{m}},(\bar{x} - \bar{y}) + z_{\alpha/2} \sqrt{ \frac{\sigma_X^2}{n} + \frac{\sigma_Y^2}{m}}\right]
\end{equation}

Al posto delle varianze possiamo usare anche gli stimatori corretti e consistenti varianza campionaria, e giungere allo stesso risultato per il teorema di Slutsky.
\\ \\
In generale non conosciamo la varianza delle distribuzioni: in base al problema che dobbiamo affrontare, può essere plausibile supporre di conoscere la distribuzione delle due popolazioni a meno di uno o più parametri.

\paragraph{Location Model.} Supponiamo di avere a disposizione due campioni casuali indipendenti \((X_1,\dotsc,X_n)\) e \((Y_1,\dotsc,Y_m)\) provenienti da distribuzioni Normali di parametri ignoti, con l'unica informazione che \(\sigma_X^2 = \sigma_Y^2 = \sigma^2\). Differentemente da quanto fatto prima, non è possibile normalizzare \(\hat{\Delta} = \bar{X} - \bar{Y}\) ottenendo una statistica pivot.
Tuttavia, considerando le varianze campionarie \(S_X^2\) e \(S_Y^2\), possiamo ottenere una statistica pivot i cui percentili sono tabulati: in particolare, ricaviamo una statistica avente distribuzione \(t\)-Student.
Le informazioni a disposizione sono le seguenti:
\begin{enumerate}
  \item \(\frac{(n-1)S_X^2}{\sigma^2} \sim \chi_{n-1}^2\) e
  \(\frac{(m-1)S_Y^2}{\sigma^2} \sim \chi_{m-1}^2\);
  \item
  \begin{equation*}
    Z := \frac{\hat{\Delta}-\Delta}{\sqrt{\frac{\sigma_X^2}{n} + \frac{\sigma_Y^2}{m}}}
    = \frac{\hat{\Delta} - \Delta}{\sigma\sqrt{\frac{1}{n}+\frac{1}{m}}}
    \sim N(0,1)
  \end{equation*}
\end{enumerate}
Ora, definiamo la seguente variabile casuale, detta \emph{pooled variance}:
\begin{equation}
  S_p^2 := \frac{(n - 1)S_X^2 + (m - 1)S_Y^2}{n + m - 2}
\end{equation}
la quale risulta essere uno stimatore corretto e consistente di \(\sigma^2\). Inoltre,
\begin{equation*}
  \frac{(n+m-2)S_p^2}{\sigma^2} \sim \chi_{n+m-2}^2.
\end{equation*}
Ora, possiamo finalmente definire una statistica pivot:
\begin{equation}
  T := \frac{Z}{\sqrt{\frac{S_p^2(n+m-2)}{\sigma^2(n+m-2)}}} =
  \frac{\hat{\Delta}-\Delta}{S_p\sqrt{\frac{1}{n} + \frac{1}{m}}}
  \sim t_{n+m-2}
\end{equation}
per cui ci è possibile costruire un intervallo di confidenza per \(\Delta\), essendo la distribuzione \(t\)-Student tabulata.
Ricalcando i passaggi delle applicazioni precedenti, fissato $\alpha$  troviamo l'intervallo casuale per $\Delta$.
\begin{equation}
  \mathrm{IC}_{\Delta}(1-\alpha) \colon
  \left[(\bar{x} - \bar{y}) - t_{n+m-2;\alpha/2}\, S_p\sqrt{ \frac{1}{n} + \frac{1}{m}},(\bar{x} - \bar{y}) t_{n+m-2;\alpha/2}\, S_p\sqrt{ \frac{1}{n} + \frac{1}{m}}\right]
\end{equation}


\subsection{Intervallo di confidenza per la differenza di proporzioni}

Supponiamo di avere \((X_1,\dotsc,X_n)\) da distribuzione \(b(1,p_X)\), con stimatore \(\hat{p}_X\) e \((Y_1,\dotsc,Y_m)\) da distribuzione \(b(1,p_Y)\), con stimatore \(\hat{p}_Y\). Supponiamo che i due campioni siano tra loro indipendenti. Allora
\begin{equation}
  \Delta := \hat{p}_X - \hat{p}_Y \stackrel{a}{\sim}
  N\left(p_X - p_Y, \frac{p_X(1-p_X)}{n} + \frac{p_Y(1-p_Y)}{m}\right).
\end{equation}
 quindi usando la statistica Pivot 
\begin{equation*}
  Z := \frac{(\hat{p}_X - \hat{p}_Y) - (p_X - p_Y)}{\sqrt{\frac{p_X(1-p_X)}{n} + \frac{p_Y(1-p_Y)}{m}}} \stackrel{a}{\sim} N(0,1)
\end{equation*}
trovo l'intervallo di confidenza, utilizzando gli stimatori campionari già introdotti per definire gli estremi dell'intervallo:
\begin{equation*}
  \mathrm{IC}_{\Delta}(1-\alpha) \colon \hat{p}_X - \hat{p}_Y \pm
  z_{\alpha/2}\sqrt{\frac{\hat{p}_X(1-\hat{p}_X)}{n} + \frac{\hat{p}_Y(1-\hat{p}_Y)}{m}}
\end{equation*}
\input{0320.tex}
\input{test-ipotesi.tex}
%%%%lezione 25 marzo%%%%

\section{Test di ipotesi}
Lezione del 25/03, ultima modifica 20/05, Andrea Gadotti
\\ \\

La procedura di test per la verifica di ipotesi che descriveremo a breve cerca di fornire una soluzione ai seguenti problemi:
\begin{enumerate}
\item Determinare quanto un'ipotesi è realistica, verosimile, compatibile con l'informazione empirica a disposizione.
\item Trovare un ragionamento oggettivo (matematico) per inferire dall'informazione disponibile (ovvero il contenuto di un campione) circa la veridicità dell'ipotesi formulata.
\item Misurare in qualche modo questa ''vicinanza'' tra ipotesi e realtà.
\end{enumerate}
Useremo statistiche pivot in ambito parametrico: la distribuzione da cui proviene il campione casuale $(X_1,...,X_n)$ è nota a meno di uno o più parametri.\\

\subsection{Tipi di ipotesi} 
Supponiamo di avere un campione casuale i.i.d.
\((X_1,\dots,X_n) \sim F_X(\mathbf{x}; \theta)\). Supponiamo che lo spazio \(\Theta{}\) in cui vive il parametro \(\theta{}\) sia partizionato in due sottoinsiemi \(\Theta_0, \Theta_1\). Le \emph{ipotesi} vengono poste nella forma:
\begin{equation}
  \begin{cases}
    H_0 \colon \theta \in \Theta_0 \\
    H_1 \colon \theta \in \Theta_1,
  \end{cases}
  \Theta = \Theta_0 \cup \Theta_1.
\end{equation}

Chiameremo \(H_0\) \emph{ipotesi nulla} e \(H_1\) \emph{ipotesi alternativa}. Idealmente, \(H_0\) rappresenta la conoscenza pregressa, la supposizione vera fino a prova contraria; invece, \(H_1\) costituisce l'ipotesi di lavoro, quella su cui ripieghiamo nel momento in cui il nostro test risulta in contraddizione con \(H_0\).

Il test si riduce a una \textit{regola di decisione} in merito a $H_0$ e $H_1$ sulla base del campione casuale $(X_1,...,X_n)$ da $X \sim F_X (x;\theta)$. Dividiamo lo spazio dei campioni in due regioni disgiunte: $C$ (regione critica del test) e $C^c$. La decisione può chiaramente essere corretta, ma anche errata, poiché il campione costituisce un'informazione non completa. Risulta quindi necessario formulare delle \textit{conclusioni in probabilità}, ovvero associare alla nostra conclusione la probabilità che questa sia corretta, cercando ovviamente di massimizzarla.\\
Possiamo riassumere le varie possibilità nella tabella e nel disegno sottostanti:
\\
\\
\begin{center}
\begin{tabular}{c||c|c}
  & \(H_0\) è \textbf{vera} & \(H_0\) è \textbf{falsa} \\ 
  \hline 
  \textbf{Rifiuto} \(H_0\) & errore di specie \textsc{i} & nessun errore \\ 
  \hline 
  \textbf{Non rifiuto} \(H_0\) & nessun errore & errore di specie \textsc{ii}
\end{tabular} 
\end{center}

\begin{center}
\includegraphics [width=12cm] {immagini/grafico_1.jpg}
\end{center}

\paragraph{Lancio di una moneta.}
Consideriamo il campione casuale \((X_1,...,X_n) \sim b(1,p)\) rappresentante
\(n\) lanci di una moneta. Vogliamo testare l'onestà della moneta, in particolare capire se essa è truccata per far uscire più frequentemente croce. In questo caso, ipotizzeremo:
\begin{equation*}
  \begin{cases}
    H_0 \colon p \ge \frac12  \\ H_1 \colon p < \frac12.
  \end{cases}
\end{equation*}
 e il numero di teste $S_n = \sum_{i=1}^n X_i$. Vorremmo stimare la probabilità che esca testa con la media campionaria $\overline{X}_n$. In questo caso potremmo avere:
\\
$$\bigg \{
\begin{array}{rl}
H_0: & p=1/2 \\
H_1: & p \neq 1/2 \\
\end{array}
$$
\\
La regola di decisione consiste quindi nel rifiutare $H_0$ se $(X_1,...,X_n) \in C$ e invece rifiutare $H_0$ se $(X_1,...,X_n) \in C^c$. Ci piacerebbe trovare una regola di decisione che permetta di minimizzare la probabilità di commettere errori di I o II tipo. Purtroppo questo non è possibile, per la natura stessa della relazione che corre tra gli errori di I e II tipo. Di seguito un esempio che ci dà un'idea del perché:\\

\noindent \textbf{Esempio} Consideriamo un campione casuale $(X_1,...,X_n)$ da $N(\mu,\sigma^2)$ con $\sigma^2$ noto. Supponiamo che le nostre due ipotesi siano:
\\
$$\bigg \{
\begin{array}{lcr}
H_0: & \mu=\mu_0 & \text{ovvero } N(\mu=\mu_0,\sigma^2) \\
H_1: & \mu=\mu_1 & \text{ovvero } N(\mu=\mu_1,\sigma^2) \\
\end{array}
$$
con $\mu_1 > \mu_0$.\\
\\
\begin{center}
\includegraphics [width=12cm] {immagini/grafico_2.jpg}
\end{center}

Consideriamo 
\begin{align*}
  \alpha :&= \mathbb{P}(\text{rifiutare}\,\,H_0\mid H_0\,\,\text{vera}) \\
  &= \mathbb{P}(\text{campione}\,\,\in C\mid H_0\,\,\text{vera}) \\
  &= \mathbb{P}(\text{il nostro campione è } \geq c \mid \text{la distribuzione corretta è quella di sinistra}) \\
  &= P(\text{commettere un errore di I tipo}).
  \intertext{e}
  \beta :&= P(\text{non rifiutare } H_0 \mid H_0 \text{ falsa}) \\
  &= P(\text{il nostro campione appartiene a } C^c \mid H_0 \text{ falsa}) \\
  &= P(\text{il nostro campione è } \leq c \mid \text{la distribuzione corretta è quella di destra}) \\
  &= P(\text{commettere un errore di II tipo})
 \end{align*}. 
(Nota: $\alpha$ è detto \emph{livello di significatività del test})\\
È evidente che non è possibile annullare contemporaneamente sia $\alpha$ che $\beta$.\\
La procedura si divide quindi in due passi: il primo consiste nel \textbf{fissare} $\alpha$, il secondo nell'individuare la regola di decisione che minimizza $\beta$, in modo da trovare un test \textit{ottimo}.\\
\\
\\
\input{0405.tex}
\input{0408.tex}
\input{0412.tex}
\part{Seconda parte del corso}
\chapter{Verosimiglianza}
\chapter{Verosimiglianza statistica}
Lezioni dal 15/04 al 06/05 comprese. Autore: Marco Peruzzetto.\\
Questa parte comprende ed amplia le cose viste a lezione, estendendo alcune dimostrazioni e osservazioni. Quanto non fatto in classe verrà denotato da un (*).\\
\\
\begin{dfn}[Funzione di verosimiglianza]
  Sia \((X_1,\dotsc,X_n)\) un vettore casuale
\end{dfn}
\textbf{Definizione:} Sia $\vec{X}\coloneqq (X_1,\ldots X_n)$ un vettore casuale da distribuzione $F(\vec{x}, \theta)$ per $\theta\in \Theta$ e sia $f_{X_i}(x_i,\theta)$ la corrispondente funzione densità di ciascuna $X_i$, $\forall 1\leq i\leq n$. Indicheremo con $\vec{x}=(x_1,\ldots,x_n)$ una qualsiasi possibile determinazione del vettore $\vec{X}$. Essa conterrà tutta l'informazione in merito a $\theta$. Possiamo allora definire la \textit{Funzione di Verosimiglianza} come la funzione:
$$L\left(\theta, \vec{x}\right)\coloneqq f_{\vec{X}}(\vec{x},\theta)=f_{(X_1,\ldots,X_n)}(x_i,\ldots,x_n,\theta)\mbox{ , } \theta\in \Theta, $$ che rappresenta quindi la funzione di densità dell'intero vettore in dipendenza del parametro $\theta$. Nel caso in cui il vettore casuale sia un campione casuale, allora tutte le variabili casuali di cui esso è composto saranno i.i.d., ragion per cui la funzione di massima verosimiglianza assumerà la seguente tipica forma:
$$L\left(\theta, \vec{x}\right)=\prod_{i=1}^n f_{X_i}\left(x_i, \theta\right) \mbox{ , } \theta\in \Theta.$$
Esiste anche la \textit{Funzione di Log-Verosimiglianza} definita come $l(\theta,\vec{x})\coloneqq \log\big(L(\theta, \vec{x})\big)$. 
\\ 
\\
\textit{Esempio:} Sia $(X_1,\ldots,X_n)\sim Poisson(\theta)$. Allora $$L(\theta, \vec{x})=\prod_{i=1}^n \frac{e^{-\theta}\theta^{x_i}}{x_i !}\mathbbm{1}_{\mathbb{N}}(x_i)=\frac{e^{-n\theta}\theta^{\sum_{i=1}^n x_1}}{\prod_{i=1}^n x_i !}\prod_{i=1}^n \mathbbm{1}_{\mathbb{N}}(x_i)$$ da cui $$l(\theta, \vec{x})=\log(\theta)\sum_{i=1}^n x_i   -n\theta - \sum_{i=1}^n \log(x_i !).$$ 
\\ \\
\textit{Osservazioni:} \begin{itemize}

\item La funzione di verosimiglianza dà un valore alla probabilità che $\vec{x}$ provenga da $F_{\vec{X}}(\vec{x},\theta)$ per tutti i differenti valori di $\theta\in \Theta$.

%\item L'approccio della verosimiglianza al problema della stima produce automaticamente un candidato stimatore. Infatti essa rappresenta una quantità numerica che esprime l'ordine di preferenza circa $\theta$ sulla base dell'informazione contenuta in $\vec{x}$.

\item Nella funzione di verosimiglianza è stato volontariamente invertito il parametro $\theta$ con il parametro $\vec{x}$ rispetto, ad esempio, alla funzione densità. La ragione si basa sulla diversa interpretazione della stessa: a tutti gli effetti la funzione di verosimiglianza non è altro che la funzione densità del vettore casuale $\vec{X}$. Quindi essa può essere vista in due modi diversi: il primo interpreta la funzione $L$ come una funzione di $\vec{x}$, e quindi del risultato, una volta fissato il valore del parametro (perciò $L$ esattamente la densità), mentre il secondo la interpreta come una funzione del parametro $\theta$, per un fissato valore del risultato $\vec{x}$. Proprio in quest'ultimo caso ha senso parlare di verosimiglianza: il valore assunto da $L$ indica quanto verosimilmente il valore di un parametro $(\theta)$ sia corretto rispetto al risultato che si possiede $(\vec{x})$. 

\item (*) Data una determinazione $\vec{x}$ di $\vec{X}$, la funzione $L(\cdot, \vec{x})$, essendo la densità del vettore casuale, esprime la probabilità che $\vec{X}$ assuma proprio il valore $\vec{x}$. Ciò avviene in modo diretto se le variabili componenti il vettore sono discrete e tramite integrazione se continue. Ha senso allora chiedersi, data una determinazione $\vec{x}_0$ di $\vec{X}$, quale sia (se esiste) un possibile valore $\theta_0\in \Theta$ capace di massimizzare il valore di $L(\theta_0,\vec{x}_0)$. Massimizzare tale valore significa infatti per quanto detto, andare a massimizzare la probabilità che $\vec{X}$ assuma il valore $\vec{x}_0$. Ciò avverrà direttamente se il vettore casuale è discreto, ma anche se è continuo, e ciò banalmente grazie alla monotonia dell'integrale, in quanto, se riusciamo a massimizzare la funzione con $\theta$ anche l'integrale (ovviamente integrando in $d\vec{x}$) sarà massimo (rivedere).

\item L'importanza di cercare il valore del parametro che massimizzi $L$ fissata la determinazione risiede nel fatto che spesso in statistica si ha a che fare con poche determinazioni e si parte dunque dall'evidente presupposto che se il campionamento effettuato ci ha fornito quelle specifiche determinazioni, esse debbano essere mediamente le più probabili. Tale presupposto viene in effetti denominato \textit{Principio di ``Rational Belief''}. La probabilità che dato quel campione casuale si ottengano quelle determinazioni la immagineremo quindi come la massima possibile. Cercheremo dunque un $\theta\in \Theta$ che soddisfi a ciò. È inevitabile che attraverso la verosimiglianza si possano ottenere degli stimatori del parametro.

\item La funzione di $\log$-verosimiglianza è stata introdotta pressoché per il semplice motivo di semplificare i calcoli quando si cerca di andare a massimizzare la funzione di verosimiglianza. Essa risulta dunque essere comoda, in quanto, essendo il logaritmo una funzione strettamente crescente, il massimizzante di $l(\cdot,\cdot)$ coinciderà con quello di $L(\cdot,\cdot)$.
\end{itemize}
\textit{Esempio} (Problema dei Pesci): Dato un lago, lo scopo è cercare di stimare la grandezza $N$ della popolazione dei pesci che vi vivono. Un modo può essere il seguente: si pescano esattamente $N_1$ pesci, i quali vengono in qualche modo marcati. In seguito, dopo aver permesso un mescolamento, si esegue un'ulteriore pesca, di $n$ pesci. Si nota che fra questi ve ne sono $n_1$ marcati. Vogliamo capire quale sia il valore di $N$ più plausibile. Nel nostro caso avremo un vettore casuale composto da una sola variabile, ovvero $\vec{X}=(X)$, la quale ha valori in $\mathbb{N}$ (ed è quindi discreta) e restituisce i possibili valori di $n_1$. La sua densità sarà allora fornita in modo diretto e coincide con la funzione di verosimiglianza in quanto vi è una singola variabile casuale nel vettore. L'insieme dei parametri sarà anch'esso $\mathbb{N}$. Chiaramente vogliamo stimare il più plausibile valore di $\theta=N$. Avremo dunque: 
$L(N)\coloneqq L(N,n_1)=\mathbb{P}[X=n_1]=\frac{\binom{N}{n_1}\binom{N-N_1}{n-n_1}}{\binom{N}{n}}.$ Per effettuare un esempio concreto: con $N_1=300$ e $n=80$, se la nostra determinazione ottenuta fosse $n_1=30$, allora il parametro che massimizza la probabilità sarebbe $N\sim 1200$. È quindi plausibile che nel lago viva una quantità di pesci che si aggira effettivamente intorno ai 1200 esemplari.
\\  \\
\textbf{Definizione:} Assumiamo che la funzione di verosimiglianza sia dervabile per il parametro $\theta$. Allora la funzione $S(\theta)\coloneqq \frac{\partial}{\partial \theta}\l(\theta, \vec{x})$ viene detta \textit{Score Function}. L'equazione $S(\theta)=0$ è chiamata \textit{Equazione di Stima}. 
\\
\\
\textit{Osservazione:} Osserviamo che poiché la funzione densità di una qualsiasi variabile casuale è sempre positiva o nulla, in quanto prodotto, lo dovrà essere anche la parte di $L(\cdot,\cdot)$ che non dipende da $\theta$. Ne segue che, se vogliamo massimizzare la funzione di verosimiglianza, possiamo direttamente limitarci a considerare solo i valori di $\theta\in \Theta$ che rendano $L(\cdot,\cdot)$ strettamente positiva per ciascuna determinazione $\vec{x}$ fissata o scelta. Dunque si può restringere senza perdere generalità l'insieme $\Theta$ in modo da avere valori che non permettono a $L(\cdot,\cdot)$ di annullarsi. $S(\theta)$ risulta quindi avere una buona definizione, in quanto non è necessario effettuare ulteriori ipotesi su $l(\cdot,\cdot)$, essendo:  
$$S(\theta)=\frac{\partial}{\partial\theta}l(\theta,\vec{x})=\frac{\partial}{\partial\theta}\log\big(L(\theta,\vec{x})\big)=\frac{1}{L(\theta,\vec{x})}\frac{\partial}{\partial\theta}L(\theta,\vec{x}).$$
\textbf{Definizione:} La funzione di verosimiglianza induce uno stimatore del parametro $\theta$. Esso sarà chiamato \textit{Stimatore di Massima Verosimiglianza} ed è così definito: $\hat{\theta}_n=\hat{\theta}_n(\vec{X})\coloneqq \ar\big\{\max_{\theta\in \Theta}\{L(\theta,\vec{X})\}\big\}=\ar\big\{\max_{\theta\in \Theta}\{l(\theta,\vec{X})\}\big\}$.
\\
\\
\textit{Osservazioni:}
\begin{itemize}
\item Da ora in poi l'argomento delle funzioni $L(\cdot,\cdot)$ e $l(\cdot,\cdot)$ verrà interpretato a seconda della convenienza e del senso sia come $(\theta,\vec{x})$, ovvero come determinazione, oppure come $(\theta, \vec{X})$, ovvero vettore casuale. Osserviamo che in quest'ultimo caso, le funzioni $L$ e $l$ diventano esse stesse automaticamente variabili casuali o stimatori, che dir si voglia. Ciò si ripercuote inevitabilmente sulle funzioni ad esse collegate, ad esempio su $S(\theta)$.
\item Se $L(\cdot,\cdot)$ o $l(\cdot,\cdot)$ sono derivabili rispetto a $\theta$, la funzione indipendente da $\theta$ che risolve $\forall \vec{x}$ l'equazione di stima $S(\theta)=0$ fornisce effettivamente lo stimatore di massima verosimiglianza. Oppure, equivalentemente, potremmo dire che lo stimatore di massima verosimiglianza $\hat{\theta}_n(\vec{X})$ è quello che soddisfa l'equazione $S\big(\hat{\theta}_n(\vec{X})\big)=0$.
\item In generale non vi è garanzia che lo stimatore di massima verosimiglianza esista, oppure, se esiste, che esso sia unico. Tuttavia nel caso di famiglie di densità che rispettino certe ipotesi di regolarità (per esempio le famiglie esponenziali) tale problema non si pone.
\item Anche assumendo che tale stimatore esista e sia unico, non è detto che sia sempre ottenibile analiticamente. Talvolta sarà necessario ricorrere a metodi numerici per la risoluzione dell'equazione di stima.
\end{itemize}
\textit{Esempi:}
\begin{enumerate}
\item Riprendiamo l'esempio precedente ove avevamo il campione casuale con variabili distribuite come poissoniane di parametro $\theta$. 
La funzione di log-verosimiglianza era data da $$l(\theta, \vec{x})=\log(\theta)\sum_{i=1}^n x_i   -n\theta - \sum_{i=1}^n \log(x_i !)$$
sicché otteniamo subito che $$S(\theta)=\frac{1}{\theta}\sum_{i=1}^n x_i   -n$$
Si deduce allora immediatamente l'equazione di stima $\frac{1}{\theta}\sum_{i=1}^n x_i   -n=0 \Rightarrow \theta=\frac{1}{n}\sum_{i=1}^n x_i$
Ma allora $$\hat{\theta}_n(\vec{X})=\overline{X}_n=\frac{1}{n}\sum_{i=1}^n X_i$$ che è la media campionaria.\\
\\
\item Sia $\vec{X}\coloneqq (X_1,\ldots,X_n)\sim U[(0,\theta)]$. 
Allora $$f_X(x,\theta)\coloneqq \frac{1}{\theta}\mathbbm{1}_{[0,\theta]}(x)$$ 
Perciò $$L(\theta, \vec{x})=\frac{1}{\theta^n}\prod_{i=1}^n\mathbbm{1}_{[0,\theta]}(x_i)=\frac{1}{\theta^n}\mathbbm{1}_{[X_{(n)},+\infty]}(\theta)\Rightarrow \hat{\theta}_n(\vec{X})=X_{(n)}.$$\\
\\
\item Sia $\vec{X}\coloneqq (X_1,\ldots,X_n)\sim \exp(\beta)$, $\beta>0$. 
Allora $f_X(x,\theta)\coloneqq \beta e^{-\beta x}\mathbbm{1}_{\mathbb{R}_+}(x)$. 
Perciò $$L(\theta, \vec{x})=\big(\beta^n e^{-\beta\sum_{i=1}^n x_i}\big)\mathbbm{1}_{\mathbb{R}_+^n}(\vec{x})\Rightarrow l(\theta, \vec{x})=n\log(\beta)-\beta\sum_{i=1}^n x_i$$ 
Otterremo allora l'equazione di stima $0=S(\beta)=n\beta-\sum_{i=1}^n x_i$, da cui subito si deduce che anche in questo caso $\hat{\theta}_n=\overline{X}_n$.\\
\\
\item \textit{(Troncamento)} Sia sempre $\vec{X}\coloneqq (X_1,\ldots,X_n)\sim \exp(\beta)$. Naturalmente ciascuna variabile casuale ha come codominio i reali non negativi. Possiamo supporre di aver effettuato gli $n$ rilevamenti dal campione casuale e di essere riusciti a individuarne esattamente $m$ puntualmente (che senza perdita di generalità immagineremo essere i primi $m$), mentre dei restanti $n-m$ immaginiamo di aver rilevato solamente che il loro valore supera una certa soglia fissta $T>0$. Il campione contiene quindi due tipi di informazione da coniugare nella funzione di massima verosimiglianza, che avrà stavolta una forma un po' diversa. La indicheremo con $L'$. Otteniamo: \\ 

\begin{eqnarray*}
L'(\beta,\vec{x}) &=&\prod_{i=1}^m f_{X_i}(x_i,\beta)\cdot\prod_{i=m+1}^n \mathbb{P}[X_i>T] \\
&=& \prod_{i=1}^m f_{X_i}(x_i,\beta)\cdot\prod_{i=m+1}^n \left(1-F_{X_i}(T,\beta)\right) \\
&=& \prod_{i=1}^m \beta e^{-\beta x_i} \cdot\prod_{i=m+1}^n \int_T^{+\infty} \beta e^{-\beta x_i}dx_i \\
&=& \beta^m e^{-\beta \sum_{i=1}^m x_i} \cdot e^{-\beta (n-m)T}
\end{eqnarray*}

Da cui $$l'(\beta,\vec{x})\coloneqq \log\left(L'(\beta,\vec{x})\right)=m\log(\beta)-\beta\sum_{i=1}^m x_i -\beta (n-m)T.$$ 
Inoltre possiamo definire anche qui una score function nel modo naturale: $$S'(\beta)\coloneqq \frac{\partial}{\partial\beta}l'(\beta,\vec{x})$$ da cui, uguagliando a 0 si può ottenere l'equazione di stima $$\frac{m}{\beta}-\sum_{i=1}^m x_i-(n-m)T=0$$ 
Si deduce così lo stimatore di massima verosimiglianza con troncamento a $T$, dato da $$\hat{\beta}_n'(\vec{X})=\frac{\sum_{i=1}^m X_i +(n-m)T}{m}.$$
\end{enumerate}

\section{Efficienza}
Dato uno stimatore $T_n$ di un campione casuale $\vec{X}\coloneqq (X_1,\ldots,X_n)$ possiamo partire dal concetto di errore quadratico medio $\mse_\theta(T_n)=\var_\theta(T_n)+B_\theta^2(T_n)$. Lo scopo sarà quello di cercare stimatori che minimizzino il più possibile tale valore. Il problema presenta alcune difficoltà: per fare un piccolo esempio, sia $\theta_0\in \Theta$ e consideriamo il seguente stimatore banale $U_n(\vec{X})\coloneqq \theta_0$. È ora evidente che se da una parte $\mse_{\theta_0}(U_n)=0$, sicché nessun altro stimatore può essere uniformemente migliore di $U_n$, dall'altra appare chiaro che di un siffatto stimatore non ci si possa attendere molto, e nemmeno fidare, in quanto esso ignora completamente tutta l'informazione contenuta nel vettore casuale. La difficoltà di trovare stimatori che abbiano errore quadratico medio minimo è dunque legata a due aspetti principali: spesso la struttura di $\mse$ è complicata in quanto contiene aspetti legati al parametro $\theta$; inoltre la classe degli stimatori competitori di $\theta$ è quasi sempre troppo ampia. \\ Cercheremo allora di semplificare il problema restringendo un po' il campo: considereremo solo gli stimatori non distorti, per andare poi a cercare tra questi quelli con varianza minima.  
\\

\textit{Esempio:} Sia $\vec{X}\coloneqq (X_1,\ldots,X_n)\sim Poisson(\lambda)$. In tal caso si verifica subito che $\mathbb{E}[X]=\var[X]=\lambda$. Ne segue che sia lo stimatore media campionaria $\overline{X}_n$ sia lo stimatore varianza campionaria $S_n^2$ sono due stimatori non distorti di $\lambda$. Si ha tuttavia che $\var[\overline{X}_n]=\frac{\lambda}{n}\leq\frac{\lambda}{n}\left(1+\frac{2n\lambda}{n-1}\right)=\var[S_n^2]$. Preferiremo dunque la media campionaria. Ma consideriamo ora il seguente stimatore così definito, per $a\in [0,1]$ fissato, $W_{n,a}(\vec{X})\coloneqq a\overline{X}_n +(1-a)S_n^2$. Anch'esso è non distorto. Sorgono così due difficoltà da affrontare: ammesso che $\overline{X}_n$ sia migliore (i.e. con varianza più piccola) di $S_n^2$, esso è anche migliore di ogni stimatore $W_{n,a} \forall a$ oppure esso è il migliore tra tutti gli stimatori non distorti di $\lambda$? Esiste un limite inferiore alla varianza? Se infatti esso esistesse, darebbe operatività alla scelta dello stimatore, in quanto se trovassimo uno stimatore che raggiunge tale limite, sapremo che non è necessario cercare ulteriormente per migliorare le nostre possibilità. Ebbene, tale limite esiste sicuramente, sotto alcune ulteriori ipotesi di regolarità da addure alla non distorsione per gli stimatori.
\\
\\
\subsection{Teorema di Rao-Cramér}
\textbf{Definizione:} Una \textit{Famiglia Regolare} è una famiglia di densità che soddisfa le seguenti condizioni di regolarità:
\begin{enumerate}[noitemsep]
\item \textit{Condizione di Indentificabilità:} i valori delle densità sono distinti al variare del parametro, ovvero $\theta\neq \theta' \Longrightarrow f_X(x,\theta)\neq f_X(x,\theta')$.
\item Le funzioni densità hanno supporto comune $\forall \theta\in\Theta$ e il loro supporto non dipende in alcun modo dal parametro $\theta$.
\item Le funzioni sono di classe $C^2$ rispetto alla variabile $\theta$
\item Rispetto a $\theta$, è lecito lo scambio tra le derivate e l'integrale.
\end{enumerate}
\textbf{Definizione:} Sia $\vec{X}=(X_1,\ldots, X_n)$ un campione casuale. Allora la funzione
\begin{align*}
I:\Theta &\longrightarrow  \mathbb{R} \\
I(\theta)  &\coloneqq   \mathbb{E}_\theta[S(\theta)^2]=\mathbb{E}_\theta\Bigg[\left(\frac{\partial}{\partial \theta}l(\theta,\vec{X})\right)^2\Bigg]
\end{align*}
viene denominata \textit{Informazione di Fisher} del campione casuale.
\\
\\
\textit{Osservazioni:}
\begin{itemize}
\item Il prossimo teorema ci garantirà nel caso di famiglie regolari che il limite inferiore della varianza di un qualsiasi stimatore non distorto di $\theta$ è la quantità $\frac{1}{I(\theta)}$. Notiamo inoltre che più la varianza di uno stimatore si avvicina a tale quantità, più è significativa la sintesi dell'informazione circa $\theta$ contenuta nel vettore $\vec{X}$ realizzata dallo stimatore non distorto.
\item Spesso si usano anche le seguenti notazioni per l'informazione di Fisher, ovvero $I(\theta)$, $I_n(\theta)$, $nI_1(\theta)$. Infatti dato un vettore casuale qualsiasi $\vec{X}=(X_1\ldots,X_n)$, la sua funzione densità, ovvero $L(\cdot,\cdot)$ non si spezza necessariamente nel prodotto delle densità di ciascuna componente $X_i$. Ciò avviene invece nel caso in cui tutte le variabili casuali siano indipendenti: in tal caso si può scrivere $I_{\vec{X}}(\theta)=\sum_{i=1}^n I_{X_i}(\theta)$. Se poi siamo di fronte ad un campione casuale, allora le variabili casuali sono addirittura $i.i.d.$, e di conseguenza $I_{\vec{X}}(\theta)=nI_{X_1}(\theta)$, da cui la notazione.
\item Si può dimostrare facilmente che, dato un campione casuale $\vec{X}\coloneqq (X_1,\ldots,X_n)$ con densità nella famiglia regolare, vale la seguente uguaglianza: 
$$\mathbb{E}_\theta\Bigg[\left(\frac{\partial}{\partial\theta}l(\theta,\vec{X})\right)^2\Bigg]=-\mathbb{E}_\theta\Bigg[\frac{\partial^2}{\partial\theta^2}l(\theta,\vec{X})\Bigg]$$ 
In effetti, come già visto, si ha: $\frac{\partial}{\partial\theta}L(\theta,\vec{X})=L(\theta,\vec{X})\frac{\partial}{\partial\theta}l(\theta,\vec{X})$. Perciò, derivando si ottiene subito che:
\begin{eqnarray*}
\frac{\partial^2}{\partial\theta^2}L(\theta,\vec{X}) &=& L(\theta,\vec{X})\frac{\partial^2}{\partial\theta^2}l(\theta,\vec{X})+\frac{\partial}{\partial\theta}L(\theta,\vec{X})\frac{\partial}{\partial\theta}l(\theta,\vec{X})\\
&=& L(\theta,\vec{X})\frac{\partial^2}{\partial\theta^2}l(\theta,\vec{X})+L(\theta,\vec{X})\left(\frac{\partial}{\partial\theta}l(\theta,\vec{X})\right)^2.
\end{eqnarray*}
Si può quindi ricavare $\left(\frac{\partial}{\partial\theta}l(\theta,\vec{X})\right)^2=\frac{1}{L(\theta,\vec{X})}\frac{\partial^2}{\partial\theta^2}L(\theta,\vec{X})-\frac{\partial^2}{\partial\theta^2}l(\theta,\vec{X})$. Per provare l'asserto basterà dunque verificare che il valore di aspettazione del primo addendo del secondo termine dell'uguaglianza sia nullo. Si ha:
\begin{eqnarray*}
\mathbb{E}\Big[\frac{1}{L(\theta,\vec{X})}\frac{\partial^2}{\partial\theta^2}L(\theta,\vec{X})\Big]
&=& \int_{\mathbb{R}^n} \frac{1}{L(\theta,\vec{x})}\frac{\partial^2}{\partial\theta^2}L(\theta,\vec{x})\cdot L(\theta,\vec{x})\cdot d\vec{x} \\
&=& \int_{\mathbb{R}^n} \frac{\partial^2}{\partial\theta^2}L(\theta,\vec{x})\cdot d\vec{x} \\
&=& \frac{\partial^2}{\partial\theta^2}\int_{\mathbb{R}^n} L(\theta,\vec{x})\cdot d\vec{x} \\
&=& \frac{\partial^2}{\partial\theta^2} 1=0.
\end{eqnarray*}
\end{itemize}
\textbf{Definizione:} Sia $\vec{X}\coloneqq (X_1,\ldots,X_n)$ un campione casuale e $T_n=T_n(\vec{X})$, $V_n=V_n(\vec{X})$ due stimatori non distorti di $\theta$. Allora:
\begin{itemize}[noitemsep]
\item Diremo \textit{Efficienza assoluta o di Bahadur} di $T_n$ il valore $\eff(T_n)\coloneqq \frac{\frac{1}{I(\theta)}}{\var_\theta[T_n]}$.
\item Diremo \textit{Efficienza relativa} di $T_n$ e $V_n$ il valore $\eff(T_n,V_n)\coloneqq \frac{\var_\theta[T_n]}{\var_\theta[V_n]}$.
\item Diremo che $T_n$ è \textit{Efficiente} se $\eff(T_n)=1$. Nel caso in cui $\eff(T_n)>1$ lo stimatore $T_n$ si dirà anche \textit{Super-Efficiente}. In generale, si dirà che $T_n$ è più (meno) efficiente di $V_n$ se $\eff(T_n,V_n)< (>)1$.
\item Diremo che $T_n$ è \textit{Asintoticamente Efficiente} se $\lim_{n\rightarrow\infty} \eff(T_n)=1$.
\end{itemize}
\begin{teo}[di Rao-Cramér]
Sia $\vec{X}\coloneqq (X_1,\ldots,X_n)$ un campione casuale di densità $f_{\vec{X}}(\vec{x},\theta)$ appartenente alla famiglia regolare con $\theta\in \Theta \subset \mathbb{R}$ un insieme di parametri. Sia poi $g:\Theta\longrightarrow \Theta$ una funzione derivabile e assumiamo l'informazione di Fisher $I(\theta)\neq 0$ $\forall \theta\in \Theta$. Allora, per qualsiasi stimatore $T_n=T_n(\vec{X})$ non distorto del parametro $g(\theta)$, vale $\var_\theta[T_n]\geq \big(g'(\theta)\big)^2\cdot\frac{1}{I(\theta)}$.
\end{teo}

\begin{proof}
Poiché $T_n$ è uno stimatore non distorto di $g(\theta)$, abbiamo:\\
$g(\theta)=\mathbb{E}_\theta [T_n]=\int_{\mathbb{R}^n} T_n(\vec{x})f_{\vec{X}}(\vec{x},\theta)d\vec{x}=\int_{\mathbb{R}^n} T_n(\vec{x})L(\theta,\vec{x})d\vec{x}$, con $\theta\in \Theta$. Perciò, derivando sotto il parametro $\theta$ e grazie alle ipotesi di regolarità otteniamo: \\
\begin{eqnarray*}
g'(\theta) &=& \frac{\partial}{\partial\theta}\int_{\mathbb{R}^n} T_n(\vec{x})L(\theta,\vec{x})d\vec{x}= \int_{\mathbb{R}^n} T_n(\vec{x})\left(\frac{\partial}{\partial\theta}L(\theta,\vec{x})\right)d\vec{x} \\
&=& \int_{\mathbb{R}^n} T_n(\vec{x})L(\theta,\vec{x})\left(\frac{\partial}{\partial\theta}l(\theta,\vec{x})\right)d\vec{x} = \int_{\mathbb{R}^n} T_n(\vec{x})\left(\frac{\partial}{\partial\theta}l(\theta,\vec{x})\right)f_{\vec{X}}(\vec{x},\theta)d\vec{x} \\
&=& \mathbb{E}_\theta \Big[T_n(\vec{X})\cdot\frac{\partial}{\partial\theta}l(\theta,\vec{X})\Big].
\end{eqnarray*}
Osserviamo ora per prima cosa che:
\begin{eqnarray*}
\mathbb{E}_\theta \Big[\frac{\partial}{\partial\theta}l(\theta,\vec{X})\Big] &= & \int_{\mathbb{R}^n}  \left(\frac{\partial}{\partial\theta}l(\theta,\vec{x})\right)L(\theta,\vec{x})d\vec{x}=\int_{\mathbb{R}^n} \left(\frac{\partial}{\partial\theta}L(\theta,\vec{x})\right)d\vec{x}\\
&=& \frac{\partial}{\partial\theta}\int_{\mathbb{R}^n}L(\theta,\vec{x})d\vec{x}=\frac{\partial}{\partial\theta}\int_{\mathbb{R}^n}f_{\vec{X}}(\vec{x},\theta)d\vec{x}\\ 
&=& \frac{\partial}{\partial\theta} 1=0
\end{eqnarray*}
Ne seguono direttamente le due seguenti relazioni:
\begin{itemize}
\item $\cova_\theta \big[T_n(\vec{X}),\frac{\partial}{\partial\theta}l(\theta,\vec{X})\big]=\mathbb{E}_\theta\big[T_n(\vec{X})\cdot\frac{\partial}{\partial\theta}l(\theta,\vec{X})\big]-\mathbb{E}_\theta\big[T_n(\vec{X})\big]\cdot \mathbb{E}_\theta\big[\frac{\partial}{\partial\theta}l(\theta,\vec{X})\big]=\mathbb{E}_\theta\big[T_n(\vec{X})\cdot\frac{\partial}{\partial\theta}l(\theta,\vec{X})\big]-\mathbb{E}_\theta\big[T_n(\vec{X})\big]\cdot 0=\mathbb{E}_\theta\big[T_n(\vec{X})\cdot\frac{\partial}{\partial\theta}l(\theta,\vec{X})\big]=g'(\theta)$;

\item $\var_\theta \big[\frac{\partial}{\partial\theta}l(\theta,\vec{X})\big]= \mathbb{E}_\theta\bigg[\left(\frac{\partial}{\partial\theta}l(\theta,\vec{X})\right)^2\bigg]-\mathbb{E}_\theta\big[\frac{\partial}{\partial\theta}l(\theta,\vec{X})\big]^2=\mathbb{E}_\theta\bigg[\left(\frac{\partial}{\partial\theta}l(\theta,\vec{X})\right)^2\bigg]$, di conseguenza $\var_\theta \big[\frac{\partial}{\partial\theta}l(\theta,\vec{X})\big]=I_n(\theta).$
\end{itemize}
D'altra parte, dalla disuguaglianza di Cauchy-Schwarz abbiamo:
\begin{eqnarray*}
\big(g'(\theta)\big)^2 &=& \cova_\theta \Big[T_n(\vec{X}),\frac{\partial}{\partial\theta}l(\theta,\vec{X})\Big]^2\\ &\leq & \var_\theta[T_n(\vec{X})]\cdot\var_\theta \Big[\frac{\partial}{\partial\theta}l(\theta,\vec{X})\Big]=\var_\theta[T_n(\vec{X})]\cdot I_n(\theta),
\end{eqnarray*}
grazie alle relazioni appena introdotte. La tesi segue subito, ricordando che sia la varianza che l'informazione di Fisher sono quantità positive.
\end{proof} 

\textit{Controesempio:} Le ipotesi di regolarità del teorema sono necessarie. Consideriamo infatti il cmpione casuale $\vec{X}\coloneqq (X_1,\ldots,X_n)\sim U([0,\theta])$. La sua densità non appartiene alla famiglia regolare in quanto ha il supporto dipendente dal parametro $\theta$. Uno stimatore non distorto di $\theta$ abbiamo già visto essere $T_n(\vec{X})\coloneqq \frac{n-1}{n}X_{(n)}$. Tuttavia $\var[T_n]<\frac{1}{I(\theta)}$ e di conseguenza è stimatore super-efficiente. La tesi del teorema non è dunque valida in questo caso.
\\
\\
\textbf{Lemma 1.} \textit{Sotto le usuali condizioni di regolarità, esiste uno stimatore non distorto $T_n$ di $\theta$ efficiente, ossia tale che la sua varianza raggiunge il limite inferiore di Rao-Cramér, se e solo se $S(\theta)=\frac{\partial}{\partial\theta}l(\theta,\vec{X})=I_n(\theta)\left(T_n(\vec{X})-\theta\right)$}.

\begin{proof}
Grazie alla disuguaglianza di Cauchy-Schwarz abbiamo la seguente relazione $\cova_\theta^2[T_n(\vec{X}), \frac{\partial}{\partial\theta}l(\theta,\vec{X})]\leq \var_\theta[T_n(\vec{X})]\cdot\var_\theta[\frac{\partial}{\partial\theta}l(\theta,\vec{X})]$, nella  quale sussiste l'uguaglianza sse vi è linarità tra i due termini, ovvero sse $\exists a,b\in \mathbb{R}$ tali che $\frac{\partial}{\partial\theta}l(\theta,\vec{X})=a+bT_n(\vec{X})$. Come già calcolato nella precedente dimostrazione, il valore di aspettazione del primo membro dell'uguaglianza è nullo, perciò $0=\mathbb{E}_\theta[\frac{\partial}{\partial\theta}l(\theta,\vec{X})]= \mathbb{E}_\theta[a+bT_n(\vec{X})]=\mathbb{E}_\theta[a]+\mathbb{E}_\theta[bT_n(\vec{X})]=a+b\theta \Rightarrow a=-b\theta$. Quindi $\frac{\partial}{\partial\theta}l(\theta,\vec{X})=b(T_n(\vec{X})-\theta)$. Se moltiplichiamo tutto per $\frac{\partial}{\partial\theta}l(\theta,\vec{X})$ abbiamo che $\left(\frac{\partial}{\partial\theta}l(\theta,\vec{X})\right)^2=bT_n(\vec{X})\frac{\partial}{\partial\theta}l(\theta,\vec{X})-b\theta \frac{\partial}{\partial\theta}l(\theta,\vec{X})$. Calcolando infine nuovamente il valore di aspettazione e riprendendo alcuni risultati ottenuti dalla dimostrazione del teorema di Rao-Cramér abbiamo che: \\
$I(\theta)=\mathbb{E}_\theta\Big[\left(\frac{\partial}{\partial\theta}l(\theta,\vec{X})\right)^2\Big]=b\mathbb{E}_\theta[T_n(\vec{X})\frac{\partial}{\partial\theta}l(\theta,\vec{X})]-b\theta\mathbb{E}_\theta [\frac{\partial}{\partial\theta}l(\theta,\vec{X})]=b\cdot 1-b\theta\cdot 0$ e di conseguenza si ha $b=I(\theta)$, da cui si deduce immediatamente la tesi.

\end{proof}
\textit{Esempio:} Consideriamo ancora $\vec{X}\coloneqq (X_1,\ldots,X_n)\sim Poisson(\lambda)$. Sappiamo che la sua densità appartiene alla famiglia regolare. Avevamo introdotto $\forall a\in [0,1]$ fissato gli stimatori non distorti $W_{n,a}(\vec{X})\coloneqq a\overline{X}_n +(1-a)S_n^2$ e ci eravamo chiesti quale fosse il migliore. Ebbene, tra tutti essi, la risposta è proprio $W_{n,1}=\overline{X}_n$, la media campionaria. Infatti si ha, come già visto, che la score function è data da $S(\lambda)=-n+\frac{1}{\lambda}\sum_{i=1}^n X_i=n\left(\frac{1}{\lambda}\overline{X}_n-1\right)$. Se ora calcoliamo $I_n(\lambda)=-\mathbb{E}_{\lambda}[\frac{\partial^2}{\partial\lambda^2}l(\lambda,\vec{X})]= -\mathbb{E}_{\lambda}[\frac{d}{d\lambda} S(\lambda)]= -\mathbb{E}_{\lambda}[-\frac{1}{\lambda^2}\overline{X}_n]=\frac{1}{\lambda^2}\cdot n\lambda=\frac{n}{\lambda}$, otteniamo che $S(\lambda)= (\overline{X}_n-\lambda)\cdot \frac{n}{\lambda}=(\overline{X}_n-\lambda)\cdot I(\lambda)$ e possiamo concludere grazie il Lemma 1.
\\
\\
\textbf{Lemma 2.} \textit{Sotto le usuali ipotesi di regolarità, sia $I(\theta)\neq 0$ $\forall \theta\in \Theta$ e supponiamo che esista uno stimatore $T_n$ non distorto di $\theta$ efficiente. Se $\hat{\theta}_n$ è lo stimatore di massima verosimiglianza di $\theta$, allora vale $T_n=\hat{\theta}_n$.}

\begin{proof}
Il limite inferiore di Rao-Cramér non è una quantità nulla. Inoltre come già osservato e grazie al Lemma 1 si ha:
$$0=S\big(\hat{\theta}_n(\vec{X})\big)= \big(T_n(\vec{X})-\hat{\theta}_n(\vec{X})\big)I_n\big(\hat{\theta}_n(\vec{X})\big),$$ da cui $T_n(\vec{X})-\hat{\theta}_n(\vec{X})=0$ e quindi la tesi.

\end{proof}
\textit{Controesempio:} Non sempre lo stimatore di massima verosimiglianza è anche stimatore efficiente, e dunque, per il Lemma 2, non sempre esiste uno stimatore efficiente. Sia infatti 
$\vec{X}\coloneqq (X_1,\ldots,X_n)\sim f_X(x,\theta)\coloneqq \theta x^{\theta -1}\cdot\mathbbm{1}_{(0,1)}(x)$ campione casuale, con $\theta>0$. 
Ora, $\frac{\partial^2}{\partial\theta^2}\log \big(f(x,\theta)\big)=-\frac{1}{\theta^2}\Rightarrow I_1(\theta)=-\mathbb{E}[-\frac{1}{\theta^2}]=\frac{1}{\theta^2}\Rightarrow I_n(\theta)=\frac{n}{\theta^2}$. Però $S(\theta)=\frac{\partial}{\partial\theta}l(\theta,\vec{X})=\frac{\partial}{\partial\theta}\log\left(\prod_{i=1}^n \theta X_i^{\theta -1}\right)=\frac{n}{\theta}+\sum_{i=1}^n \log(X_i).$ L'equazione di stima $S(\theta)=0$ ci fornisce allora $\hat{\theta}_n(\vec{X})=\frac{n}{\sum_{i=1}^n \log(X_i)}$, lo stimatore di massima verosimiglianza. Vogliamo ora trovare la sua distribuzione. Definiamo innanzi tutto il nuovo vettore casuale $\vec{Y}\coloneqq (Y_1,\ldots,Y_n)$ dove $\forall i=1..n$ si ha $Y_i \coloneqq \log(X_i)$. Osserviamo che il logaritmo è una funzione monotona crescente, e possiamo applicare il teorema 1.1 per ottenere che la densità delle nuove variabili è $f_Y(y,\theta)=\theta (e^{-y})^{\theta -1}|-e^{-y}|\cdot \mathbbm{1}_{\mathbb{R}_+}(y)=\theta e^{-\theta y}\mathbbm{1}_{\mathbb{R}_+}(y)$, e $\theta>0.$ Dunque, $\vec{Y}\sim G(\alpha =1, \beta = \frac{1}{\theta})$. Poiché $\vec{X}$ è un vettore indipendente, segue necessariamente che anche $\vec{Y}$ lo sia; quindi, grazie alla proprietà di riproducibilità della densità Gamma $W\coloneqq \sum_{i=1}^n Y_i \sim G(\alpha' =n, \beta=\frac{1}{\theta})$. Si può mostrare che: $$\mathbb{E}[W^k]=\frac{(n+k-1)!}{\theta (n-1)!}.$$ Ricordando che $\hat{\theta}_n=nW^{-1}$ possiamo calcolare subito i valori di aspettazione
\begin{itemize}[noitemsep]
\item $\mathbb{E}_\theta[\hat{\theta}_n]=\mathbb{E}_\theta [nW^{-1}]=n\mathbb{E}_\theta [W^{-1}]=\frac{n}{n-1}\theta \neq \theta$, perciò è stimatore distorto, anche se asintoticamente non distorto.
\item $\mathbb{E}[(\hat{\theta}_n)^2]=\mathbb{E}[n^2 W^{-2}]=n^2\mathbb{E}[W^{-2}]=\frac{\theta^2 n^2}{(n-2)(n-1)}$
\end{itemize}
e dunque $\var[\hat{\theta}_n]=\mathbb{E}[(\hat{\theta}_n)^2]-\mathbb{E}[\hat{\theta}_n]^2=\frac{n^2\theta^2}{(n-1)^2(n-2)^2}>\frac{1}{I(\theta)}=\frac{\theta}{n}.$ Ne segue che $\hat{\theta}_n$ non è stimatore efficiente di $\theta$ anche se $\eff(\hat{\theta}_n) \xrightarrow[n\rightarrow \infty]{} 1$.

\subsection{Estensione a un vettore di parametri:}
Possiamo, al posto di un singolo parametro, andare a considerare un vettore di parametri $\vec{\theta}\coloneqq (\theta_1,\ldots,\theta_k)\in \Theta^k$, $\Theta\subset\mathbb{R}$ che indicizza la distribuzione di una variabile casuale $X$. Ad esempio la distribuzione Gamma dipende da due parametri solitamente indicati con $\alpha$ e $\beta$. In particolare, modellare un fenomeno con un numero di parametri che sia il più piccolo possibile assume un valore importante per quanto riguarda la stabilità degli stimatori. A ciò è stato dato il nome piuttosto eloquente di \textit{Principio di Parsimonia}. Nel caso di un vettore di parametri si potrà allora estendere il concetto di Informazione di Fisher ottenendo una matrice.
\\
\\
\textbf{Definizione.} Sia $\vec{X}$ un campione casuale e $\vec{\theta}\coloneqq (\theta_1,\ldots,\theta_k)$ un campione di parametri. Allora la \textit{Matrice di Informazione di Fisher} è la matrice $I(\vec{\theta})\in \mathcal{M}(k\times k, \mathbb{R})$ il cui $i$-$j$-esimo elemento è dato dal numero $$\mathbb{E}_{\vec{\theta}}\Big[\frac{\partial}{\partial\theta_i}l(\vec{\theta},\vec{X})\cdot \frac{\partial}{\partial\theta_j}l(\vec{\theta},\vec{X})\Big].$$
\\
\\
\textbf{Proposizione.} (*) \textit{Per ogni vettore casuale $\vec{X}$ si ha che la matrice di informazione di Fisher è simmetrica e semidefinita positiva.}
\begin{proof}(*)
Il fatto che sia simmetrica è pressoché immediato e viene direttamente dalla definizione. Per mostrare che è anche semidefinita positiva, sia $w(\vec{\theta})\coloneqq \big(\frac{\partial}{\partial\theta_1}l(\vec{\theta},\vec{X}), \ldots, \frac{\partial}{\partial\theta_k}l(\vec{\theta},\vec{X})\big)$. Si vede subito che $I(\vec{\theta})=\mathbb{E}_{\vec{\theta}}[w(\vec{\theta})^t\cdot w(\vec{\theta})].$ Sia ora $\vec{u}\in \mathbb{R}^k\setminus\{0\}$. Dobbiamo mostrare che $(u\cdot I(\vec{\theta})\cdot u^t)\geq 0$. Ebbene, sfruttando la linearità del valore di aspettazione, otteniamo che: 
\begin{eqnarray*}
u I(\vec{\theta})u^t&=& u \mathbb{E}_{\vec{\theta}}[w(\vec{\theta})^t\cdot w(\vec{\theta})] u^t=\mathbb{E}_{\vec{\theta}}[u\cdot w(\vec{\theta})^t\cdot w(\vec{\theta})\cdot u^t] \\
&=& \mathbb{E}_{\vec{\theta}}[u\cdot w(\vec{\theta})^t\cdot ((u\cdot w(\vec{\theta})^t)^t] \\
&=& \mathbb{E}_{\vec{\theta}}[\|u\cdot w(\vec{\theta})^t\|^2]\geq 0.
\end{eqnarray*}
\end{proof}
\textit{Osservazione:} Osserviamo che le ipotesi di regolarità definite per il caso unidimensionale, possono essere espanse al caso $k$-dimensionale nel modo più naturale, ovvero supponendo che esse valgano per ciascuno dei parametri $\theta_i$, $\forall 1\leq i\leq k$. Ebbene, sotto le usuali ipotesi di regolarità, si ottiene facilmente con quanto già mostrato che $I(\theta)_{ij}=-\mathbb{E}_{\vec{\theta}}\big[\frac{\partial^2}{\partial\theta_i\partial\theta_j}l(\vec{\theta},\vec{x})\big]$. Osserviamo che vi è coerenza con la simmetria della matrice di informazione: essendo le densità di classe $C^2$, vale il teorema di Schwarz sullo scambio delle derivate. 
\\
\\
\textbf{Lemma.} (*) \textit{Siano $A\in \mathcal{M}(n\times n, \mathbb{R})$ una matrice simmetrica e definita positiva, e $b\in \mathbb{R}^n$. Allora, se definiamo la funzione 
\begin{eqnarray*}
f : \mathbb{R}^n &\longrightarrow & \mathbb{R} \\
f(x) &\coloneqq & x\cdot A \cdot x^t -2b\cdot x^t
\end{eqnarray*}
abbiamo che $f$ ha un unico punto di minimo $\hat{x}\coloneqq b\cdot A^{-1}$.}



\begin{proof}(*)
Scriviamo per semplicità $x=(x_1,\ldots,x_n)$, $b=(b_1,\ldots,b_n)$ e $A=(a_{ij})_{ij}$. utilizzeremo il metodo della matrice hessiana. Abbiamo che:
$$f(x)=\sum_{i,j=1}^n x_i a_{ij}x_j  -   2\sum_{h=1}^n b_h x_h= \sum_{k=1}^n a_{kk}x_k^2  +  2\sum_{i=2}^n\sum_{j=1}^{i-1} x_i a_{ij} x_j   -2\sum_{h=1}^n b_h x_h $$ dove l'ultima uguaglianza viene direttamente dal fatto che $A$ è una matrice simmetrica. Definendo ora $\sum_{1}^0 \coloneqq 0$,  calcoliamo la $r$-esima derivata, $\forall 1\leq r\leq n$:
\begin{eqnarray*}
\frac{\partial}{\partial x_r}f(x) &=& 2a_{rr}x_r +2\sum_{j=1}^{r-1}  a_{rj}x_j +2\sum_{i=r+1}^n x_i a_{ir}  -2b_r  \\
&=& 2a_{rr}x_r +2\sum_{j=1}^{r-1}  a_{rj}x_j +2\sum_{i=r+1}^n x_i a_{ri}   -2b_r \\
&=& 2a_{rr}x_r +2\sum_{s=1, s\neq r}^n a_{sr}x_s  -2b_r \\
&=& 2\sum_{s=1}^n a_{rs}x_s  -2b_r = 2(A_r \cdot x^t) -2b_r,
\end{eqnarray*}
ove $A_r$ è la $r$-esima riga della matrice $A$. Ora, per trovare i possibili punti stazionari sarà necessario eguagliare a 0 tutte le derivate parziali e risolvere il sistema. Non avendo termini quadratici esso sarà lineare:

$$
\left\{
\begin{array}{lr}
\frac{\partial}{\partial x_1}f(x)  =  0 \\
\vdots   \\
\frac{\partial}{\partial x_n}f(x)  =  0 
\end{array}
\right.
\Leftrightarrow
\left\{
\begin{array}{lr} 
2(A_1 \cdot x^t) -2b_1 =0 \\
\vdots \\
2(A_n \cdot x^t) -2b_n =0
\end{array}
\right.
\Leftrightarrow
\left\{
\begin{array}{lr} 
A_1 \cdot x^t= b_1  \\
\vdots \\
A_n \cdot x^t = b_n
\end{array}
\right.
\Leftrightarrow
A\cdot x^t = b^t.
$$
Allora, poiché la matrice $A$ è invertibile, otterremo l'unico punto stazionario $\hat{x}^t=A^{-1}\cdot b^t$, ed essendo la matrice simmetrica, sarà $\hat{x}=b\cdot A^{-1}$. Per verificare adesso che si tratta di un punto di minimo, andiamo a calcolare la matrice hessiana di tutte le derivate seconde. Palesemente $f$ è una funzione di classe $C^{\infty}$, ragion per cui deve valere il teorema di Schwarz sullo scambio delle derivate seconde. In generale $\forall 1\leq s,t \leq n$, si avrà:
$$\frac{\partial^2}{\partial x_s\partial x_t} f(x)=\frac{\partial}{\partial x_s}\left(\frac{\partial}{\partial x_t} f(x)\right)=\frac{\partial}{\partial x_s}\left(2\sum_{i=1}^n a_{ti}x_i  -2b_t\right)=2a_{ts}.$$
Ne segue quindi che la matrice hessiana di f sarà $\forall x\in \mathbb{R}^n$ data da $2A$. Essa risulta quindi definita positiva e conferma di conseguenza che $\hat{x}$ è un punto di minimo.
\end{proof}

%\newpage
$\vspace{1cm}$

\begin{teo}[di Rao-Cramér] Sia $\vec{X}\coloneqq (X_1,\ldots,X_n)$ un campione casuale di densità $f_{\vec{X}}(\vec{x},\vec{\theta})$ appartenente alla famiglia regolare e $\theta\in \Theta \subset \mathbb{R}^k$ un insieme di parametri. Sia poi $g:\Theta\longrightarrow \mathbb{R}$ una funzione derivabile e assumiamo che la matrice di informazione di Fisher $I(\theta)$ sia invertibile $\forall \theta\in \Theta$. Definiamo ora il vettore $\gamma(\vec{\theta})\coloneqq \big(\frac{\partial}{\partial\theta_1}g(\vec{\theta}),\ldots, \frac{\partial}{\partial\theta_n}g(\vec{\theta})\big)$. Allora, per ogni stimatore $T=T(\vec{X})$ non distorto del numero $g(\vec{\theta})$ si ha che $\var[T]\geq \gamma(\vec{\theta})\cdot I^{-1}(\vec{\theta}) \cdot \gamma(\vec{\theta})^{t}$.
\end{teo}


\begin{proof} (*)
La dimostrazione si articola sfruttando alcuni passaggi già utilizzati nel caso unidimensionale. Innanzi tutto $\forall 1\leq j \leq k$ si ha:
\begin{eqnarray*}
\frac{\partial}{\partial\theta_j}g(\vec{\theta}) &=& \frac{\partial}{\partial\theta_j}\mathbb{E}_{\vec{\theta}}[T]=\frac{\partial}{\partial\theta_j}\int_{\mathbb{R}^n} T(\vec{x})f_{\vec{X}}(\vec{x},\vec{\theta})d\vec{x}=\frac{\partial}{\partial\theta_j}\int_{\mathbb{R}^n} T(\vec{x})L(\vec{\theta},\vec{x})d\vec{x} \\
&=& \int_{\mathbb{R}^n} T(\vec{x})\frac{\partial}{\partial\theta_j}L(\vec{\theta},\vec{x})d\vec{x}= \int_{\mathbb{R}^n} T(\vec{x})\left(\frac{\partial}{\partial\theta_j}l(\vec{\theta},\vec{x})\right)L(\vec{\theta},\vec{x}) d\vec{x} \\
&=& \mathbb{E}_{\vec{\theta}}\Big[T(\vec{x})\cdot\left(\frac{\partial}{\partial\theta_j}l(\vec{\theta},\vec{x})\right)\Big].
\end{eqnarray*}
Osserviamo allora che essendo 
\begin{eqnarray*}
\mathbb{E}_{\vec{\theta}} \Big[\frac{\partial}{\partial\theta_j}l(\vec{\theta},\vec{X})\Big] &= & \int_{\mathbb{R}^n}  \frac{\partial}{\partial\theta_j}l(\vec{\theta},\vec{x})L(\vec{\theta},\vec{x})d\vec{x}=\int_{\mathbb{R}^n} \left(\frac{\partial}{\partial\theta_j}L(\vec{\theta},\vec{x})\right)d\vec{x}\\
&=& \frac{\partial}{\partial\theta_j}\int_{\mathbb{R}^n}L(\vec{\theta},\vec{x})d\vec{x}=\frac{\partial}{\partial\theta_j}\int_{\mathbb{R}^n}f_{\vec{X}}(\vec{x},\vec{\theta})d\vec{x}\\ 
&=& \frac{\partial}{\partial\theta_j} 1=0,
\end{eqnarray*}
$\cova_{\vec{\theta}} \big[T_n(\vec{X}),\frac{\partial}{\partial\theta_j}l(\vec{\theta},\vec{X})\big]=\mathbb{E}_{\vec{\theta}}\big[T_n(\vec{X})\cdot\frac{\partial}{\partial\theta_j}l(\vec{\theta},\vec{X})\big]-\mathbb{E}_{\vec{\theta}}\big[T_n(\vec{X})\big]\cdot \mathbb{E}_{\vec{\theta}}\big[\frac{\partial}{\partial\theta_j}l(\vec{\theta},\vec{X})\big]=\mathbb{E}_{\vec{\theta}}\big[T_n(\vec{X})\cdot\frac{\partial}{\partial\theta_j}l(\vec{\theta},\vec{X})\big]-\mathbb{E}_{\vec{\theta}}\big[T_n(\vec{X})\big]\cdot 0=\mathbb{E}_{\vec{\theta}}\big[T_n(\vec{X})\cdot\frac{\partial}{\partial\theta_j}l(\vec{\theta},\vec{X})\big]=\frac{\partial}{\partial\theta_j}g(\vec{\theta}).$
\\
\\
Fatto ciò, sia ora $\vec{c}=(c_1,\ldots,c_k)\in \mathbb{R}^k$ e definiamo ora la seguente funzione: $W(\vec{x}, \vec{\theta})\coloneqq \sum_{i=1}^k c_i \frac{\partial}{\partial \theta_i} l(\vec{\theta}, \vec{x})$. È chiaro che $W(\vec{X}, \vec{\theta})$ sarà allora una variabile casuale. Vogliamo calcolare il valore di $\var[T(\vec{X})-W(\vec{X}, \vec{\theta})]$. Per prima cosa osserviamo che, sfruttando la linearità del valore atteso,
\begin{eqnarray*}
\var[T-W] &=& \mathbb{E}[(T-W)^2]-\mathbb{E}[(T-W)]^2 \\
&=& \mathbb{E}[T^2-2TW+W^2]-(\mathbb{E}[T]-\mathbb{E}[W])^2 \\
&=& \mathbb{E}[T^2]-2\mathbb{E}[TW]+\mathbb{E}[W^2]-\mathbb{E}[T]^2 +2\mathbb{E}[T]\mathbb{E}[W] -\mathbb{E}[W]^2 \\
&=& \var[T] -2\cova[T, W] +\var[W]
\end{eqnarray*}
Adesso osserviamo invece che i conti si possono semplificare in quanto
$$
\mathbb{E}[W(\vec{X}, \vec{\theta})]=\mathbb{E}\Big[\sum_{i=1}^k c_i \frac{\partial}{\partial \theta_i} l(\vec{\theta}, \vec{X})\Big]=\sum_{i=1}^k c_i \mathbb{E}\Big[\frac{\partial}{\partial \theta_i} l(\vec{\theta}, \vec{X})\Big]=\sum_{i=1}^k c_i\cdot 0 = 0.
$$
Perciò otteniamo subito che:
\begin{eqnarray*}
\cova[T(\vec{X}), W(\vec{X},\vec{\theta})]&=&\mathbb{E}[T(\vec{X}) W(\vec{X},\vec{\theta})]-\mathbb{E}[T(\vec{X})]\mathbb{E}[ W(\vec{X},\vec{\theta})] \\
&=& \mathbb{E}[T(\vec{X}) W(\vec{X},\vec{\theta})] \\
&=& \mathbb{E}\Big[\sum_{i=1}^k c_i T(\vec{X})\frac{\partial}{\partial \theta_i} l(\vec{\theta}, \vec{X})\Big]=\sum_{i=1}^k c_i\mathbb{E}\Big[T(\vec{X})\frac{\partial}{\partial \theta_i} l(\vec{\theta}, \vec{X})\Big] \\
&=& \sum_{i=1}^k c_i \cova_{\vec{\theta}} \big[T_n(\vec{X}),\frac{\partial}{\partial\theta_i}l(\vec{\theta},\vec{X})\big] = \sum_{i=1}^k c_i \frac{\partial}{\partial\theta_i}g(\vec{\theta}) \\
&=& \vec{c}\cdot \gamma(\vec{\theta})^t;
\end{eqnarray*}
\begin{eqnarray*}
\var[W(\vec{X}, \vec{\theta})]&=&\mathbb{E}[\big(W(\vec{X}, \vec{\theta})\big)^2]-\mathbb{E}[W(\vec{X}, \vec{\theta})]^2=\mathbb{E}[\big(W(\vec{X}, \vec{\theta})\big)^2] \\
&=& \mathbb{E}\Big[\big(\sum_{i=1}^k c_i \frac{\partial}{\partial\theta_i} l(\vec{\theta}, \vec{X})\big)^2\Big] \\
&=& \mathbb{E}\Big[\sum_{i=1}^k \sum_{j=1}^k \left(c_i \frac{\partial}{\partial\theta_i} l(\vec{\theta}, \vec{X})\right)\left(\frac{\partial}{\partial\theta_j} l(\vec{\theta}, \vec{X}) c_j\right)\Big] \\
&=& \sum_{i=1}^k \sum_{j=1}^k   c_i \mathbb{E}\Big[\frac{\partial}{\partial\theta_i} l(\vec{\theta}, \vec{X})\cdot \frac{\partial}{\partial\theta_j} l(\vec{\theta}, \vec{X})\Big]c_j \\
&=& \sum_{i=1}^k \sum_{j=1}^k   c_i I_{ij}(\vec{\theta})c_j = \vec{c}\cdot I(\vec{\theta})\cdot (\vec{c})^t.
\end{eqnarray*}
Osservando infine che la varianza di una qualsiasi variabile casuale è un numero sempre positivo, otteniamo la relazione definitiva: 
$$0\leq \var[T(\vec{X})-W(\vec{X}, \vec{\theta})]= \var[T(\vec{X})] + \vec{c}\cdot I(\vec{\theta})\cdot (\vec{c})^t -2\vec{c}\cdot \gamma(\vec{\theta})^t.$$
Poiché ciò deve valere $\forall \vec{c}\in \mathbb{R}^k$ possiamo allore scrivere:
$$ 0\leq \min_{\vec{c}\in \mathbb{R}^k} \{\var[T(\vec{X})-W(\vec{X}, \vec{\theta})]\}=\var[T(\vec{X})] + 
 \min_{\vec{c}\in \mathbb{R}^k} \{\vec{c}\cdot I(\vec{\theta})\cdot (\vec{c})^t -2\vec{c}\cdot \gamma(\vec{\theta})^t\}.$$
Ora, la matrice di informazione di Fisher è per ipotesi invertibile, sicché, grazie alla precedente proposizione, essa è quindi definita positiva. Possiamo allora applicare il lemma, secondo cui il minimizzatore è: 
$$\hat{c}\coloneqq \arg\big\{\min_{\vec{c}\in \mathbb{R}^k} \{\vec{c}\cdot I(\vec{\theta})\cdot (\vec{c})^t -2\vec{c}\cdot \gamma(\vec{\theta})^t\}\big\}= \gamma(\vec{\theta})\cdot I^{-1}(\vec{\theta}).$$
Sostituendo allora $\hat{c}$ nella varianza e ricordando che $I(\vec{\theta})$ è simmetrica (e quindi anche la sua inversa) si ottiene infine la tesi:
\begin{eqnarray*}
0\leq \min_{\vec{c}\in \mathbb{R}^k} \{\var[T(\vec{X})-W(\vec{X}, \vec{\theta})]\} &=& \var[T(\vec{X})] + \hat{c}\cdot I(\vec{\theta})\cdot (\hat{c})^t -2\hat{c}\cdot \gamma(\vec{\theta})^t \\
&=& \var[T(\vec{X})] - \gamma(\vec{\theta})\cdot I^{-1}(\vec{\theta}) \cdot \gamma(\vec{\theta})^t.
\end{eqnarray*}
\end{proof}


\textbf{Corollario.} \textit{Nelle ipotesi del teorema di Rao-Cramér, se $\forall 1\leq i,j\leq k$ abbiamo che $T_j=T_j(\vec{X})$ è uno stimatore non distorto del parametro $\theta_j$, allora $\var[T_j(\vec{X})]\geq I_{jj}^{-1}(\vec{\theta})$. }
\\
\\
\textit{Esempio:} Sia $\vec{X}\coloneqq (X_1,\ldots,X_n)\sim N(\mu, \sigma^2)$ un campione casuale. 
Allora  $$L(\mu,\sigma^2, \vec{x})=\prod_{i=1}^n \frac{1}{\sqrt{2\pi\sigma^2}} e^{-\frac{1}{2\sigma^2}(x_1-\mu)^2}= (2\pi\sigma^2)^{-\frac{n}{2}} e^{\frac{1}{2\sigma^2}\sum_{i=1}^n (x_i-\mu)^2}$$ La funzione di log-verosimiglianza sarà allora $$l(\mu,\sigma^2,\vec{x})=-\frac{n}{2}\log(2\pi\sigma^2)-\frac{1}{\sigma^2}\sum_{i=1}^n (x_i-\mu)^2.$$ Facilmente possiamo calcolare la matrice di informazione di Fisher, che risulterà:
$$
{I(\vec{\theta})}=\left(
\begin{array}{cc}
\frac{n}{\sigma^2} & 0 \\
0 & \frac{n}{2\sigma^4}
\end{array}
\right)
\mbox{,  e quindi }
{\left(I(\vec{\theta})\right)^{-1}}=\left(
\begin{array}{cc}
\frac{\sigma^2}{n} & 0 \\
0 & \frac{2\sigma^4}{n}
\end{array}
\right)
$$
Osserviamo anche che $\var[\overline{X}_n]=\frac{\sigma^2}{n}$ e $\var[S_n^2]=\frac{2\sigma^4}{n-1}>\frac{2\sigma^4}{n}$, dove sono stati calcolati i valori dell'inversa della matrice di Fisher. Si deduce allora dal teorema di Rao-Cramér che la media campionaria è stimatore efficiente per $\mu$, mentre non lo è la varianza campionaria per $\sigma^2$, anche se lo è asintoticamente.
Coerentemente, gli stimatori di massima verosimiglianza si possono ottenere risolvendo il sistema composto dalle rispettive score-functions dei parametri,
$$
\left\{
\begin{array}{lr}
\frac{\partial}{\partial\mu}l(\mu,\sigma^2,\vec{X})=0 \\
\frac{\partial}{\partial\sigma^2}l(\mu,\sigma^2,\vec{X})=0
\end{array}
\right.
$$
che ci restituisce le due soluzioni $\hat{\mu}_n=\overline{X}_n$ e $\hat{\sigma^2}_n=\frac{n}{n-1}S_n^2$. Notiamo infine che i due valori non diagonali della matrice di Fisher sono nulli perché ci troviamo in distribuzione normale, ove la non correlazione implica anche l'indipendenza.
\\
\\
\textit{Osservazione:} Sia $Z\sim N(0,1)$ una variabile casuale. Allora vale la seguente relazione: $$\mu_{2s}\coloneqq \mathbb{E}[Z^{2s}]=\frac{(2s)!}{2^s s!}$$ Essa risulta comoda per il calcolo dei momenti delle variabili normali, tenendo conto che $X\coloneqq \mu+\sigma Z \Longrightarrow X\sim N(\mu,\sigma^2)$.
\\ 
\\
\textit{Esempio:} Sia $\vec{X}\coloneqq (X_1,\ldots,X_n)\sim f_X(x,\eta)\coloneqq \eta e^{-\eta(x-3)}\mathbbm{1}_{[3,+\infty]}$, con $\eta>0$. 
Vogliamo calcolare il limite inferiore di Rao-Cramér per uno stimatore non distorto di $g(\eta)\coloneqq \frac{1}{\eta}$, 
individuare possibilmente un siffatto stimatore e, dopo averlo trovato, calcolare se esso sia o meno efficiente. 
In base al teorema di Rao-Cramér si ha che 
$$I(g(\eta))=\left(g'(\eta)\right)^2\frac{1}{I(\eta)}=\frac{1}{\eta^4}\frac{1}{I(\eta)}.$$ 

Per calcolare $I(\eta)$, troviamo innanzi tutto la funzione di log-verosimiglianza. 
Si ha per prima cosa che: \\
$$L(\eta, \vec{x})=\eta^n e^{-\eta\sum_{i=1}^n (x_i-3)} \Rightarrow l(\eta,\vec{x})=n\log(\eta)-\eta\sum_{i=1}^n (x_i-3)$$ Possiamo calcolare adesso $$I(\eta)=-\mathbb{E}_\eta \Big[\frac{\partial^2}{\partial\eta^2} l(\eta,\vec{x})\Big]=-\mathbb{E}_\eta [-\frac{n}{\eta^2}]=\frac{n}{\eta^2}.$$ 
Ne segue subito che il limite inferiore cercato sarà allora
 $$I(g(\eta))=\frac{1}{\eta^4}\cdot \frac{\eta^2}{n}=\frac{1}{n\eta^2}.$$

Per trovare un possibile stimatore, ricordiamo la relazione già dimostrata durante la dimostrazione del teorema di Rao-Cramér $\mathbb{E}[S(\eta)]=0$. 
Ne segue che 
$0=\mathbb{E}_\eta\Big[\frac{\partial}{\partial\eta}l(\eta,\vec{X})\Big]=
\mathbb{E}_\eta[\frac{n}{\eta}-\sum_{i=1}^n (X_i-3)]=
n\mathbb{E}_\eta[\frac{1}{\eta} -(\overline{X}_n-3)]=
n\left(\frac{1}{\eta}-\mathbb{E}_\eta[\overline{X}_n-3]\right)$ da cui  $$T_n(\vec{X})\coloneqq (\overline{X}_n-3)$$ è lo stimatore cercato. Ora, è semplice calcolare che $$\var[T_n]=\var[\overline{X}_n]=\frac{1}{n^2}\sum_{i=1}^n \var[X_i]=\frac{1}{n}\var[X]=\frac{1}{n\eta^2}$$ Ne segue che lo stimatore cercato è effettivamente anche efficiente.

\chapter{Sufficienza}

\section{Introduzione}

\begin{dfn}[\textsc{mvue}]
  Sia \((X_1,\dotsc,X_n)\) un campione casuale da una distribuzione di 
  densità \(f_X(\mathbf{x};\theta)\), \(\theta \in \Theta\). Sia
  \(Y = u(X_1,\dotsc,X_n)\) uno stimatore del parametro \(\theta\). Si dice
  \(Y\) essere \emph{stimatore a minima varianza nella classe dei non distorti} per \(\theta\) se
  \begin{enumerate}
    \item \(Y\) è uno stimatore non distorto di \(\theta\), ovvero
          \(\mathbb{E}(Y) = \theta\) e
    \item per ogni stimatore non distorto \(W\) di \(\theta\),
          \(\var(Y) \le \var(W)\).
  \end{enumerate}
\end{dfn}

Rinfreschiamo il concetto di distribuzione condizionata:
\begin{dfn}[Distribuzione condizionata]
Siano \(X\), \(Y\) variabili casuali. Sia \(x \in \mathcal{S}_X\).
Per ogni punto \(y \in \mathcal{S}_Y\), definiamo:
\begin{description}
  \item[\textsc{caso discreto}] la \emph{pmf condizionata}
  \begin{equation}
    p_{Y\mid X}(y \mid x) = \frac{p_{X,Y}(x,y)}{p_X(x)}
  \end{equation}
  dove \(p_{X,Y}\) è la funzione di massa congiunta di \(X\),\(Y\) e \(p_X\) rappresenta la funzione di massa probabilistica marginale di \(X\).
  \item[\textsc{caso continuo}] la \emph{pdf condizionata}
  \begin{equation}
    f_{Y\mid X}(y \mid x) = \frac{f_{X,Y}(x,y)}{f_X(x)}
  \end{equation}
  dove \(f_{X,Y}\) è la densità congiunta delle due variabili e \(f_X\) è la densità marginale di \(X\).
\end{description}
\end{dfn}

\noindent \textbf{NB:} $f_{Y|X}(y | X=x) = f_Y(y)$ per ogni y se e solo se X e Y sono indipendenti.

\paragraph{Cosa rappresenta la sufficienza?}
La sufficienza di una statistica definisce formalmente la capacità di tale funzione di rappresentare in maniera \emph{sintetica} l'informazione contenuta nel campione, senza perdita di \emph{informazione rilevante}.
Intuitivamente si può pensare alla proprietà di conservazione dell'informazione rilevante in questo modo: qualsiasi altra statistica, calcolata a partire dallo stesso campione, non porta più informazioni rispetto a \(\theta\) di quante ne abbia già portato la statistica sufficiente.
Per questo, banalmente, l'intero campione è sicuramente una statistica sufficiente (molto poco sintetica, ma a volte non c'è di meglio).

\begin{dfn}[Sufficienza]
  Sia \((X_1,\dotsc,X_n)\) un campione casuale da una distribuzione avente funzione densità/massa probabilistica \(f_X(x;\theta)\). Sia inoltre \(T_n(X_1,\dotsc,X_n)\) uno stimatore di \(\theta\) avente funzione di densità o massa probabilistica \(f_{T_n}(t; \theta)\).
  La statistica \(T_n\) viene detta \emph{sufficiente} per \(\theta\) se la distribuzione di \(X_1,\dotsc,X_n\) condizionata a \(T_n(x_1,\dotsc,x_n) = t_n\)
  \begin{equation*}
    H(x_1,\dotsc,x_n) = \frac{\prod_{i=1}^n f_{X}(x_i;\theta)}{f_{T_n}(t_n;\theta)}
  \end{equation*}
  è una funzione non dipendente da \(\theta\).
\end{dfn}

Abbiamo quindi formalizzato le richieste fatte sopra: il fatto che la distribuzione condizionata (del vettore allo stimatore) non dipenda da \(\theta\) implica di fatto l'impossibilità di perdere informazioni rilevanti su \(\theta\) stesso, e cioè, lo stimatore \(T_n\) contiene tutte le informazioni necessarie riguardanti \(\theta\) per fare inferenza sul parametro incognito.
In altre parole, \textit{le informazioni contenute in \(T_n\)} riguardo a \(\theta\) \textit{sono le stesse contenute nell'intero campione}.

Vedremo che la sufficienza può essere verificata usando il teorema di fattorizzazione di Neyman, che di fatto fornisce una caratterizzazione 'più comoda' da verificare (rispetto all'uso brutale della definizione) per assicurarsi che una data statistica abbia la proprietà desiderata.

\subsection{Esempi} 


\begin{ese}
  Sia \(X = (X_1,\dotsc,X_n) \sim b(1,p)\).
  Allora, \(T_n(X)\coloneqq \sum_{i=1}^n X_i\) è una statistica sufficiente per \(p\). Infatti, considerando la massa condizionata di \(X\) rispetto a \(T_n\), si verifica che
  \begin{equation*}
    f_{X|T_n}(\mathbf{x},t_n) = \frac{f_{X,T_n}(\mathbf{x},t_n)}{f_{T_n}(t_n)} =
    \frac{p^{t_n}(1-p)^{n-t_n}}{\binom{n}{t_n}p^{t_n}(1-p)^{n-t_n}} =
    \frac{1}{\binom{n}{t_n}}.
  \end{equation*}
  Notiamo che \(\binom{n}{t_n}^{-1}\) è indipendente da \(p\), per cui \(T_n\) è statistica sufficiente per \(p\).
\end{ese}

\begin{ese}
  Sia \(X = (X_1,\dotsc,X_n) \sim \mathcal{G}(\alpha=2,\beta)\). Allora, \(T_n(X)\coloneqq \sum_{i=1}^n X_i\) è una statistica sufficiente per \(\beta\).
  Infatti, si ha innanzi tutto che \(T_n(X)\sim \mathcal{G}(2n,\beta)\) grazie all'indipendenza delle osservazioni e alla proprietà di riproducibilità di \(G\). Allora
  \begin{equation*}\begin{split}
    f_{X|T_n}(\mathbf{x},t_n) &= \frac{f_{X,T_n}(\mathbf{x},t_n)}{f_{T_n}(t_n)} = \frac{f_{X}(\mathbf{x})}{f_{T_n}(t_n)} \\
    &= \frac{\prod_{i=1}^n \frac{1}{\Gamma(2)\beta^2}x_i^{2-1}e^{-\frac{1}{\beta}x_i}}{\frac{1}{\Gamma(2n)\beta^{2n}}t_n^{2n-1}e^{-\frac{1}{\beta}t_n}} \\
    &=\frac{\Gamma(2n)\prod_{i=1}^n x_i}{{(2!)}^n t_n^{2n-1}},
  \end{split}
\end{equation*}
  da cui si evince la sufficienza di \(T_n\).
\end{ese}

\begin{ese}
Sia $\vec{X}\coloneqq (X_1,\ldots,X_n)$ un campione casuale avente funzione di densità $f_X(x,\theta)\coloneqq e^{-(x-\theta)}\mathbbm{1}_{(\theta, +\infty)}(x)$, $\theta>0$. 
Allora la statistica d'ordine $X_{(1)}$ è sufficiente per \(\theta\). Infatti: 
$$f_{X_{(1)}}(x,\theta)=ne^{-n(x-\theta)}$$ ed otteniamo subito che $$f_{\vec{X} | X_{(1)}}(\vec{x},x_{(1)},\theta)=\frac{f_{\vec{X}}(\vec{x},\theta)}{f_{X_{(1)}}(x_{(1)})}=\frac{\prod_{i=1}^n e^{-(x_i-\theta)}\mathbbm{1}_{(\theta,+\infty)}(x_i)}{ne^{-n(x_{(1)}-\theta)}\mathbbm{1}_{(\theta,+\infty)}(x_{(1)})}=\frac{e^{-\sum_{i=1}^n x_i}}{ne^{-nx_{(1)}}}$$ che non dipende dal parametro \(\theta\).
\end{ese}

\begin{thm}[Fattorizzazione di Neyman] Sia \(X = (X_1,\dotsc,X_n)\) un campione casuale da una popolazione di densità/massa \(f_X(x,\theta)\).
La statistica \(T_n=T_n(X_1,\dotsc,X_n)\) è statistica sufficiente per \(\theta\) se e solo se esistono due funzioni non negative \(g\),\(h\) con le seguenti richieste:
\begin{enumerate}
\item \(h=h(\mathbf{x})\) sia indipendente da \(\theta{}\) e
\item \(g=g(\theta, t_n)\) dipenda da \(\theta\) e dalla determinazione del campione \(\mathbf{x}\) solo attraverso \(t_n\),
\end{enumerate}
e tali che, per ogni \(\mathbf{x}_0 \in \mathfrak{X}\), \(\theta_0 \in \Theta\),
la funzione di verosimiglianza possa essere scritta come
\begin{equation}
  L(\theta_0,\mathbf{x}_0)=h(\mathbf{x}_0)\cdot g(\theta_0, t_n(\mathbf{x}_0)).
\end{equation}
\end{thm}

\begin{proof}
  Assumiamo che \(T_n\) sia statistica sufficiente per \(\theta{}\).
  Per definizione di densità condizionata e indipendenza delle osservazioni di un campione,
  \begin{equation*}
    f_{X \mid T_n}(\mathbf{x},t_n,\theta) = 
    \frac{\prod_{i=1}^n f_X(x_i;\theta)}{f_{T_n}(t_n,\theta)}
    \implies
    L(\theta; \mathbf{x}) = f_{X \mid T_n}(\mathbf{x},t_n,\theta)
    \cdot f_{T_n}(t_n,\theta);
  \end{equation*}
  inoltre, dalla sufficienza di \(T_n\) si ricava l'indipendenza della densità condizionata rispetto a \(\theta{}\). Dunque, ponendo
  \begin{align*}
    h(\mathbf{x}) &:= f_{X \mid T_n}(\mathbf{x},t_n) \\
    g(\theta,t_n) &:= f_{T_n}(t_n,\theta)
  \end{align*}
  si ottiene la fattorizzazione \(L(\theta;\mathbf{x}) =
  h(\mathbf{x})g(\theta,t_n)\).

  Assumiamo ora che, con le condizioni poste dal teorema, che sia possibile fattorizzare \(L(\theta;\mathbf{x})\) nel modo sopra descritto. Dimostriamo il caso in cui il campione sia composto da variabili casuali discrete.
  Definiamo l'insieme di livello
  \begin{align*}
    A_n :&= \lbrace \mathbf{x} \in \mathcal{X} \colon T_n(\mathbf{x})
         = t_n \rbrace.
    \intertext{e osserviamo che}
    f_{T_n}(t_n) &= \sum_{\mathbf{x} \in A_n} L(\theta,\mathbf{x}) =
    \sum_{\mathbf{x}\in A_n}h(\mathbf{x})g(\theta,t_n) =
    g(\theta,t_n)\sum_{\mathbf{x}\in A_n}h(\mathbf{x}).
  \end{align*}
  Ricorrendo alla definizione di densità condizionata, si ottiene la relazione
  \begin{equation*}
    f_{X \mid T_n}(\mathbf{x},t_n) =
    \frac{L(\theta,\mathbf{x})}{f_{T_n}(t_n)} =
    \frac{h(\mathbf{x})g(\theta,t_n)}{g(\theta,t_n)\sum_{\mathbf{x}\in A_n}h(\mathbf{x})} =
    \frac{h(\mathbf{x})}{\sum_{\mathbf{x}\in A_n}h(\mathbf{x})}.
  \end{equation*}
  Non essendoci dipendenza dal parametro \(\theta{}\), possiamo concludere che \(T_n\) è statistica sufficiente per esso.
  La dimostrazione per il caso continuo è analoga.
\end{proof}


\newpage

\vspace*{\fill}
\noindent \large Attenzione: da questo punto in poi, le dispense sono state scritte durante la sessione d'esame. Per questo motivo, gli autori non hanno potuto dedicarvi il tempo e l'attenzione che un lavoro di questo tipo richiede. Crediamo che possano essere comunque più precise e complete degli appunti presi in classe, ma non garantiamo nulla. Appena possibile le dispense verranno ricontrollate e rese definitive.
\vspace*{\fill}
\newpage


%%%%lezione 10 maggio%%%%

Lezione del 10/05, ultima modifica 10/06, Andrea Gadotti\\
\\
Il teorema di fattorizzazione costituisce un criterio per l'individuazione, se esiste, di una statistica sufficiente.
\\
\\
\textbf{Esempio} Sia $(X_1,...,X_n)$ da $N(\mu,\sigma^2)$.
\begin{enumerate}
\item [(a)] se $\mu$ è noto, allora cerco una statistica sufficiente per $\sigma^2$.\\
$T_n (X_1,...,X_n) = \displaystyle\sum_{i=1}^n (X_i-\mu)^2$
\item [(b)] Se $\sigma^2$ è noto, allora cerco una statistica sufficiente per $\mu$.
\item [(c)] Se $\mu$ e $\sigma^2$ non sono noti, allora cerco una statistica congiuntamente sufficiente per $\mu$ e $\sigma^2$.\\
$S_n (X_1,...,X_n) = \left( \sum_{i=1}^n X_i, \sum_{i=1}^n (X_i - \overline{X}_n^2) \right)$.
\end{enumerate}

Risulta quindi evidente che il concetto di statistica sufficiente è \emph{problem dependent}.
\\
\begin{esempio} Sia $(X_1,...,X_n)$ da $U(\theta, \theta +1)$ con $\theta \in \mathbb{R}$. Vogliamo trovare una statistica sufficiente per $\theta$.\\
Calcoliamo innanzitutto la funzione di verosimiglianza:
\begin{eqnarray}
L(\theta;\underline{x}) &=& \displaystyle\prod_{i=1}^n \mathbbm{1}_{[\theta; \theta +1]}(x_i)\\
&=& \displaystyle\prod_{i=1}^n \mathbbm{1}_{\mathbb{R}}(x_i) \mathbbm{1}_{[\theta; \theta +1]}(x_{(n)}) \mathbbm{1}_{[\theta; \theta +1]}(x_{(1)})\\
&=& \displaystyle\prod_{i=1}^n \mathbbm{1}_{\mathbb{R}}(x_i) \mathbbm{1}_{[X_{(n)}-1; X_{(n)}]}(\theta)
\end{eqnarray}
Notiamo che l'ultima espressione è il prodotto di due funzioni dove la prima dipende solo dal campione, mentre la seconda dipende sia dal parametro $\theta$ che dal campione. Quindi grazie al teorema di fattorizzazione di Neyman abbiamo che $(X_{(1)},X_{(n)})$ è statistica sufficiente per $U$.\\
Nota: poiché $U_{(\theta, \theta +1)}$ non appartiene alla famiglia regolare non è necessariamente vero che \emph{dimensione statistica sufficiente = dimensione vettore parametri}. Infatti in questo caso abbiamo $2 \neq 1$.
\end{esempio}

\begin{oss} Spesso vediamo il campione casuale come diverso da una statistica, che è funzione del campione. Ma esso porta in sè tutta l'informazione disponibile relativamente al vettore parametrico. Anche il campione stesso è una statistica, in particolare è una statistica sufficiente, l'unico problema è che non è per nulla "sintetico".
\end{oss}

\textbf{Problema:} esiste una statistica sufficiente che sia migliore (ovvero più sintetica) delle altre, a parità di informazione contenuta circa $\theta$? In effetti una tale statistica esiste e viene detta \textit{statistica sufficiente minimale}.

\section{Sufficienza minimale}

\textbf{Esempio introduttivo} Pensiamo di replicare $n=4$ volte una prova bernoulliana con probabilità di successo $p$. Consideriamo lo spazio campionario $\varkappa$, composto da $2^4=16$ elementi. Vogliamo fare inferenza su $p$. Definiamo di seguito tre statistiche differenti:

\begin{enumerate}
\item [1)] $Y_1 := $ risultato della prima prova
\item [2)] $Y_2 := $ numero di successi nelle quattro prove
\item [3)] $Y_3 := (Y_1,Y_2)$
\end{enumerate}

A questo punto notiamo che:

\begin{enumerate}
\item [1)] $Y_1$ non è una statistica sufficiente per $p$ (il fatto che ci sia stato un successo o meno nella prima prova non ci dà alcuna informazione rispetto a $p$)
\item [2)] $Y_2$ è statistica sufficiente per $p$ (cfr. Teorema di fattorizzazione) e consiste in un unico valore, ovvero la somma degli elementi del campione
\item [3)] $Y_3$ consiste in due valori, ovvero è un vettore di dimensione 2. È statistica sufficiente per $p$, ma è eccessivamente raffinata per il problema che vogliamo affrontare, poiché l'apporto di $Y_1$ è inutile. Infatti $Y_3$ non è statistica sufficiente minimale per $p$.
\end{enumerate}

\begin{dfn}[stat. suff. minimale]
  Una statistica \(T_n(\mathbf{X})\) è detta \emph{sufficiente minimale} se è
  sufficiente ed è funzione di ogni altra statistica sufficiente, ossia
  \(T(\mathbf{X} = g(Q(\mathbf{X}))\) dove \(Q(\mathbf{X})\) è statistica
  sufficiente.
\end{dfn}

\begin{thm}
  Sia \((X_1,\dotsc,X_n)\) un campione casuale di cui \(\mathbf{x}\),
  \(\mathbf{y}\) sono due distinte determinazioni. Se la statistica
  \(T(\mathbf{X})\) gode della proprietà che il rapporto di verosimiglianze
  \begin{equation*}
    \frac{L(\theta;\mathbf{x})}{L(\theta;\mathbf{y})}
  \end{equation*}
  non dipenda da \(\theta\) se e solo se \(T(\mathbf{x})=T(\mathbf{y})\),
  allora \(T(\mathbf{X})\) è statistica sufficiente minimale per \(\theta\).
\end{thm}

\begin{esempio} Sia $(X_1,...,X_n)$ da $X \sim \Gamma(\alpha,\beta), \alpha, \beta >0$. Vogliamo trovare una statistica congiuntamente sufficiente e minimale per $\alpha$ e $\beta$:
\begin{enumerate}
\item [1)] Cerchiamo una statistica sufficiente:
$$L(\alpha,\beta;\underline{x}) = \left( \left( \Gamma(\alpha)\beta^{\alpha}\right)^{-n} \left( e^{-\frac{1}{\beta} \sum_{i=1}^n x_i} \right) \left( \prod_{i=1}^n x_i \right) \right) \cdot \left( \prod_{i=1}^n \mathbbm{1}_{\mathbb{R_+}}(x_i) \right)$$
Osserviamo che abbiamo scritto $L(\alpha,\beta;\underline{x})$ come $g \left( \alpha,\beta; \sum_{i=1}^n x_i, \prod_{i=1}^n x_i \right) \cdot h(\underline{x})$.\\
Quindi per il teorema di fattorizzazione abbiamo che $S_n:=\left( \sum_{i=1}^n x_i, \prod_{i=1}^n x_i \right)$ è statistica congiuntamente sufficiente per $\alpha$ e $\beta$.
\item [2)] Verifichiamo ora che tale statistica è anche minimale. Supponiamo che $S_n(\underline{x})=S_n(\underline{y})$, ovvero $\left( \sum_{i=1}^n x_i, \prod_{i=1}^n x_i \right) = \left( \sum_{i=1}^n y_i, \prod_{i=1}^n y_i \right)$. Allora:
$$\frac{L(\alpha,\beta;\underline{x})}{L(\alpha,\beta;\underline{y})} = e^{\frac{1}{\beta} \left( \sum_i x_i - \sum_i y_i \right)} \left( \prod_i \frac{x_i}{y_i} \right) = e^{0} \left( \prod_i (1) \right)^{\alpha} = 1$$
Per il teorema di Lehmann-Scheffé concludiamo che $S_n$ è statistica sufficiente minimale.\\
Nota: in realtà noi abbiamo verificato che vale solo una delle due implicazioni richieste dalle ipotesi del teorema (quella da destra a sinistra). Questo è ciò che è stato fatto a lezione, e per il momento scriviamo solo questo.
\end{enumerate}
\end{esempio}
\section{Principio di verosimiglianza}

La verosimiglianza combina due tipi di informazione:
\begin{itemize}
\item informazione pre-sperimentale espressa attraverso il modello statistico scelto per descrivere il fenomeno di indagine
$$\mathfrak{F}_\theta=\left\{f(x;\theta):\ \theta\in\Theta\subseteq\mathbb{R}^k\right\}$$
\item informazione sperimentale contenuta in quella che è la determinazione campionaria $\underline{x}=(x_1,\ldots, x_n)$
$$L(\theta;\underline{x})=\prod_{i=1}^n f_{X_i}(x_i;\theta)$$
Nota: In generale $L(\theta;\underline{x})=f_{X_1,\ldots, X_n}(x_1,\ldots, x_n;\theta)$, densità congiunta, ovvero è definita anche nel caso in cui le variabili casuali non siano indipendenti ed equidistribuite. Quando invece lo sono, possiamo esprimere $L(\theta; \underline{x})$ nella solita forma, ovvero come produttoria.
\end{itemize}

\paragraph{Principio di verosimiglianza.}
  Con riferimento ad un dato modello statistico avente spazio campionario
  \(\mathcal{X}\) e famiglia di distribuzioni
  \begin{equation*}
    \mathfrak{F}_\theta=
    \left\lbrace f(x;\theta) \colon
    \theta\in\Theta\subseteq\mathbb{R}^k\right\rbrace,
  \end{equation*}
  l'evidenza circa i parametri del modello è contenuta totalmente nella
  funzione di verosimiglianza: due determinazioni \(\mathbf{x}\),
  \(\mathbf{y}\) per cui \(L(\theta;\mathbf{x})=L(\theta;\mathbf{y})\)
  devono condurre alle medesime conclusioni inferenziali circa \(\theta\).

\begin{thm}
L'informazione di Fisher fornita da uno stimatore basato su una statistica sufficiente \(W_n\) coincide con l'informazione fornita dall'intero campione
\end{thm}
\begin{proof}
  Grazie alla sufficienza di \(W_n\), dal teorema di fattorizzazione di Neyman, si ottiene l'identità
  \begin{equation*}
    \frac{d}{d\theta}l(\theta;\mathbf{x}) =
    \frac{d}{d\theta}\ln h(\mathbf{x}) + \frac{d}{d\theta} \ln g(\theta,w_n)
    = \frac{d}{d\theta} \ln g(\theta,w_n).
  \end{equation*}
  Ciò significa che
  \begin{equation*}
    L(\theta;\mathbf{x}) = L(\theta;\mathbf{y}) \implies
    \frac{d}{d\theta} \ln g(\theta,w_n(\mathbf{x})) =
    \frac{d}{d\theta} \ln g(\theta,w_n(\mathbf{y}));
  \end{equation*}
  essendo in questo caso l'informazione di Fisher pari a
  \begin{equation*}
    \mathbb{E}\left(\left(\frac{d}{d\theta}\ln g(\theta;t_n)\right)^2%
    \right)
  \end{equation*}
  deduciamo che essa coincide con l'informazione totale fornita dal campione.%
  ~\footnote{In parole semplici, l'informazione di Fisher data dalle due
  determinazioni sarà la stessa, e ciò rispetta il principio di verosimiglianza.}
\end{proof}

\begin{esempio} Sia $(X_1,\ldots,X_n)$ da $N(\mu,\sigma^2=1)$, vogliamo individuare una statistica sufficiente minimale per $\mu$ e la relativa informazione di Fisher.
\begin{enumerate}
\item[1)] Notiamo che 
$$W_n:=\sum_{i=1}^n X_i \sim N(n\mu,n)$$
è statistica sufficiente (si verifica subito usando il teorema di fattorizzazione).\\
Vogliamo mostrare che l'informazione di Fisher che si ottiene usando solo $W_n$ è la stessa che si trova usando l'intero campione.
\item[2)] Calcoliamo $I(\mu)$ supponendo di avere in mano, anziché l'intero campione $(x_1,...,x_n)$, il solo valore $w_n=\sum_{i=1}^n x_i$ (di cui conosciamo la distribuzione, come visto al punto 1). Iniziamo calcolando la funzione di verosimiglianza di $w_n$ (che coincide con la sua densità, essendo un unico valore e non un vettore):
$$L(\mu; w_n)=\frac{1}{\sqrt{2 \pi n}} e^{-\frac{1}{2n} (w_n-n\mu)^2}$$
A questo punto, calcolando la funzione di log-verosimiglianza, con un po' di conti si trova facilmente che $I_{W_n}(\mu)=n$.
\item[3)] Calcoliamo $I(\mu)$ supponendo di avere in mano l'intero campione. Iniziamo trovando la funzione di verosimiglianza di $(x_1,...,x_n)$:
$$L(\mu; \underline{x})= \prod_{i=1}^n \left( \frac{1}{\sqrt{2 \pi}} e^{-\frac{1}{2}(x_i-\mu)^2} \right)$$
Come prima, con un po' di conti si ottiene $I_{\underline{x}}(\mu)=n$, ovvero le due informazioni di Fisher coincidono.
\\
\\
\end{enumerate}
\end{esempio}
%%%%lezione 13 maggio%%%%

Lezione del 13/05, ultima modifica 16/06, Andrea Gadotti

\section{Famiglie esponenziali}

Notiamo innanzitutto che una qualsiasi funzione di densità per una distribuzione appartenente alla famiglia esponenziale può essere scritta come
$$f_X(x;\theta) = \exp \left\lbrace C(x) + D(\theta) \displaystyle \sum_{m=1}^k A_m(\theta) B_m(x) \right\rbrace$$

\begin{teo}

Sia $(X_1,...,X_n)$ un campione casuale appartenente alla famiglia esponenziale a k parametri. Allora il vettore $T_n:=(\sum_{i=1}^n B_1(x_i),...,\sum_{i=1}^n B_k(x_i))$ è statistica sufficiente per $\theta$.

\begin{proof}
Da inserire (comunque è facile, basta usare il teorema di fattorizzazione).
\end{proof}

\end{teo}

\begin{esempio}
Sia $(X_1,...X_n)$ da $N(\mu, \sigma^2)$ con $\mu$ e $\sigma^2$ non noti (dunque $\vec{X}$ proviene da una famiglia esponenziale a 2 parametri). Per il teorema appena visto si ha che $I_n:= (\sum_{i=1}^n B_1(x_i), \sum_{i=1}^n B_2(x_i)) = (\sum_{i=1}^n x_i, \sum_{i=1}^n x_i^2$ è statistica sufficiente per $(\mu, \sigma^2)$. (Nota: la seconda uguaglianza si verifica immediatamente).\\
Affermiamo (senza dimostrazione) che lo stimatore di massima verosimiglianza per $(\mu, \sigma^2)$ è $\hat{\theta} = \left( \frac{1}{n} \sum_{i=1}^n X_i, \frac{1}{n} \sum_{i=1}^n (X_i - \overline{X}_n)^2 \right)$ e notiamo che esso è funzione della statistica sufficiente $I_n$.\\
\\
\end{esempio}
\input{0517.tex}
\input{0520.tex}
\input{0524.tex}

Lezione del 27/05, ultima modifica 03/06, Michele Nardin

\chapter{}

\((X_1,\dots,X_n)\) campione casuale proveniente da una distribuzione
\(F_X(x;\theta)\), con \(\theta \in \Theta\). Vogliamo verificare un sistema di ipotesi parametriche
\begin{equation*}
  \begin{cases}
    H_0 \colon \theta = \theta_0 \\ H_1 \colon \theta = \theta_1
  \end{cases}
\end{equation*}
Il sistema di ipotesi è detto \emph{semplice} se esso dipende solamente dal valore di \(\theta\), cioè se le ipotesi determinano completamente la distribuzione della statistica.

Abbiamo visto che la funziona di massima verosimiglianza fornisce un "ordinamento" tra i valori assunti da \(\theta\), ossia una misura della preferenza di un valore rispetto ad un altro.
\begin{equation*}
  L(\mathbf{x};\theta) = \prod_{i=1}^n f_X(x_i;\theta)\mathbb{I}_S(x_i)
\end{equation*}
In particolare, il rapporto
\begin{equation*}
  \frac{L(\theta_1;\mathbf{x})}{L(\theta_0;\mathbf{x})}
\end{equation*}
che costituisce una statistica del rapporto di verosimiglianza, può fornire importanti informazioni sulla regione critica del test.


\section{Teoria dei Test più Potenti ed Uniformemente più Potenti}
Useremo la notazione: MP = most powerful e UMP = uniformly most powerful.\\
\\
\noindent \textbf{Test più potenti per verifica d'ipotesi semplici}\\
\\
Per ora ci concentriamo sui test semplici, ossia i test in cui le ipotesi determinano completamente la distribuzione della statistica sotto la data ipotesi.

Ricordiamo che per la verifica d'ipotesi riguardo ad un parametro $\vartheta$ della forma

$$\bigg \{
\begin{array}{rl}
H_0: & \vartheta = \vartheta_0 \\
H_1: & \vartheta = \vartheta_1 \\
\end{array}
$$

un dato test fornisce una regione di rifiuto $C$, (nota: quando ci riferiamo ad un test implicitamente ci riferiamo alla sua regione di rifiuto!) le cui probabilità d'errore sono

$$\bigg \{
\begin{array}{rl}
\alpha =&  P(\text{rifiuto } H_0 | H_0) \\
\beta =& P(\text{non rifiuto } H_0 | H_1) \\
\end{array}
$$

Fissato $\alpha$, ricordiamo che $1 - \beta$ è detto potenza del test. Possiamo sempre pensare $\beta$ come funzione di $\vartheta_1$:
$$\beta(\vartheta_1) = P(\text{non rifiuto } H_0 | \vartheta = \vartheta_1)$$

\begin{definizione}[test $MP_\alpha$]
 Fissato $\alpha \in (0,1)$, un test che minimizza $\beta(\vartheta_1)$ è detto test più potente di livello $\alpha$ ($MP_\alpha$) \small{(per la verifica d'ipotesi semplice contro alternativa semplice)}. 

\end{definizione}

Il nostro obiettivo sarà ovviamente quello di trovare tale test. Vediamo come i concetti visti fin'ora (soprattutto verosimiglianza e sufficienza) siano legati a quanto cerchiamo tramite il seguente

\begin{lem}[Neyman - Pearson]
Sia \((X_1,\dots,X_n)\) un campione casuale proveniente da una distribuzione di densità \(f_{X_i}(\mathbf{x}; \theta)\).
Sia inoltre

\begin{equation*}
  \begin{cases}
    H_0 \colon \theta = \theta_0 \\ H_1 \colon \theta = \theta_1
  \end{cases}
\end{equation*}

il sistema di ipotesi (entrambe semplici) da verificare. Indicata con \(L(\theta;\mathbf{x})\) la funzione di massima verosimiglianza, 
il test $MP_\alpha$ per la verifica di \(H_0\) ha regione critica (o di rifiuto) data da
\begin{equation*}
  C = \left\lbrace \mathbf{x} \in X \colon
      \frac{L(\theta_1;\mathbf{x})}{L(\theta_0;\mathbf{x})} > A
      \right\rbrace, \quad A \in \mathbb{R}_+
\end{equation*}
dove \(A\) è un valore costante, dipendente da \(\alpha{}\).
\end{lem}
\textbf{Nota:} tale risultato è abbastanza intuitivo: se $A$ è sufficientemente grande, 
quando si vede che $L(\vartheta_1;\vec{x}) \geq A \cdot L(\vartheta_0;\vec{x})$ 
è assurdo pensare che $\vartheta_0$ sia una scelta plausibile, in quanto lo è molto di più $\vartheta_1$!

\begin{proof}

Sia $C$ la regione critica del test (di livello $\alpha$) dato dal lemma, e sia $C^*$ la regione critica di un qualsiasi altro test (sempre di livello $\alpha$).

Per provare il lemma dobbiamo mostrare quindi che $\beta^* \geq \beta$ 
(che sono ovviamente le probabilità di errore del 2o tipo delle due regioni in considerazione)
(indicheremo nel seguito con $\overline{C}, \overline{C^*}$ i complementari delle regioni $C$ e $C^*$).
Notiamo subito che
$$\overline{C} = \overline{C} \cap ({C^*} \cup \overline{C^*}) = (\overline{C} \cap {C^*}) \cup (\overline{C} \cap \overline{C^*})$$
e che 
$$\overline{C}^* = \overline{C}^* \cap ({C} \cup \overline{C}) = (\overline{C}^* \cap {C}) \cup (\overline{C}^* \cap \overline{C})$$

\begin{align*}
\beta^* - \beta 
&= P(\vec{x} \in \overline{C}^* | \vartheta = \vartheta_1) 
	- P(\vec{x} \in \overline{C} | \vartheta = \vartheta_1)
\\ &= \int_{\vec{x} \in \overline{C}^* } L(\vartheta_1;\vec{x}) d \vec{x} 
	- \int_{\vec{x} \in \overline{C} } L(\vartheta_1;\vec{x}) d \vec{x}
\\ &= \int_{\vec{x} \in (\overline{C}^* \cap {C}) \cup (\overline{C}^* \cap \overline{C}) } L(\vartheta_1;\vec{x}) d \vec{x} 
	- \int_{\vec{x} \in (\overline{C} \cap {C^*}) \cup (\overline{C} \cap \overline{C^*}) } L(\vartheta_1;\vec{x}) d \vec{x}
\\ &= \int_{\vec{x} \in \overline{C}^* \cap C } L(\vartheta_1;\vec{x}) d \vec{x} 
	- \int_{\vec{x} \in \overline{C} \cap C^* } L(\vartheta_1;\vec{x}) d \vec{x}
\\ & \geq A \left[  \int_{\vec{x} \in \overline{C}^* \cap C } L(\vartheta_0;\vec{x}) d \vec{x} 
	- \int_{\vec{x} \in \overline{C} \cap C^* } L(\vartheta_0;\vec{x}) d \vec{x} \right]
\\ & = A \left[ \int_{\vec{x} \in (\overline{C}^* \cap {C}) \cup ({C}^* \cap {C}) } L(\vartheta_0;\vec{x}) d \vec{x} 
	- \int_{\vec{x} \in (\overline{C} \cap {C^*}) \cup ({C} \cap {C^*}) } L(\vartheta_0;\vec{x}) d \vec{x} \right]
\\ & = A \left[  \int_{\vec{x} \in C} L(\vartheta_0;\vec{x}) d \vec{x} 
	- \int_{\vec{x} \in C^* } L(\vartheta_0;\vec{x}) d \vec{x} \right]
\\ & = A ( \alpha - \alpha) = 0
\end{align*}

Da cui $\beta^* \geq \beta $
\end{proof}

\noindent \textbf{Importante:} Da nessuna parte abbiamo supposto che $\vartheta$ sia uno scalare, e guardando la dimostrazione notiamo che tale ipotesi sarebbe inutile: quindi (come al solito d'altronde) $\vartheta$ può benissimo essere un vettore di parametri incogniti!\\
\\
\noindent \textbf{Test uniformemente più potenti per verifica d'ipotesi semplici contro alternative composte}\\
\\
Vediamo come estendere i risultati della sezione precedente quando le ipotesi alternative sono un pochino più complesse:

A noi interessano soprattutto le ipotesi unilaterali o bilaterali, ossia del tipo

$$\bigg \{
\begin{array}{rl}
H_0: & \vartheta = \vartheta_0 \\
H_1: & \vartheta > (<) \vartheta_0 \text{ (oppure } \vartheta = \vartheta_1,\ \vartheta_1 > \vartheta_0 (\vartheta_1 < \vartheta_0) )\\
\end{array}
$$

$$\bigg \{
\begin{array}{rl}
H_0: & \vartheta = \vartheta_0 \\
H_1: & \vartheta \neq \vartheta_0 \text{ (oppure } \vartheta = \vartheta_1,\ \vartheta_1 > \vartheta_0 \vee \vartheta_1 < \vartheta_0 )\\
\end{array}
$$

\begin{definizione}[$UMP_\alpha$]
Un test per la verifica d'ipotesi semplice ($H_0$) contro alternativa composta ($H_1$) viene detto uniformemente più potente di livello $\alpha$ ($UMP_\alpha$) se esso è $MP_\alpha$ per ogni possibile ipotesi semplice contenuta in $H_1$ (cioè minimizza $\beta(\vartheta_1)$ per ogni possibile fissato $\vartheta_1$ contemplato in $H_1$).
\end{definizione}

Ovviamente nessuno ci assicura che tale test esista: vediamo due esempi, entrambi basati su campione casuale da normale, uno di esistenza (in cui vediamo anche come applicare il lemma) e uno di non esistenza.\\
\\
\textbf{Esempio:}[\textit{esistenza}] Sia $(X_1,...,X_n)$ da $N(\mu,\sigma^2)$ (varianza nota).
Vogliamo trovare un test $UMP_\alpha$ per le seguenti ipotesi:

$$\bigg \{
\begin{array}{rl}
H_0: & \mu = \mu_0 \\
H_1: & \mu > \mu_0 \\
\end{array}
$$

Per farlo, considero le ipotesi

$$\bigg \{
\begin{array}{rl}
H_0: & \mu = \mu_0 \\
H_1: & \mu = \mu_1 \\
\end{array}
$$

per $\mu_1 > \mu_0$ fissato e vado a costruire il test $MP_\alpha$ di Neyman-Pearson associato.
Innanzitutto troviamo che
\begin{align*}
\frac{L(\mu_1;\vec{x})}{L(\mu_0;\vec{x})} 
&= exp\{ -\frac{1}{2\sigma^2} \sum_{i=1}^n (x_i - \mu_1)^2 - (x_i - \mu_0)^2 \}
\\&= exp\{ -\frac{1}{2\sigma^2} \sum_{i=1}^n 2x_i(\mu_0 - \mu_1) +\mu_1^2 - \mu_0^2 \}
\end{align*}

Ci prepariamo ad applicare il lemma (prendiamo il logaritmo per semplificare i conti): notiamo che

$$ -\frac{1}{2\sigma^2} \sum_{i=1}^n 2x_i(\mu_0 - \mu_1) +\mu_1^2 - \mu_0^2 \geq ln(A)$$
se e solo se
$$\frac{\sum_{i=1}^n x_i}{n} \geq \frac{\sigma^2 ln(A)}{n(\mu_1 - \mu_0)} + \frac{\mu_1 + \mu_0}{2} (=:B)$$

Per rendere operativa la regola di decisione (ossia rifiuto $H_0$ se $\overline{X_n} \geq B$)
occorre specificare il valore di B. 
Sotto $H_0$, $$\overline{X_n} \sim N(\mu_0, \frac{\sigma^2}{n})$$
da cui, fissato un livello di confidenza $\alpha$, devo risolvere la seguente equazione in B:

$$P(\overline{X_n} \geq B | \mu=\mu_0) = \alpha$$

equivalente a

$$P(\frac{\overline{X_n} - \mu_0} {\sigma/\sqrt{n}} \geq \frac{B - \mu_0} {\sigma/\sqrt{n}} | \mu=\mu_0) = \alpha$$

Ma sotto $H_0$, $\frac{\overline{X}_n - \mu_0} {\sigma/\sqrt{n}} \sim N(0,1)$, quindi  $\frac{B - \mu_0} {\sigma/\sqrt{n}} = z_\alpha$ (restituito dalle tavole).

Definitivamente, la regione critica sarà

$$C=\{ \vec{x} \in X : \overline{x_n} \geq \mu_0 + z_\alpha \frac{\sigma}{\sqrt{n}} \}$$ 

Abbiamo quindi ritrovato il test che abbiamo usato nel capitolo precedente, senza passare per il concetto di statistica pivot.

Rimane da mostrare che questo test è $UMP_\alpha$. Ma ciò è già stato fatto, infatti il ragionamento precedente vale quale che sia $\mu_1 (> \mu_0)$ fissato!\\
\\
\textbf{Esempio:}[\textit{non esistenza}] Sia $(X_1,...,X_n)$ da $N(\mu,\sigma^2)$ (varianza nota).
Vogliamo trovare un test $UMP_\alpha$ per le seguenti ipotesi:

$$\bigg \{
\begin{array}{rl}
H_0: & \mu = \mu_0 \\
H_1: & \mu \neq \mu_0 \\
\end{array}
$$

Seguendo la falsariga di quanto visto nell'esempio precedente, possiamo ovviamente dividere l'ipotesi $H_1$ nei due seguenti sottocasi:

$$H_1: \mu = \mu_1,\ \mu_1 > \mu_0 \vee \mu_1 < \mu_0 $$

Per il primo caso ($ \forall \mu_1 > \mu_0$), come visto sopra, abbiamo la regione critica
$$C=\{ \vec{x} \in X : \overline{x_n} \geq \mu_0 + z_\alpha \frac{\sigma}{\sqrt{n}} \}$$ 
mentre per il secondo caso ($\forall \mu_1 < \mu_0$), in maniera del tutto analoga, avremo
$$C=\{ \vec{x} \in X : \overline{x_n} \leq \mu_0 - z_\alpha \frac{\sigma}{\sqrt{n}} \}$$  
Notiamo quindi che le due regioni critiche non coincidono su ogni ipotesi semplice contenuta in $H_1$, e quindi (per definizione) non esiste il test $UMP_\alpha$ per verifica d'ipotesi bilaterali.\\
\\
Ci poniamo ora il problema di vedere quando esistono i test $UMP_\alpha$.\\
\\
\noindent \textbf{Osservazione:}[\textit{Legame tra statistiche sufficienti e test di Neyman-Pearson}]
Supponiamo di avere un campione casuale da una distribuzione con funzione di densità $f(x;\vartheta)$.
Supponiamo inoltre che $T_n$ sia una statistica sufficiente per $\vartheta$.
In accordo con il teorema di fattorizzazione, risulta che possiamo scrivere la funzione di verosimiglianza come

$$L(\vartheta;\vec{x}) = h(\vec{x}) g(t_n(\vec{x}),\vartheta)$$

ove $h$ non dipende da $\vartheta$.
Quindi, il \textit{rapporto di verosimiglianze (RV)} può essere scritto come

$$RV(\vartheta_1,\vartheta_0; \vec{x}) = \frac{L(\vartheta_1;\vec{x})}{L(\vartheta_0;\vec{x})} = \frac{g(t_n(\vec{x}), \vartheta_1)}{g(t_n(\vec{x}), \vartheta_0)}$$

Segue che, avendo a disposizione una statistica sufficiente per $\vartheta$, tale rapporto dipende da $\vec{x}$ solo attraverso $t_n$.

Introduciamo il concetto di \textit{monotonia del rapporto di verosimiglianza}.

\begin{definizione}
Nelle ipotesi dell'osservazione qua sopra, nel caso in cui $\vartheta_1 < \vartheta_0$ ($\vartheta_1 > \vartheta_0$) diciamo che $RV(\vartheta_1,\vartheta_0; \vec{x})$ è monotono se esso è una funzione crescente (decrescente) di $t_n$. 
\end{definizione}

Intuitivamente, se il rapporto tra le verosimiglianze è monotono, è facile immaginare che il test $UMP_\alpha$ per ipotesi unilaterali esiste sempre!
\\
Ciò è sicuramente vero se ci restringiamo a famiglie esponenziali:
consideriamo la famiglia esponenziale nella sua forma più semplice, ossia con funzione di densità della forma
$$f_X(x;\theta) = \exp \{ A(\theta) B (x) +  C(x) +  D(\theta) \}$$

\begin{lemma}
Un campione casuale da una famiglia esponenziale con $A(\vartheta)$ monotona ammette $RV$ monotono.
\end{lemma}

\begin{proof}
Infatti, 
\begin{align*}
RV(\vartheta_1,\vartheta_0; \vec{x}) 
&= \frac{\exp \{ A(\vartheta_1) \sum_i B (x_i) + \sum_i C(x_i) +  n D(\vartheta_1) \}}{\exp \{  A(\vartheta_0) \sum_i B (x_i) +  \sum_i C(x_i) +  n D(\vartheta_0) \}}
\\&= \exp \{ [A(\vartheta_1) - A(\vartheta_0)] \cdot \sum_i B(x_i) + n[ D(\theta_1) -  D(\theta_0)] \}
\end{align*}

Essendo $A$ monotona, si nota subito che se $\vartheta_1 > \vartheta_0$ allora $A(\vartheta_1) - A(\vartheta_0)\geq 0$ e quindi $RV$ è una funzione crescente rispetto a $\sum_i B (x_i)$
\end{proof}

Da quanto visto sopra, segue immediatamente che, testando le ipotesi 

$$\bigg \{
\begin{array}{rl}
H_0: & \vartheta = \vartheta_0 \\
H_1: & \vartheta < \vartheta_0 \\
\end{array}
$$

preso un qualunque $\vartheta_1 < \vartheta_0$, la condizione 
$$RV(\vartheta_1,\vartheta_0;\vec{x}) \leq k$$

è equivalente a 

$$\sum_{i=1}^n B(x_i) \leq c$$ per una costante $c$ ricavabile facilmente dall'equazione sopra, e valido per ogni $\vartheta_1 < \vartheta_0$. Questo fornisce il tesi $UMP_\alpha$
per la famiglia esponenziale (in questo caso vista nella forma semplice) per test unilaterali. Ovviamente, nel caso in cui $H_1: \vartheta > \vartheta_0$, la regione critica sarà fornita da $\sum_{i=1}^n B(x_i) \geq c$.\\

\subsection{Rapporto di massime verosimiglianze}

Introduciamo un test valido in generale, con il defetto di esser senza garanzie (se non asintotiche) di ottimalità ma con il pregio di non aver bisogno di particolari forme d'ipotesi. 

\begin{definizione}
Sia $(X_1,...,X_n)$ un campione casuale da una distribuzione avente funzione di ripartizione $F(x,\vartheta)$, con $\vartheta \in \Theta$.
Supponiamo di avere un sistema di ipotesi
$$\bigg \{
\begin{array}{rl}
H_0: & \vartheta \in \Theta_0 \\
H_1: & \vartheta \in \Theta \setminus \Theta_0 \\
\end{array}
$$

Definiamo la funzione Rapporto di Verosimiglianza generalizzato (RV generalizzato) come la funzione $\lambda: \mathcal{X} \in \mathbb{R}^n \to [0,1]$
definita da
$$\lambda(\vec{x}) = \frac{\max_{\vartheta \in \Theta_0} L(\vartheta,\vec{x})}{\max_{\vartheta \in \Theta} L(\vartheta,\vec{x})}$$
\end{definizione}

$\lambda$ sta in $[0,1]$ poiché, essendo $L$ una funzione di densità, $\lambda \geq 0$, ed invece $\lambda \leq 1$ essendo $\Theta_0 \subset \Theta$.\\
\\
Fissato un $A \in (0,1)$, una regione di rifiuto del test è data da 
$$C := \{ \vec{x} \in \mathcal{X}: \lambda(\vec{x})<A \}$$
Ovviamente $A$ sarà scelto in modo che 
$$P(\lambda(\vec{x})<A | H_0) = \alpha$$
con $\alpha$ probabilità (fissata a priori) di commettere un errore del primo tipo.\\
\\
La potenza di questo metodo è data dal seguente

\begin{teo}
Sotto $H_0$ la statistica (test di Wald) $W=-2ln[\lambda(\vec{x})]$, converge in distribuzione (all'aumentare di n) a $\chi^2_\nu$, ove $\nu$ è la differenza tra i parametri da stimare sotto $H_1$ rispetto ad $H_0$.
\end{teo}

Quindi, per campioni numerosi, è possibile ricavare agevolmente il parametro $A$ dalle tavole della $\chi^2$.\\
\\
\noindent \textbf{Esempio:} Sia $(X_1,...,X_n)$ un campione casuale da Poisson con parametro $\theta$, e quindi $$f(x,\theta) = P(X_i = x) = \frac{e^{-\theta} \theta^x}{x!} \mathbb{I}_{(0,1,2,...)}(x)$$ e $\theta>0$.
$$\bigg \{
\begin{array}{rl}
H_0: & \theta = \theta_0  \\
H_1: & \theta \neq \theta_0 \ \ [\theta \in \mathbb{R}^+ \setminus \{\theta_0\}]\\
\end{array}
$$
Calcoliamo inoltre la funzione di verosimiglianza:
$$L(\theta,\vec{x})=\frac{e^{-n\theta} \theta^{\sum x_i}}{\prod_{i=1}^n x_i!}$$
troviamo che, posto $\hat{\theta} = \overline{X}_n$ lo stimatore di massima verosimiglianza
$$\lambda(\vec{x}) = \frac{\max_{\theta \in \Theta_0} L(\theta,\vec{x})}{\max_{\theta \in \Theta} L(\theta,\vec{x})} = \frac{L(\theta_0,\vec{x})}{L(\hat{\theta},\vec{x})}$$

(notare che $\max_{\theta \in \Theta} L(\theta,\vec{x})=L(\hat{\theta},\vec{x}) $ proprio per definizione di stimatore di massima verosimiglianza!)

$$\lambda(\vec{x}) = \dfrac{ e^{-n\theta_0} \theta_0^{\sum x_i}}{e^{-n\hat{\theta}} \hat{\theta}^{\sum x_i}} = 
\left( \frac{\theta_0}{\hat{\theta}} \right)^{\sum x_i} e^{n(\hat{\theta} - \theta_0)}$$

Usiamo la statistica test di Wald:
$W = -2 ln (\lambda (\vec{x})) =  -2\left[\sum x_i ln \left( \frac{\theta_0}{\hat{\theta}} \right) + n(\hat{\theta} - \theta_0) \right]$
la quale, sotto $H_0$, $W \stackrel{D}{\longrightarrow} \chi^2$.
La regione critica che cerchiamo è data da
$$C := \{ \vec{x} \in \mathcal{X}: \lambda(\vec{x})<A \}$$
Vale $\lambda(\vec{x})<A \Leftrightarrow -2ln[ \lambda(\vec{x})]> -2ln(A)$.
Fissato $\alpha$, se $n$ è abbastanza grande possiamo approssimare
 $$P(\lambda(\vec{x})<A \Bigm\vert H_0) = P(-2ln[ \lambda(\vec{x})]> -ln(A^2) \Bigm\vert H_0) \cong P(\chi^2_1> -ln(A^2)\Bigm\vert H_0) = \alpha$$ sse $$-ln(A^2) = \chi_{1;\alpha/2}^2$$ da cui 

\begin{align*}
C  &= \{ \vec{x} \in \mathcal{X}: -2\left[\sum x_i ln \left( \frac{\theta_0}{\hat{\theta}} \right) + n(\hat{\theta} - \theta_0) \right] >  \chi_{1;\alpha/2}^2 \}
\end{align*}
%\begin{ese}
  Dato il campione \((X_1,\dots,X_n)\) da \(b(1,p)\), vogliamo trovare una
  UMVVE? per
  \begin{equation*}
    \var(X) = p(1-p) = \eta(\phi), \uqad p \in (0,1)
  \end{equation*}
  sapendo che
  \begin{enumerate}
    \item 
      \begin{align*}
        L(p;\mathbf{x}) &= \prod_{i=1}^n
        p^{x_i}{(1-p)}^{1-x_i} \mathbb{I}_{\lbrace 0,1 \rbrace}(x_i) \\
        &= \underbrace{p^{\sum_{i=1}^n x_i}{(1-p)}^{n-\sum_{i=1}^n x_i}}_{g(p,\sum_{i=1}^n x_i)}
        \prod_{i=1}^n \mathbb{I}_{\lbrace 0,1 \rbrace}(x_i) \\
        &\implies T_n(X_1,\dots,X_n) = \sum_{i=1}^n X_i
      \end{align*}
      dove \(T_n\) è statistica sufficiente minimale e completa;
    \item
      \(\cap{p}_n = \frac{1}{n}\sum_{i=1}^n x_i\)
      è stimatore di massima verosimiglianza (MV) per \(p\);
    \item \(\cap{\eta}(p) = \eta(\cap{p}_n) =
            \cap{p}_n(1-\cap{p}_n) = \frac{1}{n^2}T_n(n-T_n)\).
          Possiamo dire che \(\eta(\cap{p}_n)\) è stimatore non distorto di \(\eta(p)\)?
          \begin{align*}
            \mathbb{E}_p(\frac{1}{n^2}T_n(n-T_n)) &=
            \frac{1}{n^2}\mathbb{E}_p(T_n(n-T_n)) \\
            &= \frac{1}{n^2}\left(n\mathbb{E}_p(T_n) - \mathbb{E}_p(T_n^2)) \\
            &= \frac{1}{n^2}(n\cdot np - (\underbrace{np(1-p)}_{\var(T_n)} + \underbrace{n^2p^2}_{(\mathbb{T_n})^2})) = \frac{n-1}{n}p(1-p) \ne p(1-p) 
          \end{align*}
          per cui risulta che lo stimatore è distorto per valori finiti di \(n\), mentre è asintoticamente non distorto.
          Introduciamo
          \begin{equation*}
            W_n = \frac{n}{n-1}\eta(\cap{p}_n),
          \end{equation*}
          che è uno stimatore non distorto di \(\eta(p)\), che è funzione di statistica suffiiente minimale.
          
          Allora, per il corollario del teorema di Rao-Blackwell, possiamo afferamre che \(W_n\) è \emph{lo} stimatore UMVU di \(\eta(p) = p(1-p)\), grazie anche alla completezza di \(T_n\).
  \end{enumerate}
\end{ese}

Lo stimatore di massima verosimiglianza è (sempre?) funzione di statistica sufficiente. Per questo, nella nostra ricerca dobbiamo considerarlo come passaggio obbligato. Vorremmo anche che lo stimatore fosse non distorto.

Riusciamo a dire qualcosa in più circa la distribuzione degli stimatori UMVU, usando le informazioni già in nostro possesso?

Riprendendo l'esempio di prima, abbiamo
\begin{equation*}
  \eta(p) = p(1-p) \implies
  \begin{cases}
    \cap{\eta}(p) = \cap{p}_n(1-\cap{p}_n)\, \text{massima verosimiglianza} \\
    \tilde{\eta}(p) = \frac{n}{n-1} \cap{p}_n(1-\cap{p}_n) \,\text{UMVU stimatore a minima varianza nella classe dei non distorti}
  \end{cases}
\end{equation*}
\begin{enumerate}
  \item in merito allo stimatore di massima verosimigliana, sappiamo che \(\cap{\eta}(p) = \eta(\cap{p}_n)\) è uno stimatore consistente di \(\eta(p)\) e che si ha quindi convergenza in probabilità;
  \item \(\sqrt{n}(\eta(\cap{p}_n) - \eta(p))\) è asintoticamente normale
  \item \(\tilde{\eta}(p) - \cap{\eta}(p)) = \frac{1}{n-1}\cap{\eta}(p) \to 0\) in probabilità, sicché \(\tilde{\eta}(p)\) è anch'esso stimatore consistente di \(\eta(p)\). Inoltre,
  \begin{equation*}
    \sqrt{n}(\tilde{\eta} - \eta) - \sqrt{n}(\cap{\eta} - \eta) =
    \frac{\sqrt{n}}{n-1}\cap{\eta} \to 0
  \end{equation*}
  in probabilità, perciò \(\sqrt{n}(\tilde{n}-n)\) ha la sts distribuzione asintotica di \(\sqrt{n}(\cap{n}-n)\).
  Ma \(p \to^D N(p, \frac{p(1-p)}{n})\) e
  \(\sqrt{n}(\cap{p}_n-p) \to^D N(0,p(1-p))\).
  Allora, \(\sqrt{n}(\cap{\eta}-\eta) \to^D N(0,(1-2p)^2p(1-p)).\)
  Non solo, \(\sqrt{n}(\tilde{\eta}-\eta)\to^D N(0,(1-2p)^2p(1-p))\).
\end{enumerate}

\begin{ese}
  Vediamo qualche sviluppo del teorema di Rao-Blackwell.
  Sia \((X_1,\dots,X_n)\) campione da \(\mathcal{P}(\lambda)\), con \(\lambda \in \mathbb{R}_+\). Sia
  \begin{equation*}
    \eta(\lambda) = \mathbb{P}_\lambda(X_1 > 0) = 1 - e^{-\lambda} = 1 - \mathbb{P}_\lambda(X_1 = 0).
  \end{equation*}
  Trovare uno stimatore UMVU di \(\eta(\lambda)\).
  
  Cosa ci serve per applicare Rao-Blackwell?
  \begin{enumerate}
    \item uno stimatore non distorto di \(\eta(\lambda)\), \(V_n\);
    \item una statistica sufficiente (e completa per applicare Scheffer) per \(\lambda\), \(T_n\);
    \item a questo punto,
    \begin{equation*}
      V_{n;T_n} = \mathbb{E}_\lambda{V_n \mid T_n = t_n}
    \end{equation*}
  \end{enumerate}ù
  Sappiamo già che \(T_n = \sum_{i=1}^n X_i\) è statistica sufficiente, minimale e completa per \(\lambda\).
  \begin{equation*}
    V_n := \mathbb{I}_{\lbrace X_1 > 0 \rbrace} (x_i) =
    \begin{cases}
      1 & X_1 > 0 \\ 0 & X_1 = 0
    \end{cases}
  \end{equation*}
  Ricordiamo che la funzione indicatore è una variabile casuale la cui media è pari a \(\mathbb{P}_\lambda(X_1 > 0)\).
  \begin{equation*}
    \mathbb{E}(V_n) = \sum_{x_1 = 1}^{\infty}\frac{e^{-\lambda}\lambda^{x_1}}{x_1!} = 1 - e^{\lambda}. 
  \end{equation*}
  Per cui \(V_n\) è stimatore non distorto.
  Allora,
  \begin{equation*}
    V_{n;T_n} = \mathbb{E}_\lambda{\mathbb{I}_{\lbrace X_1 > 0 \rbrace} \mid T_n} = \\
    1 - \mathbb{P}_\lambda{X_1 = 0 \mid T_n = t_n} = \\
    1 - \frac{\mathbb{P}_\lambda{X_1 = 0, T_n = t_n}}{\mathbb{P}_\lambda{T_n = t_n}} \\
    = 1 - \frac{\mathbb{P}_\lambda(X_1 = 0)\mathbb{P}_\lambda(\sum_{i=2}^n X_i = t_n)}{\mathbb{P}_\lambda(T_n=t_n)} \\
    \implies V_{n;T_n} = 1 - {(\frac{n-1}{n})}^{t_n}
  \end{equation*}
  è stimatore UMVU per \(\eta{\lambda}\).
  
  Uno stimatore di massima verosimiglianza per \(\eta(\lambda)\) è \(\cap{\eta}(\lambda) = \eta(\cap{\lambda}_n) = 1-e^{-\cap{\lambda}}\) dove
  \(\cap{\lambda}_n = \frac{1}{n}\sum_{i=1}^nX_i
\end{ese}
%%%%%%%%%%%%%%%%%%%%


\end{document}
